
\chapter{Changes in 0.7.3}
\label{ch:073changes}

\section{Commenting out of dataset records}
\label{sec:ignore}
Upon request within IOG, this version has been extended 
to support filtering out of rows in a datasets by allowing defining 
'ignore line' characters. For example $\#$ is often used for such 
purpose in NONMEM and Monolix-type datasets. 

These character(s) can contain up to three arbitrary characters 
without white spaces. The following snippet shows two such specifications, 
where the various symbols are defined, using <IgnoreLine> element, 
within the definition of the dataset, to identify rows to be omitted

\lstset{language=XML}
\begin{lstlisting}
    <DataSet xmlns="http://www.pharmml.org/pharmml/0.7/Dataset">
        <Definition>
            <Column columnId="TIME" columnType="idv" valueType="real" columnNum="1"/>
            <!-- skipped columns -->
            <Column columnId="MDV" columnType="mdv" valueType="int" columnNum="5"/>
            <IgnoreLine symbol="$10"/>
            <IgnoreLine symbol="#"/>
        </Definition>
        <ExternalFile oid="datasetID">
            <path>dataset.csv</path>
        </ExternalFile>
    </DataSet>
\end{lstlisting}
Any number of such 'ignore line' characters can be defined.

\section{\xatt{linkFunction} in discrete data models}
\label{sec:link}

The \xatt{linkFunction} attribute has been renamed to the (optional) \xatt{transform}
in the \xelem{PMF} element of the count data block. It was erroneously associated
with the probability mass function. Instead the transformation can be used to 
indicate that a transformed PMF has been encoded to be passed in such form to
a target tool, if supported. 

\paragraph{Example}
The common Poisson model can be implemented in its 
\begin{itemize}
\item 
default, untransformed form
\lstset{language=XML}
\begin{lstlisting}
<ObservationModel blkId="om1">
    <Discrete>
        <CountData>
            <CountVariable symbId="Y"/>
            <NumberCounts symbId="k"/>
            
	    <!-- P(Y=k) = (Lambda^k * exp(-Lambda) / k! -->
            <PMF>
                <math:LogicBinop op="eq">
                    <ct:SymbRef symbIdRef="Y"/>
                    <ct:SymbRef symbIdRef="k"/>
                </math:LogicBinop>
                <Distribution>
                    <po:ProbOnto name="Poisson1">
                        <po:Parameter name="rate">
                            <ct:Assign>
                                <ct:SymbRef symbIdRef="lambda"/>
                            </ct:Assign>
                        </po:Parameter>
                    </po:ProbOnto>
                </Distribution>
            </PMF>
\end{lstlisting}

\item 
in a log-transformed form
\lstset{language=XML}
\begin{lstlisting}
            <!-- log(P(Y=k)) = -Lambda+k*log(Lambda)-log(k!) -->
            <PMF transform="log">
                <math:LogicBinop op="eq">
                    <ct:SymbRef symbIdRef="Y"/>
                    <ct:SymbRef symbIdRef="k"/>
                </math:LogicBinop>
                <Distribution>
                    <po:ProbOnto name="Poisson1">
                        <po:Parameter name="rate">
                            <ct:Assign>
                                <ct:SymbRef symbIdRef="lambda"/>
                            </ct:Assign>
                        </po:Parameter>
                    </po:ProbOnto>
                </Distribution>
            </PMF>
\end{lstlisting}

\item 
or in log-transformed explicit form
\lstset{language=XML}
\begin{lstlisting}
            <!-- explicit log(PMF) -->
            <!-- log(P(Y=k)) = -Lambda+k*log(Lambda)-log(k!) -->
            <PMF transform="log">
                <math:LogicBinop op="eq">
                    <ct:SymbRef symbIdRef="Y"/>
                    <ct:SymbRef symbIdRef="k"/>
                </math:LogicBinop>
                <ct:Assign>
                    <math:Binop op="minus">
                        <math:Binop op="plus">
                            <math:Uniop op="minus">
                                <ct:SymbRef blkIdRef="pm1" symbIdRef="lambda"/>
                            </math:Uniop>
                            <math:Binop op="times">
                                <ct:SymbRef symbIdRef="k"/>
                                <math:Uniop op="log">
                                    <ct:SymbRef blkIdRef="pm1" symbIdRef="lambda"/>
                                </math:Uniop>
                            </math:Binop>
                        </math:Binop>
                        <math:Uniop op="factln">
                            <ct:SymbRef symbIdRef="k"/>
                        </math:Uniop>
                    </math:Binop>
                </ct:Assign>
            </PMF>
        </CountData>
    </Discrete>
</ObservationModel>
\end{lstlisting}
\end{itemize}
Note, that the last implementation should be used only if a certain 
distribution is not covered by ProbOnto/UncertML. It is the option allowing
the implementation of an arbitrary user-defined distributions.

