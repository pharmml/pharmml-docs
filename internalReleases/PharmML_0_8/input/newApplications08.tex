
\chapter{New applications using 0.8}
\label{ch:074applications}

\section{Baseline models}
\label{sec:baselineMapping}

Hansson model \cite{Hansson:2013yq} is a complex model
and particularly interesting from the interoperability perspective. 
It contains number of component, both continuous and discrete, such as 
\begin{itemize}
\item 
Biomarker model -- ODEs and algebraic equations 
\item
Model for tumor growth inhibition -- ODEs and algebraic equations 
including a baseline model, based on \cite{dansirikul2008approaches}
\item
Dropout model -- logistic regression model (simulation only)
\item
Survival model -- time-to-event data model 
\end{itemize}
Here we discuss the baseline model because of its structure not
discussed so far in any of the use cases. It requires the implementation of
mapping of the dependent variable at time t=0.
The NMTRAN code for the baseline model elements reads
\lstset{language=NM}
\begin{lstlisting}
	 IF(TIME.EQ.0.AND.FLAG.EQ.4)THEN
	   OBASE  =    DV                        		; observed tumor size at baseline (T= 0)
	 ENDIF
	
	 W1       =    THETA(4)*OBASE
	 IBASE    =    OBASE+ETA(5)*W1		; observed tumor size at baseline acknowledging residual error
	 ...
	 A_0(4) = IBASE                            ; TUMOR
	 
\end{lstlisting}
PharmML implementation of the baseline is done in two steps, first
by declaring the variable (could also be declared as covariate), i.e.
\lstset{language=XML}
\begin{lstlisting}
            <ct:Variable symbolType="real" symbId="OBASE"/>
\end{lstlisting}
which then can be mapped to the dependent variable, DV, column in the
dataset conditional on TIME==0 as the following snippet shows
\lstset{language=XML}
\begin{lstlisting}
    <ColumnMapping>
        <ColumnRef xmlns="http://www.pharmml.org/pharmml/0.7/Dataset" columnIdRef="DV"/>
        <Piecewise xmlns="http://www.pharmml.org/pharmml/0.7/Dataset">
            <math:Piece>
                <ct:SymbRef blkIdRef="sm1" symbIdRef="OBASE"/>
                <math:Condition>
                    <math:LogicBinop op="eq">
                        <ColumnRef columnIdRef="TIME"/>
                        <ct:Real>0</ct:Real>
                    </math:LogicBinop>
                </math:Condition>
            </math:Piece>
        </Piecewise>
    </ColumnMapping>
\end{lstlisting}
Once this is done, the variable OBASE can be used to define the 
initial condition for the tumor growth variable defined by an ODE, $A4$,
\begin{align}
\frac{A4}{dt} &= \mbox{KG} \mbox{A4}-[\mbox{AUC1}+(-\mbox{SKIT})+(-\mbox{VEG3})] \exp(-(\mbox{LAMBDA}\times \mbox{T}))\mbox{A4} \nonumber \\
A4(t=0)	& = \mbox{IBASE} = \mbox{OBASE}+\mbox{ETA(5)}*\mbox{W1} \nonumber % + \mbox{W1}\times \eta_5 \nonumber \\
% \mbox{with} \quad W1 & = \mbox{THETA4} \times \mbox{OBASE} \nonumber
\end{align}


\lstset{language=XML}
\begin{lstlisting}
    <ct:Variable symbolType="real" symbId="W1">
        <ct:Assign>
            <math:Binop op="times">
                <ct:SymbRef blkIdRef="pm1" symbIdRef="theta4"/>
                <ct:SymbRef symbIdRef="OBASE"/>
            </math:Binop>
        </ct:Assign>
    </ct:Variable>
    
    <!-- initial condition value, IBASE -->
    <!-- IBASE = A4(t=0) = OBASE+ETA(5)*W1 -->
    <ct:Variable symbolType="real" symbId="IBASE">
        <ct:Assign>
            <math:Binop op="plus">
                <ct:SymbRef blkIdRef="sm1" symbIdRef="OBASE"/>
                <math:Binop op="times">
                    <ct:SymbRef symbIdRef="W1"/>
                    <ct:SymbRef blkIdRef="pm1" symbIdRef="eta5"/>
                </math:Binop>
            </math:Binop>
        </ct:Assign>
    </ct:Variable>
    
    <!-- dA4/dt -->
    <ct:DerivativeVariable symbolType="real" symbId="A4">
        <ct:Assign>
            <!-- skipped RHS expression: -->
            <!-- KG*A(4)-[AUC1+(-SKIT)+(-VEG3)]*EXP(-(LAMBDA*T))*A(4) -->
        </ct:Assign>
        <ct:InitialCondition>
            <ct:InitialValue>
                <ct:Assign>
                    <ct:SymbRef blkIdRef="pm1" symbIdRef="IBASE"/>
                </ct:Assign>
            </ct:InitialValue>
            <ct:InitialTime>
                <ct:Assign><ct:Real>0</ct:Real></ct:Assign>
            </ct:InitialTime>
        </ct:InitialCondition>
    </ct:DerivativeVariable>
\end{lstlisting}


%\section{Mapping overview}
%\label{sec:mappingOverview}
%example4\_NONMEM.xml
%\lstset{language=XML}
%\begin{lstlisting}
%            <ColumnMapping>
%                <ds:ColumnRef columnIdRef="OCC"/>
%                <ct:SymbRef blkIdRef="cm1" symbIdRef="Occasion"/>
%                <ds:CategoryMapping>
%                    <ds:Map modelSymbol="occ1" dataSymbol="1"/>
%                    <ds:Map modelSymbol="occ2" dataSymbol="2"/>
%                </ds:CategoryMapping>
%            </ColumnMapping>
%\end{lstlisting}
%
%\begin{figure}[ht!]
%\centering
%  \includegraphics[width=120mm]{pics/mapping1.pdf}
% \caption{....}
% \label{fig:mappingDV}
%\end{figure}


\section{Markov models}
\label{sec:markovModels}
Stimulated by the recent discussion on the DDMoRe forum around Markov models
two examples are discussed with a detailed description and implementation. \marginpar{ADD URL}

It turns out that they have, beyond for example disease modelling demonstrated 
in 2nd example, other quite unexpected yet relevant application such as zombie 
attack modelling. These attacks have been documented in well-known movies 
such as \emph{Shaun of the Dead} although there many who see this as pure science
fiction. It is regardless an educational use case described in the first example.

\subsection{Example 1 -- Zombie attack}
\label{subsec:exp1}
This basic zombie attack model is based on lecture notes on Markov 
Models\footnote{http://www.poritz.net/jonathan/matvec/markov.html}.
More advanced models are available as well, e.g. \cite{munz2009zombies}, 
\cite{witkowski2013bayesian}, \cite{woolley2014long}.

\begin{figure}[ht!]
\centering
  \includegraphics[width=120mm]{pics/MarkovZombie.pdf}
 \caption{Basic Markov model for zombie attack.}
 \label{fig:MarkovZombie}
\end{figure}


\subsection*{Model definition}
\subsubsection*{Observation model}

\begin{itemize}
\item
Type of observed variable -- discrete / categorical
\item
Category variable: $Y$
\item
Initial state variable: $Y_{init}$
\item
Set of categories: $\{\mbox{Human}, \mbox{Zombie}\}$
\item
Transition probabilities
\begin{itemize}
\item
as pairwise conditional transition probabilities
\begin{align}
& P(\mbox{Human} \rightarrow \mbox{Zombie}) = 0.8 \nonumber \\
& P(\mbox{Zombie} \rightarrow \mbox{Human}) = 0.01 \nonumber \\
& P(\mbox{Human} \rightarrow \mbox{Human}) = 0.2 \nonumber \\
& P(\mbox{Zombie} \rightarrow \mbox{Zombie}) = 0.99 \nonumber
\end{align}
\item
or as transition (aka stochastic) matrix
\[
\begin{blockarray}{ccc}
& H & Z \\
\begin{block}{c[cc]}
H &    0.1 & 0.8  \\
Z &    0.01 & 0.99  \\
\end{block}
\end{blockarray}
\]
\end{itemize}
\end{itemize}

\subsection*{Trial Design}

\begin{itemize}
\item
Observations: Y at t=1,...,12 (months).
\end{itemize}


\subsection*{Modelling steps}

\begin{itemize}
\item
Initial states
\begin{align}
& Y_{init} = \left( \begin{array}{c} 100 \\ 0 \end{array} \right) \nonumber
\end{align}
\end{itemize}


\subsubsection*{PharmML implementation}

\lstset{language=XML}
\begin{lstlisting}
    <ModelDefinition xmlns="http://www.pharmml.org/pharmml/0.8/ModelDefinition">
        <!-- OBSERVATIONS -->
        <ObservationModel blkId="om1">
            <Discrete>
                <CategoricalData>
                    
                    <ListOfCategories> 
                        <Category symbId="Human"/>
                        <Category symbId="Zombie"/>
                    </ListOfCategories>
                    
                    <CategoryVariable symbId="Y"/>
                    <InitialStateVariable symbId="Yinit"/> 
                    <PreviousStateVariable symbId="Yp"/>
                    <Dependance type="discreteMarkov"/>
                    
                    <TransitionMatrix type="leftStochastic">
                        <ct:Matrix matrixType="Any">
                            <ct:RowNames>
                                <ct:SymbRef symbIdRef="Human"/><ct:SymbRef symbIdRef="Zombie"/>
                            </ct:RowNames>
                            <ct:MatrixRow>
                                <ct:Real>0.2</ct:Real><ct:Real>0.8</ct:Real>
                            </ct:MatrixRow>
                            <ct:MatrixRow>
                                <ct:Real>0.01</ct:Real><ct:Real>0.99</ct:Real>
                            </ct:MatrixRow>
                        </ct:Matrix>
                    </TransitionMatrix>
                    
                    <!-- ALTERNATIVELY usign Pairwise probabilities -->
                    <!-- P(Y=Zombie|Yp=Human)=0.8 -->
                    <ProbabilityAssignment>
                        <Probability symbId="p1">
                            <CurrentState>
                                <math:LogicBinop op="eq">
                                    <ct:SymbRef symbIdRef="Y"/>
                                    <ct:SymbRef symbIdRef="Zombie"/>
                                </math:LogicBinop>
                            </CurrentState>
                            <PreviousState>
                                <math:LogicBinop op="eq">
                                    <ct:SymbRef symbIdRef="Yp"/>
                                    <ct:SymbRef symbIdRef="Human"/>
                                </math:LogicBinop>
                            </PreviousState>
                        </Probability>
                        <ct:Assign>
                            <ct:Real>0.8</ct:Real>
                        </ct:Assign>
                    </ProbabilityAssignment>
                    <!-- other probabilities analog - skipped here -->
                    
                </CategoricalData>
            </Discrete>
        </ObservationModel>
    </ModelDefinition>
\end{lstlisting}
Trial design: output of variable $Y$ at t=1,...,12 (months).
\lstset{language=XML}
\begin{lstlisting}
    <!-- OBSERVATION DEFINITION: Number of humans/zombies for months 1-12 -->
    <TrialDesign xmlns="http://www.pharmml.org/pharmml/0.8/TrialDesign">
        <Observations>
            <Observation oid="obsOid">
                <ObservationTimes>
                    <ct:Assign>
                        <ct:Sequence>
                            <ct:Begin>
                                <ct:Real>1</ct:Real>
                            </ct:Begin>
                            <ct:StepSize>
                                <ct:Real>1</ct:Real>
                            </ct:StepSize>
                            <ct:End>
                                <ct:Real>12</ct:Real>
                            </ct:End>
                        </ct:Sequence>
                    </ct:Assign>
                </ObservationTimes>
                <Discrete>
                    <ct:SymbRef blkIdRef="om1" symbIdRef="Y"/>
                </Discrete>
            </Observation>
        </Observations>
    </TrialDesign>
\end{lstlisting}
Modelling step definition and initial assignments, $Y_{init} = (100, 0)$
\lstset{language=XML}
\begin{lstlisting}
    <mstep:ModellingSteps>
        <mstep:SimulationStep oid="simOid">
            
            <mstep:ObservationsReference>
                <ct:OidRef oidRef="obsOid"/>
            </mstep:ObservationsReference>
            
            <ct:VariableAssignment>
                <ct:SymbRef blkIdRef="om1" symbIdRef="Yinit"/>
                <ct:Assign>
                    <ct:Vector>
                        <ct:VectorElements>
                            <ct:Real>100</ct:Real>
                            <ct:Real>0</ct:Real>
                        </ct:VectorElements>
                    </ct:Vector>
                </ct:Assign>
            </ct:VariableAssignment>
            
            <mstep:Operation order="1" opType="Number of humans/zombies for months 1-12"/>
        </mstep:SimulationStep>
    </mstep:ModellingSteps>
\end{lstlisting}


\subsubsection{Scientific background}
Tara Smith, an infectious disease professor at the University of Iowa, uses the 
paper, \cite{munz2009zombies}, to show how math models can predict the 
effects of quarantines, vaccines and other public health measures.
\begin{quotation}
The zombie model's methods have already proved useful in at least one real-life 
analysis. While working on a model of HPV (human papillomavirus), Robert Smith's 
team noted that transmission via both gay and straight sex introduced two 
nonlinear variables to the equation. Fortunately, the zombie model had already 
blazed this path, demonstrating how to handle multiple nonlinear factors.
\end{quotation}
More on: \url{http://www.livescience.com/38527-surviving-a-zombie-apocalypse-math.html}




%\subsection{Example 2 -- Three State Markov Model}
%\label{subsec:exp2}
%
%The next example is build around a Markov chain for three states/categories
%which is shown in the figure \ref{fig:example2} and expressed by a set
%of pair-wise transition probabilities or a so-called \textbf{left} transition 
%(aka stochastic) matrix, on the right in the figure.
%\begin{figure}[ht!]
%\centering
%  \includegraphics[width=120mm]{pics/example2.pdf}
% \caption{Example 2 can be expressed by the basic Markov chain with the
% transition probabilities given below or the transition matrix (right).}
% \label{fig:example2}
%\end{figure}
%
%
%\subsection*{Model definition}
%\subsubsection*{Observation model}
%
%\begin{itemize}
%\item
%Type of observed variable -- discrete / categorical
%\item
%Category variable: $Y$
%\item
%Initial state variable: $Y_{init}$
%\item
%Set of categories: $\{\mbox{Healthy}, \mbox{Sick}, \mbox{Dead}\}$
%\item
%Transition probabilities
%\begin{itemize}
%\item
%as pairwise conditional transition probabilities
%\begin{align}
%& P(\mbox{Healthy} \rightarrow \mbox{Dead}) = 0.1 \nonumber \\
%& P(\mbox{Healthy} \rightarrow \mbox{Sick}) = 0.2 \nonumber \\
%& P(\mbox{Sick} \rightarrow \mbox{Healthy}) = 0.1 \nonumber \\
%& P(\mbox{Sick} \rightarrow \mbox{Dead}) = 0.3 \nonumber
%\end{align}
%\item
%or as transition (aka stochastic) matrix
%%\[
%%\begin{bmatrix}
%%    1-0.2-0.1 & 0.2 & 0.01  \\
%%    0.1 & 1-0.1-0.3 & 0.03  \\
%%    0 & 0 & 1  \\
%%\end{bmatrix}
%%\]
%\[
%\begin{blockarray}{cccc}
%& H & S & D \\
%\begin{block}{c[ccc]}
%H &    1-0.2-0.1 & 0.2 & 0.1  \\
%S &    0.1 & 1-0.1-0.3 & 0.3  \\
%D &    0 & 0 & 1  \\
%\end{block}
%\end{blockarray}
%\]
%
%\end{itemize}
%\end{itemize}
%
%
%\subsection*{Trial Design}
%
%\begin{itemize}
%\item
%Observations: Y at t=1,...,10.
%\end{itemize}
%
%
%\subsection*{Modelling steps}
%
%\begin{itemize}
%\item
%Initial states
%\begin{align}
%& Y_{init} = \left( \begin{array}{c} 100 \\ 0 \\ 0 \end{array} \right) \nonumber
%\end{align}
%\end{itemize}
%
%
%
%\subsection*{PharmML implementation}
%The model is a straightforward extension of example 1 for three categories and should
%not be shown here except the transition matrix which can be used instead of
%the transition probabilities 
%\lstset{language=XML}
%\begin{lstlisting}
%<TransitionMatrix>
%    <ct:Matrix matrixType="Any">
%        <ct:ColumnNames>
%            <ct:SymbRef symbIdRef="Healthy"/>
%            <ct:SymbRef symbIdRef="Sick"/>
%            <ct:SymbRef symbIdRef="Dead"/>
%        </ct:ColumnNames>
%        <ct:MatrixRow>
%            <ct:Real>0.7</ct:Real><ct:Real>0.2</ct:Real><ct:Real>0.1</ct:Real>
%        </ct:MatrixRow>
%        <ct:MatrixRow>
%            <ct:Real>0.1</ct:Real><ct:Real>0.6</ct:Real><ct:Real>0.3</ct:Real>
%        </ct:MatrixRow>
%        <ct:MatrixRow>
%            <ct:Real>0</ct:Real><ct:Real>0</ct:Real><ct:Real>0.1</ct:Real>
%        </ct:MatrixRow>
%    </ct:Matrix>
%</TransitionMatrix>
%\end{lstlisting}
%

\subsection{Example 2 - HIV model}
\label{subsec:exp3}
\cite{Lee:2014kq}

\begin{figure}[ht!]
\centering
  \includegraphics[width=140mm]{pics/Markov_HIV.pdf}
 \caption{Markov model HIV.}
 \label{fig:Markov_HIV}
\end{figure}

\subsection{Example 3 -- Stratified Markov Model}
\label{subsec:exp3}

The transition probabilities vary now conditional on the sex of 
the subjects. For example the probability for a healthy female to 
get sick is lower, 0.1, then for a male, 0.2. See Figure \ref{fig:markovMatrixCond}
for the Markov chain network and transition matrix for this model.

\begin{figure}[ht!]
\centering
  \includegraphics[width=120mm]{pics/markovMatrixCond.pdf}
 \caption{Example 3 -- a transition matrix defined conditional on sex of the subjects.}
 \label{fig:markovMatrixCond}
\end{figure}

\subsection*{Model definition}

\subsubsection*{Covariate model}
\begin{itemize}
\item
Race = \{African Americans, Caucasians\} -- categorical covariate
\end{itemize}

\subsubsection*{Observation model}

\begin{itemize}
\item
Type of observed variable -- discrete / categorical
\item
Category variable: $Y$
\item
Initial state variable: $Y_{init}$
\item
Set of categories: $\{\mbox{Healthy}, \mbox{Sick}, \mbox{Dead}\}$
\item
Transition probabilities
\begin{itemize}
\item
as pairwise conditional transition probabilities
\begin{align}
& P(\mbox{Healthy} \rightarrow \mbox{Sick} | \; SEX==F) = 0.1 \nonumber \\
& P(\mbox{Healthy} \rightarrow \mbox{Sick} | \; SEX==M) = 0.2 \nonumber \\
& P(\mbox{Sick} \rightarrow \mbox{Healthy}) = 0.1 \nonumber \\
& P(\mbox{Sick} \rightarrow \mbox{Dead}) = 0.3 \nonumber \\
& \mbox{other probabilities follow from the property 'left stochastic matrix'} \nonumber
\end{align}
\item
or as transition (aka stochastic) matrix
\[
\begin{blockarray}{cccc}
& H & S & D \\
\begin{block}{c[ccc]}
H &	0.7 & \left\{\begin{array}{ll}
      	0.1 & SEX== F \\
      	0.2 & SEX== M \\
\end{array}\right. & 0.1  \\\\
S &    0.1 & 0.6 & 0.3  \\\\
D &    0 & 0 & 1  \\
\end{block}
\end{blockarray}
\]
\end{itemize}
\end{itemize}

\subsection*{Trial Design}

\begin{itemize}
\item
Observations: Y at t=1,...,10.
\end{itemize}

\subsection*{Modelling steps}

\begin{itemize}
\item
Initial states
\begin{align}
& Y_{init} = \left( \begin{array}{c} 100 \\ 0 \\ 0 \end{array} \right) \nonumber
\end{align}
\end{itemize}

\subsection*{PharmML implementation}
Most of the model elements is identical to that of example in section \ref{subsec:exp2}
except the conditional transition probabilities, which XML encoding will be shown
here only.  The probabilities can be expresses pairwise or a transition matrix as the following 
code snippet show.

\begin{itemize}
\item
Pairwise probabilities (only the conditional ones are shown for brevity)

\lstset{language=XML}
\begin{lstlisting}
        <!-- P(Yp=Healthy -> Y=Sick | SEX=F)=0.1 -->
        <ProbabilityAssignment>
            <Probability>
                <CurrentState>
                    <math:LogicBinop op="eq">
                        <ct:SymbRef symbIdRef="Y"/>
                        <ct:SymbRef symbIdRef="Dead"/>
                    </math:LogicBinop>
                </CurrentState>
                <PreviousState>
                    <math:LogicBinop op="eq">
                        <ct:SymbRef symbIdRef="Yp"/>
                        <ct:SymbRef symbIdRef="Sick"/>
                    </math:LogicBinop>
                </PreviousState>
                <Condition>
                    <math:LogicBinop op="eq">
                        <ct:SymbRef symbIdRef="SEX"/>
                        <ct:CatRef catIdRef="F"/>
                    </math:LogicBinop>
                </Condition>
            </Probability>
            <ct:Assign>
                <ct:Real>0.1</ct:Real>
            </ct:Assign>
        </ProbabilityAssignment>
        
        <!-- P(Yp=Healthy -> Y=Sick | SEX=M)=0.1 -->
        <ProbabilityAssignment>
            <Probability>
                <CurrentState>
                    <math:LogicBinop op="eq">
                        <ct:SymbRef symbIdRef="Y"/>
                        <ct:SymbRef symbIdRef="Sick"/>
                    </math:LogicBinop>
                </CurrentState>
                <PreviousState>
                    <math:LogicBinop op="eq">
                        <ct:SymbRef symbIdRef="Yp"/>
                        <ct:SymbRef symbIdRef="Healthy"/>
                    </math:LogicBinop>
                </PreviousState>
                <Condition>
                    <math:LogicBinop op="eq">
                        <ct:SymbRef symbIdRef="SEX"/>
                        <ct:CatRef catIdRef="M"/>
                    </math:LogicBinop>
                </Condition>
            </Probability>
            <ct:Assign>
                <ct:Real>0.2</ct:Real>
            </ct:Assign>
        </ProbabilityAssignment>
\end{lstlisting}

\item
Transition matrix -- we use the fact that a matrix element can contain an arbitrary expression,
also a piecewise function used here for the transition probability from the \emph{Healthy} to 
\emph{Sick} state conditioned on the covariate \emph{Sex}. 
\lstset{language=XML}
\begin{lstlisting}
        <TransitionMatrix symbId="T1">
            <ct:Matrix matrixType="Any">
                <ct:ColumnNames>
                    <ct:SymbRef symbIdRef="Healthy"/>
                    <ct:SymbRef symbIdRef="Sick"/>
                    <ct:SymbRef symbIdRef="Dead"/>
                </ct:ColumnNames>
                <ct:MatrixRow>
                    <ct:Real>0.7</ct:Real>
                    <ct:Assign>
                        <ct:Piecewise>
                            <math:Piece>
                                <ct:Real>0.8</ct:Real>
                                <math:Condition>
                                    <math:LogicBinop op="eq">
                                        <ct:SymbRef symbIdRef="Sex"/>
                                        <ct:CatRef catIdRef="F"/>
                                    </math:LogicBinop>
                                </math:Condition>
                            </math:Piece>
                            <math:Piece>
                                <ct:Real>0.7</ct:Real>
                                <math:Condition>
                                    <math:LogicBinop op="eq">
                                        <ct:SymbRef symbIdRef="Sex"/>
                                        <ct:CatRef catIdRef="M"/>
                                    </math:LogicBinop>
                                </math:Condition>
                            </math:Piece>
                        </ct:Piecewise>
                    </ct:Assign>
                    <ct:Real>0.1</ct:Real>
                </ct:MatrixRow>
                <ct:MatrixRow>
                    <ct:Real>0.1</ct:Real><ct:Real>0.6</ct:Real><ct:Real>0.3</ct:Real>
                </ct:MatrixRow>
                <ct:MatrixRow>
                    <ct:Real>0</ct:Real><ct:Real>0</ct:Real><ct:Real>0.1</ct:Real>
                </ct:MatrixRow>
            </ct:Matrix>
        </TransitionMatrix>
\end{lstlisting}
\end{itemize}
Note, that the order of probabilities is defined by the order of categories/states names in the \xelem{ColumnNames}.
Because the transition matrix is symmetric specifying only the column names is sufficient. 

\subsection{Example 4 -- Micro Simulation Model}
\label{subsec:exp4}

\subsection*{Model definition}

\subsubsection*{Covariate model}
\begin{itemize}
\item
Age -- continuous covariate
\item
Sex = \{F, M\} -- categorical covariate
\end{itemize}
Individual covariates are stored in the trial design, see below.

\subsubsection*{Observation model}

\begin{itemize}
\item
Type of observed variable -- discrete / categorical
\item
Category variable: $Y$
\item
Initial state variable: $Y_{init}$
\item
Set of categories: $\{\mbox{Healthy}, \mbox{Sick}, \mbox{Dead}\}$
\item
Transition probabilities
\begin{itemize}
\item
as pairwise conditional transition probabilities
%
%Healthy to Dead: Age/1000\\
%Healthy to Sick: 0.1* (1 + Male) + 0.01*Age and never more than 0.8\\
%Sick to Healthy 0.1\\
%Sick to Dead:  0.01*Age+0.2*Male and never more than 0.9
\begin{align}
& P(\mbox{Healthy} \rightarrow \mbox{Dead}) = \mbox{Age}/100 \nonumber \\
p_{HS} := \quad & P(\mbox{Healthy} \rightarrow \mbox{Sick} | \; p_{HS} < 0.8) = 0.1(1 + \mbox{Male}) + 0.01\times\mbox{Age} \nonumber \\
& P(\mbox{Sick} \rightarrow \mbox{Healthy}) = 0.1 \nonumber \\
p_{SD} := \quad & P(\mbox{Sick} \rightarrow \mbox{Dead} | \; p_{SD} < 0.9) = 0.01\times\mbox{Age}+0.2\times\mbox{Male} \nonumber
\end{align}
\item
Transition matrix becomes complex with the additional boundaries
on the probabilities and it's probably a better to encode it pairwise
as above.
%\[
%T1 = 
%\begin{blockarray}{cccc}
%& \mbox{H} & \mbox{S} & \mbox{D} \\
%\begin{block}{c[ccc]}
%\mbox{H} & 0.7 	& 0.1(1 + \mbox{Male}) + 0.01\times\mbox{Age } \& \;p_{HS} < 0.8 & 0.1  \\\\
%\mbox{S} 	& 0.1 	& 0.6 & 0.01\times\mbox{Age}+0.2\times\mbox{Male } \& \;p_{SD} < 0.9  \\\\
%\mbox{D} 	& 0 		& 0 & 1  \\
%\end{block}
%\end{blockarray}
%%
%%\begin{bmatrix}
%%    0.7 & 0.2 & 0.01  \\
%%    0.1 & 0.6 & 0.03  \\
%%    0 & 0 & 1  \\
%%\end{bmatrix}
%\]
\end{itemize}
\end{itemize}



\subsection*{Trial Design}

\begin{itemize}
\item
Observations: Y at t=1,...,10.
\item
Covariates
\begin{itemize}
\item
continuous: Age = 1, 2, 3, 4, ..., 100

\item
categorical: Sex = \{F, M\} with Male = 0,1,0,1 .... (100 times)
\end{itemize}
\end{itemize}



\subsection*{Modelling steps}

\begin{itemize}
\item
Initial states
\begin{align}
& Y_{init} = \left( \begin{array}{c} 100 \\ 0 \\ 0 \end{array} \right) \nonumber
\end{align}
\end{itemize}

\subsection*{PharmML implementation}
The covariates are coming here as individual values, meaning that they
have to be provided as a table. This is when the \xelem{TrialDesign} is
very helpful as it is equipped with the required structure.
It is sufficient to declare the covariates in the \xelem{CovariateModel} within 
\xelem{ModelDefinition} as the following snippet shows
\lstset{language=XML}
\begin{lstlisting}
        <CovariateModel blkId="cm1">
            <Covariate symbId="AGE">
                <Continuous/>
            </Covariate>
            
            <Covariate symbId="MALE">
                <Categorical>
                    <Category catId="F"/>
                    <Category catId="M"/>
                </Categorical>
            </Covariate>
        </CovariateModel>
\end{lstlisting}
the proper values are stored then in the \xelem{IndividualCovariates} in 
\xelem{TrialDesign}
\lstset{language=XML}
\begin{lstlisting}
        <Covariates>
            <!--    Male = 0,1,0,1 .... (100 times)
                    Age = 1,2,3,4,...100-->
            <IndividualCovariates>
                <ColumnMapping>
                    <ds:ColumnRef columnIdRef="SEX"/>
                    <ct:SymbRef blkIdRef="cm1" symbIdRef="Sex"/>
                    <ds:CategoryMapping>
                        <ds:Map dataSymbol="1" modelSymbol="F"/>
                        <ds:Map dataSymbol="0" modelSymbol="M"/>
                    </ds:CategoryMapping>
                </ColumnMapping>
                <ColumnMapping>
                    <ds:ColumnRef columnIdRef="AGE"/>
                    <ct:SymbRef blkIdRef="cm1" symbIdRef="Age"/>
                </ColumnMapping>
                <ds:DataSet>
                    <ds:Definition>
                        <ds:Column columnId="ID" columnType="id" valueType="id" columnNum="1"/>
                        <ds:Column columnId="SEX" columnType="covariate" valueType="real" columnNum="2"/>
                        <ds:Column columnId="AGE" columnType="covariate" valueType="real" columnNum="3"/>
                    </ds:Definition>
                    <ds:Table>
                        <ds:Row>
                            <ct:Id>i1</ct:Id><ct:Real>0</ct:Real><ct:Real>1</ct:Real>
                        </ds:Row>
                        <ds:Row>
                            <ct:Id>i1</ct:Id><ct:Real>1</ct:Real><ct:Real>2</ct:Real>
                        </ds:Row>
                        <ds:Row>
                            <ct:Id>i1</ct:Id><ct:Real>0</ct:Real><ct:Real>3</ct:Real>
                        </ds:Row>
                        <ds:Row>
                            <ct:Id>i1</ct:Id><ct:Real>1</ct:Real><ct:Real>4</ct:Real>
                        </ds:Row>
                        <!-- omitted subjects -->
                        <ds:Row>
                            <ct:Id>i100</ct:Id><ct:Real>1</ct:Real><ct:Real>100</ct:Real>
                        </ds:Row>
                    </ds:Table>
                </ds:DataSet>
            </IndividualCovariates>
        </Covariates>
    </TrialDesign>
\end{lstlisting}
\xelem{ColumnMapping} elements are declared to provide the required information 
how to connect the model and the covariates.

From the model definition we show here only the implementation of one
probability, $P(\mbox{Sick} \rightarrow \mbox{Dead} | \; p_{SD} < 0.9)$, the remaining transition 
probabilities are analog.
\lstset{language=XML}
\begin{lstlisting}
        <!-- Healthy to Sick -->
        <!-- pHS:= P(y=Dead|yp=Healthy & pHS<0.8 = 0.1* (1 + Male) +  0.01*Age -->
        <ProbabilityAssignment>
            <Probability symbId="pHS">
                <CurrentState>
                    <math:LogicBinop op="eq">
                        <ct:SymbRef symbIdRef="y"/>
                        <ct:SymbRef symbIdRef="Sick"/>
                    </math:LogicBinop>
                </CurrentState>
                <PreviousState>
                    <math:LogicBinop op="eq">
                        <ct:SymbRef symbIdRef="yp"/>
                        <ct:SymbRef symbIdRef="Healthy"/>
                    </math:LogicBinop>
                </PreviousState>
                <Condition>
                    <math:LogicBinop op="lt">
                        <ct:SymbRef symbIdRef="pHS"/>
                        <ct:Real>0.8</ct:Real>
                    </math:LogicBinop>
                </Condition>
            </Probability>
            <ct:Assign>
                <math:Binop op="plus">
                    <math:Binop op="times">
                        <ct:Real>0.1</ct:Real>
                        <math:Binop op="plus">
                            <ct:Real>1</ct:Real>
                            <ct:SymbRef symbIdRef="Male"/>
                        </math:Binop>
                    </math:Binop>
                    <math:Binop op="times">
                        <ct:Real>0.01</ct:Real>
                        <ct:SymbRef symbIdRef="Age"/>
                    </math:Binop>
                </math:Binop>
            </ct:Assign>
        </ProbabilityAssignment>
\end{lstlisting}
Note, that the probability is assigned a symbol identifier \xatt{symbId="pHS"}
which is then used to express the condition $p\_{SD}<0.9$

\subsection*{Out of scope}
Compared to the original description there are model elements which 
definition will differ between the proposed models formulation and their  
PharmML implementation such as
\begin{itemize}
\item 
Definition of the simulation execution, here from the original description: \emph{ [...] each time 
step we want Age of each individual to increase by 1 before any transitions 
are executed: Age $\leftarrow$ Age + 1: This is executed for living individuals only}

Although it is not possible to define such simulation steps explicitly, it
is possible to encode them implicitly using covariate/regressors via lookup 
tables or other elements available in the \xelem{TrialDesign}. 
\item 
Output definition, here from the original description: \emph{[...] number of people in each 
state each year total and stratified by age groups above and below 50, and Male.} \\
Such analysis is assumed to be carried out by a downstream tool as it was 
not in the scope and requirements for PharmML.
\end{itemize}


%\section{Multiple infusions}
%
%Alternatively, you could use DesignParameter to encode the 
%amount vector
%\lstset{language=XML}
%\begin{lstlisting}
%		<mdef:DesignParameter symbId="doseAmountVector">
%			<ct:Assign>
%				<ct:Vector>
%					<ct:VectorElements>
%						<ct:Real>1</ct:Real>
%						<ct:Real>2</ct:Real>
%						<ct:Real>1</ct:Real>
%						<ct:Real>5</ct:Real>
%						<!-- ... -->
%					</ct:VectorElements>
%				</ct:Vector>
%			</ct:Assign>
%		</mdef:DesignParameter>
%\end{lstlisting}
%
%and then refer to it in (see in bold below)
%\lstset{language=XML}
%\begin{lstlisting}
%	<Interventions>
%		<Administration oid="inf1">
%			<Infusion>
%				<DoseAmount>
%					<ct:SymbRef symbIdRef="Ad"/>
%					<ct:Assign>
%						<math:Binop op="divide">
%							<math:Binop op="times">
%								<ct:SymbRef symbIdRef="doseAmountVector"/>
%								<ct:SymbRef blkIdRef="mmcvt" symbIdRef="wgt" />
%							</math:Binop>
%							<ct:SymbRef blkIdRef="mmpar" symbIdRef="V" />
%						</math:Binop>
%					</ct:Assign>
%				</DoseAmount>
%				<DosingTimes>
%					<ct:Assign>
%						<ct:Vector>
%							<ct:VectorElements>
%								<ct:Real>0</ct:Real>
%								<ct:Real>5</ct:Real>
%								<ct:Real>10</ct:Real>
%								<!-- ... -->
%							</ct:VectorElements>
%						</ct:Vector>
%					</ct:Assign>
%				</DosingTimes>
%				<Rate>
%					<ct:Assign>
%						<ct:Vector>
%							<ct:VectorElements>
%								<ct:Real>10</ct:Real>
%								<ct:Real>15</ct:Real>
%								<ct:Real>20</ct:Real>
%								<!-- ... -->
%							</ct:VectorElements>
%						</ct:Vector>	
%					</ct:Assign>
%				</Rate>
%				ALTERNATIVE 
%				<Duration>
%					<ct:Assign>
%						<ct:Vector>
%							<ct:VectorElements>
%								<ct:Real>10</ct:Real>
%								<ct:Real>15</ct:Real>
%								<ct:Real>20</ct:Real>
%								<!-- ... -->
%							</ct:VectorElements>
%						</ct:Vector>
%					</ct:Assign>
%				</Duration>
%			</Infusion>
%		</Administration>
%\end{lstlisting}
%















