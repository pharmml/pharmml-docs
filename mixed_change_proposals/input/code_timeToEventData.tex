%%%%%%%%%%%%%%%%%%%%%%%%%%%%%%%%%%%%%%%%%%%%%%%%%%%%%%%%%%%%%%%%
%%%%%%%%%%%%%%%%%%%%%%%%%%%%%%%%%%%%%%%%%%%%%%%%%%%%%%%%%%%%%%%%
\section{Time-to-event data models}
\label{subsec:timeToEventData}
The discussion on how time-to-events data models are to be encoded in NM-TRAN
did not reach consensus yet. Therefore, models encoded in this language comes in two alternative versions.

%%%%%%%%%%%%%%%%%%%%%%%%%%%%%%%%%%%%%%%%%%%%%%%%%%%%%%%%%%%%%%%%
\subsection{Constant hazard function}

\paragraph{Observation model}

\begin{itemize}
\item
Type of observed variable -- discrete/time-to-event
\item
Event variable: $y$
\item
Hazard function: $h(t; \psi_i) = 1/\lambda$
\end{itemize}


\subsubsection{NM-TRAN code provided by Elodie Plan:}
The following code presents two suggested versions.

\myStartLine

\lstset{language=NONMEMdataSet}
\begin{lstlisting}
;;;;;;;;;;;;;; Elodie Plan's version;;;;;;;;;;;;;; 

$SIZES NO=1002 LIM6=1002
;Event 1 model, implementation 1
$PROB TTE model with a right censoring time equal to 100
$INPUT ID TIME DV CENS INTERVAL ; dataset goes until time 100
$DATA tte_sim_data.csv IGNORE=@ 

$SUB ADVAN6 TOL6
$MODEL COMP=HAZ

$PK
LAMBDA=THETA(1)
LAMB=1/LAMBDA
BETA=THETA(2)

$DES
;Set up hazard
IF (T.LE.0) THEN
  DHAZ=0
ELSE
  DHAZ=LAMB*BETA*(LAMB*T)**(BETA-1)
ENDIF
  DADT(1)=DHAZ

$ERROR
;Set up hazard
IF (TIME.LE.0) THEN
  HAZ=0
ELSE
  HAZ=LAMB*BETA*(LAMB*TIME)**(BETA-1)
ENDIF

;;;;;;;;;;;;;; Survival Nick's way ;;;;;;;;;;;;;; 
CUMHAZ=A(1)
SUR=EXP(-CUMHAZ)

;For estimation
;IF(CENS.LT.1 Y=SUR*HAZ
;IF(CENS.EQ.1) Y=SUR

;Settings for simulation
MAXEVT=1 

;Simulation
 IF(ICALL.EQ.4) THEN
    IF(NEWIND.NE.2) NEVT=0 ;0 event per individual at start of each individual
    DV=0 ; 0 event at observation by default
    IF(NEVT.LT.MAXEVT.AND.INTERVAL.EQ.1)THEN ; apply simulation only if max event not reached
      CALL RANDOM (2,R) 
      IF(R.GT.SUR) THEN 
         DV=1
         NEVT=NEVT+1
      ENDIF
    ENDIF
  ENDIF

$THETA
 (0,50)  ;LAMBDA
 (0,2.5)  ;BETA
 
 $OMEGA
 0 FIX

;$ESTIM MAXEVAL=9999 METHOD=0 LIKE PRINT=1 
$SIM (1234) (45678 UNI) ONLYSIM NOPRED NSUB=1

$TABLE ID TIME CENS DV NEVT HAZ SUR CUMHAZ MAXEVT INTERVAL
NOPRINT NOAPPEND ONEHEADER FILE=mytab19
\end{lstlisting}

\myEndLine

\subsubsection{MLXTRAN code from  MLXTRAN tutorial \cite{Monolix4.3Tutorial:2014}:}

\myStartLine

\lstset{language=MLXTRANcode}
\begin{lstlisting}
DESCRIPTION:
Time-to-event data model 
Constant hazard function

INPUT:
parameter = lambda

OBSERVATION:
Y= { 
    type = event 
    hazard = 1/lambda 
}

OUTPUT:
output = Y
\end{lstlisting}

\myEndLine

\subsubsection{PharmML code:}
The parameters $lambda$ is assumed to be defined in the \xelem{ParameterModel} \emph{pm1}.

\lstset{language=XML}
\begin{lstlisting}
        <ObservationModel blkId="om1">
            <Discrete>
                <TimeToEventData>
                    <EventVariable symbId="y"/>
                    
                    <!-- h(t) = 1/lambda -->
                    <HazardFunction symbId="H">
                        <ct:Assign>
                            <math:Equation>
                                <math:Binop op="divide">
                                    <ct:Real>1</ct:Real>
                                    <ct:SymbRef blkIdRef="pm1" symbIdRef="lambda"/>
                                </math:Binop>
                            </math:Equation>
                        </ct:Assign>
                    </HazardFunction>
                </TimeToEventData>
            </Discrete>            
        </ObservationModel>
\end{lstlisting}


%%%%%%%%%%%%%%%%%%%%%%%%%%%%%%%%%%%%%%%%%%%%%%%%%%%%%%%%%%%%%%%%
\subsection{Weibull hazard function}

\paragraph{Observation model}

\begin{itemize}
\item
Type of observed variable -- discrete/time-to-event
\item
Event variable: $y$
\item
Hazard function: $h(t; \psi_i) = \frac{\beta}{\lambda}\frac{t}{\lambda}^{\beta-1}$
\end{itemize}


\subsubsection{NM-TRAN code:}

\myStartLine

\lstset{language=NONMEMdataSet}
\begin{lstlisting}
missing
\end{lstlisting}
\myEndLine

\subsubsection{MLXTRAN code from MLXTRAN tutorial \cite{Monolix4.3Tutorial:2014}:}

\myStartLine
\lstset{language=MLXTRANcode}
\begin{lstlisting}
DESCRIPTION:
Time-to-event data model 
Weibull hazard function

INPUT:
parameter = {lambda, beta}

OBSERVATION:
Y= { type = event 
    hazard = (beta/lambda)*(t/lambda)^(beta-1) 
}

OUTPUT:
output = Y
\end{lstlisting}
\myEndLine

\subsubsection{PharmML code:}
The parameters $beta$ and $lambda$ are assumed to be defined in the \xelem{ParameterModel} \emph{pm1}.

\lstset{language=XML}
\begin{lstlisting}
        <ObservationModel blkId="om1">
            <Discrete>
                <TimeToEventData>
                    <EventVariable symbId="y"/>
                    
                    <!-- h(t) = beta/lambda * t^(beta-1)/lambda -->
                    <HazardFunction symbId="h">
                        <ct:Assign>
                            <math:Equation>
                                <math:Binop op="times">
                                    <math:Binop op="divide">
                                        <ct:SymbRef blkIdRef="pm1" symbIdRef="beta"/>
                                        <ct:SymbRef blkIdRef="pm1" symbIdRef="lambda"/>
                                    </math:Binop>
                                    <math:Binop op="divide">
                                        <math:Binop op="power">
                                            <ct:SymbRef symbIdRef="t"/>
                                            <math:Binop op="minus">
                                                <ct:SymbRef blkIdRef="pm1" symbIdRef="beta"/>
                                                <ct:Real>1</ct:Real>
                                            </math:Binop>
                                        </math:Binop>
                                        <ct:SymbRef blkIdRef="pm1" symbIdRef="lambda"/>
                                    </math:Binop>
                                </math:Binop>
                            </math:Equation>
                        </ct:Assign>   
                    </HazardFunction>
                </TimeToEventData>
            </Discrete>
        </ObservationModel>
\end{lstlisting}


%%%%%%%%%%%%%%%%%%%%%%%%%%%%%%%%%%%%%%%%%%%%%%%%%%%%%%%%%%%%%%%%
\subsection{Concentration dependent hazard function}

\paragraph{Observation model}

\begin{itemize}
\item
Type of observed variable -- discrete/time-to-event
\item
Event variable: $y$
\item
Hazard function: $h(t; \psi_i) = \beta \times Cc$
%\item
%Maximum number of possible events = 1
\end{itemize}


\subsubsection{NM-TRAN code:}

\myStartLine
\lstset{language=NONMEMdataSet}
\begin{lstlisting}
missing
\end{lstlisting}
\myEndLine

\subsubsection{MLXTRAN code based on MLXTRAN tutorial \cite{Monolix4.3Tutorial:2014}:}

\myStartLine
\lstset{language=MLXTRANcode}
\begin{lstlisting}
DESCRIPTION: 
RTTE with concentration dependent hazard function

INPUT:
parameter = {ka, V, Cl, ke0}

EQUATION:
{Cc, Ce} = pkmodel(ka, V, Cl, ke0)

OBSERVATION: 
hemorrhaging = {
    type=event, 
    hazard=beta*Cc
}

OUTPUT: 
output = hemorrhaging
\end{lstlisting}
\myEndLine

\subsubsection{PharmML code:}
The parameter $beta$ is assumed to be defined in the \xelem{ParameterModel} \emph{pm1}.
$Cc$ is assumed to be defined in the \xelem{StructuralModel} \emph{sm1}.

\lstset{language=XML}
\begin{lstlisting}
        <ObservationModel blkId="om1">
            <Discrete>
                <TimeToEventData>
                    <EventVariable symbId="hemorrhaging"/>
                    
                    <!-- h(t) = beta*Cc -->
                    <HazardFunction symbId="h">
                        <ct:Assign>
                            <math:Equation>
                                <math:Binop op="times">
                                    <ct:SymbRef blkIdRef="pm1" symbIdRef="beta"/>
                                    <ct:SymbRef blkIdRef="sm1" symbIdRef="Cc"/>
                                </math:Binop>
                            </math:Equation>
                        </ct:Assign>   
                    </HazardFunction>
                </TimeToEventData>
            </Discrete>
        </ObservationModel>
\end{lstlisting}



%%%%%%%%%%%%%%%%%%%%%%%%%%%%%%%%%%%%%%%%%%%%%%%%%%%%%%%%%%%%%%%%
\subsection{Constant hazard function with censoring}

\paragraph{Observation model}

\begin{itemize}
\item
Type of observed variable -- discrete/time-to-event
\item
Event variable: $y$
\item
Event type -- interval censored
\begin{itemize}
\item
Interval length = 5
\item
Right censoring time = 200
\end{itemize}
\item
Hazard function: $h(t; \psi_i) = 1/\lambda$
\item
Maximum number of possible events = 3
\end{itemize}


\subsubsection{NM-TRAN code provided by Elodie Plan:}
The following code presents two suggested versions.

\myStartLine
\lstset{language=NONMEMdataSet}
\begin{lstlisting}
;;;;;;;;;;;;;; Elodie Plan's version;;;;;;;;;;;;;; 

$SIZES NO=1002 LIM6=1002
;Event 2 model, implementation 2
$PROB RTTE model with a right censoring time equal to 100
$INPUT ID TIME DV CENS INTERVAL ; dataset goes until time 100
$DATA rtte_sim_data.csv IGNORE=@ 

$SUB ADVAN6 TOL6
$MODEL COMP=HAZ

$PK
LAMBDA=THETA(1)*EXP(ETA(1))
LAMB=1/LAMBDA
BETA=THETA(2)

$DES
;Set up hazard
IF (T.LE.0) THEN
  DHAZ=0
ELSE
  DHAZ=LAMB*BETA*(LAMB*T)**(BETA-1)
ENDIF
  DADT(1)=DHAZ

$ERROR
;Set up hazard
IF (TIME.LE.0) THEN
  HAZ=0
ELSE
  HAZ=LAMB*BETA*(LAMB*TIME)**(BETA-1)
ENDIF

;;;;;;;;;;;;;;;Survival Uppsala way;;;;;;;;;;;;;; 
IF(NEWIND.NE.2)  OLDCHZ=0
                 CHZ=A(1)-OLDCHZ
IF(INTERVAL.EQ.1)OLDCHZ=A(1)            
SUR=EXP(-CHZ)

;For estimation
;IF(CENS.LT.99) Y=SUR*HAZ
;IF(CENS.EQ.99) Y=SUR

;Settings for simulation
MAXEVT=10000 ; MORE THAN MAX NUMBER OF SIMULATION STEPS 

;Simulation
 IF(ICALL.EQ.4) THEN
    IF(NEWIND.NE.2) NEVT=0 ;0 event per individual at start of each individual
    DV=0 ; 0 event at observation by default
    IF(NEVT.LT.MAXEVT.AND.INTERVAL.EQ.1)THEN ; apply simulation only if max event not reached
          CALL RANDOM (2,R) 
      IF(R.GT.SUR) THEN 
         DV=1
         NEVT=NEVT+1
      ENDIF
    ENDIF
  ENDIF

$THETA
 (0,50)  ;LAMBDA
 (0,2.5)  ;BETA
 
 $OMEGA
 0.8

;$ESTIM MAXEVAL=9999 METHOD=COND LAPLACE LIKE PRINT=1 
$SIM (1234) (45678 UNI) ONLYSIM NOPRED NSUB=1

$TABLE ID TIME CENS DV NEVT HAZ SUR CHZ MAXEVT INTERVAL
NOPRINT NOAPPEND ONEHEADER FILE=mytab29
\end{lstlisting}
\myEndLine

\subsubsection{MLXTRAN code from  MLXTRAN tutorial \cite{Monolix4.3Tutorial:2014}:}

\myStartLine
\lstset{language=MLXTRANcode}
\begin{lstlisting}
DESCRIPTION:
Time-to-event data model
Constant hazard function with interval censoring 

INPUT:
parameter = lambda

OBSERVATION:
Y = { type = event
    eventType=intervalCensored,
    maxEventNumber=3,
    intervalLength=5,
    rightCensoringTime=200,
    hazard = 1/lambda
}

OUTPUT:
output = Y
\end{lstlisting}
\myEndLine

\subsubsection{PharmML code:}
The parameter $lambda$ is assumed to be defined in the \xelem{ParameterModel} \emph{pm1}.

\lstset{language=XML}
\begin{lstlisting}
        <ObservationModel blkId="om1">
            <Discrete>
                <TimeToEventData>
                    <EventVariable symbId="Y"/>
                    
                    <!-- h(t) = beta*Cc -->
                    <HazardFunction symbId="h">
                        <ct:Assign>
                            <math:Equation>
                                <math:Binop op="times">
                                    <ct:SymbRef blkIdRef="pm1" symbIdRef="beta"/>
                                    <ct:SymbRef blkIdRef="sm1" symbIdRef="Cc"/>
                                </math:Binop>
                            </math:Equation>
                        </ct:Assign>   
                    </HazardFunction>
                    
                    <Censoring censoringType="intervalCensored">
                        <IntervalLength>
                            <ct:Assign>
                                <ct:Real>5</ct:Real> 
                            </ct:Assign>
                        </IntervalLength>
                        
                        <RightCensoringTime>
                            <ct:Assign>
                                <ct:Real>200</ct:Real> 
                            </ct:Assign>
                        </RightCensoringTime>
                    </Censoring>
                    
                    <MaximumNumberEvents>
                        <ct:Assign>
                            <ct:Real>3</ct:Real> 
                        </ct:Assign>
                    </MaximumNumberEvents>
                </TimeToEventData>
            </Discrete>
        </ObservationModel>
 \end{lstlisting}


%%%%%%%%%%%%%%%%%%%%%%%%%%%%%%%%%%%%%%%%%%%%%%%%%%%%%%%%%%%%%%%%
\subsection{PK with Time-to-event model}
\label{subsec:PKPDTTE}
Based on the CTS2 document \cite{Lavielle:2011}.


%%%%%%%%%%%%%%%%%%%%%%%%%%%%%%%%%%%%%%%%%%%%%%%%%%%%%%%%%%%%%%%%
\subsubsection{Introduction}
\label{subsec:exp3_intro}

This particular case is about the so called 'Time-to-event' data models. The cumulative hazard, $Ch(t)$, eq.\eqref{eq:hazardODE1} depends on the time varying hazard function, which is the function of the concentration $Cc(t)$. It follows for the cumulative hazard function
\begin{eqnarray}
\frac{dCh}{dt} = hazard = \beta \times Cc(t) \quad \Longrightarrow \quad Ch(t) = \beta \int_0^{t} Cc(\tilde{t}) \;d\tilde{t} = \beta \times AUC(Cc(0,t))  \label{eq:hazardODE1}
\end{eqnarray}
The survival function reads, eq.\eqref{eq:survivalFct},
\begin{eqnarray}
S(t) = \exp(-Ch(t)) = \exp(- \beta \times AUC(Cc(0,t))) \label{eq:survivalFct}
\end{eqnarray}


%%%%%%%%%%%%%%%%%%%%%%%%%%%%%%%%%%%%%%%%%%%%%%%%%%%%%%%%%%%%%%%%
%\subsubsection{Model description}
%\label{subsec:exp3_TaskDescription}

\paragraph{Individual parameters model}

Details omitted here...

\paragraph{Structural model}
DESCRIPTION oral 1-cpt model ka k V CL and survival model with beta -- hazard parameter
\begin{eqnarray}
k &=& CL/V \nonumber \\
\frac{dAd}{dt} &=&-ka \times Ad \nonumber \\
\frac{dAc}{dt}&=&ka \times Ad - k \times Ac \nonumber \\ 
\frac{dCh}{dt}&=& hazard \nonumber \\ 
Cc &=& Ac/V \nonumber
\end{eqnarray}


%Implementation:
%The whole time span $(0,220)$ can be divided into equidistant intervals and the function $S(t)$ will evaluated and compared with a uniform random number, $p$. If $S(t) < p$, then one event occurs, here a hemorrhaging. Because of \eqref{eq:MaxNumberEvent} only one event is taken into account, i.e. at this event we decrease $n$ by 1.

\paragraph{Observation model}
\begin{itemize}
\item
Type of observed variable -- discrete/time-to-event
\item
Event variable: $y$
\item
Event type -- right censored
\begin{itemize}
\item
Right censoring time = 200
\end{itemize}
\item
Hazard function: $h(t) = \beta \times Cc(t)$
\item
Survival function: $S(t) =exp(-Ch)$
\item
Maximum number of possible events = 1
\end{itemize}

\subsubsection{NM-TRAN code:}

\myStartLine
\lstset{language=NONMEMdataSet}
\begin{lstlisting}
missing
\end{lstlisting}
\myEndLine

\subsubsection{MLXTRAN code:}

\myStartLine
\lstset{language=MLXTRANcode}
\begin{lstlisting}
DESCRIPTION:
Joint PK and time to event data model

INPUT:
   parameter = {ka, V, Cl}

EQUATION:
   {Cc} = pkmodel(ka, V, Cl)
   dCh_dt = hazard	

OBSERVATION:
   hemorrhaging = { type = event, hazard = beta*Cc, survival=exp(-Ch)}

OUTPUT:
   output = {Cc, hemorrhaging}

\end{lstlisting}
\myEndLine

\subsubsection{PharmML code:}
The parameter $beta$ is assumed to be defined in the \xelem{ParameterModel} \emph{pm1}. 
$Cc$ is assumed to be defined in the \xelem{StructuralModel} \emph{sm1}.

\lstset{language=XML}
\begin{lstlisting}
        <ObservationModel blkId="om1">
            <Discrete>
                <TimeToEventData>
                    <EventVariable symbId="Hemorrhaging"/>
                    
                    <!-- h(t) = 1/lambda -->
                    <HazardFunction symbId="h">
                        <ct:Assign>
                            <math:Equation>
                                <math:Binop op="times">
                                    <ct:SymbRef blkIdRef="pm1" symbIdRef="beta"/>
                                    <ct:SymbRef blkIdRef="sm1" symbIdRef="Cc"/>
                                </math:Binop>
                            </math:Equation>
                        </ct:Assign>   
                    </HazardFunction>
                    
                    <!-- S = exp(-Ch) -->
                    <SurvivalFunction symbId="S">
                        <ct:Assign>
                            <math:Equation>
                                <math:Uniop op="exp">
                                    <math:Uniop op="minus">
                                        <ct:SymbRef blkIdRef="sm1" symbIdRef="Ch"/>
                                    </math:Uniop>
                                </math:Uniop>
                            </math:Equation>
                        </ct:Assign>
                    </SurvivalFunction>
                    
                    <Censoring censoringType="rightCensored">
                        <RightCensoringTime>
                            <ct:Assign>
                                <ct:Real>220</ct:Real> 
                            </ct:Assign>
                        </RightCensoringTime>
                    </Censoring>                    
                    
                    <MaximumNumberEvents>
                        <ct:Assign>
                            <ct:Real>1</ct:Real> 
                        </ct:Assign>
                    </MaximumNumberEvents>
                </TimeToEventData>
            </Discrete>
        </ObservationModel>
\end{lstlisting}




%\subsubsection{Censored time to event data}
%\paragraph{NM-TRAN code provided by Mats Karlsson}
%\begin{lstlisting}
%$PROBLEM Time to event data
%
%$INPUT ID TIME DV DOSE ICL IV IKA 
%       TYPE SMAX SMXH THR CAV CAVH CON   
%       CNT=DROP CNT2=DROP CNT3=DROP 
%       HC=DROP HC2=DROP HC3=DROP FE EVID
%       ;ETA1 ETA2 ETA3 ETA4 
%
%$DATA data.csv IGNORE=@  
%;Sim_start : remove from simulation model   
%ACCEPT=(FE.EQ.1) ACCEPT=(TYPE.EQ.1)
%;Sim_end
%
%$SUBR ADVAN=6 TOL=6
%$MODEL COMP=(HAZARD)
%
%$PK
%  ;Sim_start : add for simulation
%  ;RTTE=0
%  ;Sim_end
%
%  CL  = ICL
%  V   = IV
%  KA  = IKA
%
%  BASE = THETA(1)*EXP(ETA(1)) ;the ETA is a placeholder here
%
%$DES
%  CP  =DOSE*KA/V/(-KA+CL/V)*(EXP(-KA*T)-EXP(-CL/V*T))   
%  DADT(1)=BASE    ;hazard   
%
%$ERROR
%  CONC =DOSE*KA/V/(-KA+CL/V)*(EXP(-KA*TIME)-EXP(-CL/V*TIME)) 
%;----------RTTE Model------------------------------
%  IF(NEWIND.NE.2) OLDCHZ=0  ;reset the cumulative hazard
%  CHZ = A(1)-OLDCHZ         ;cumulative hazard 
%                            ;  from previous time point
%                            ;  in data set
%  OLDCHZ = A(1)             ;rename old cumulative hazard
%  SUR = EXP(-CHZ)           ;survival probability
%
%  HAZNOW=BASE               ; rate of event
%                            ;  each time pt
%                            ;  NB: update with each new model
%
%
%  IF(DV.EQ.0)   Y=SUR         ;censored event (prob of survival)
%  IF(DV.NE.0)   Y=SUR*HAZNOW  ;prob density function of event
%
%  ;Sim_start : add for simulation model 
%  ;IF(ICALL.EQ.4) THEN
%  ;  CALL RANDOM (2,R)
%  ;  DV=0
%  ;  RTTE = 0
%  ;  IF(TIME.EQ.480) RTTE = 1 ; for the censored observation at 480 min
%  ;  IF(R.GT.SUR) THEN
%  ;     DV=1
%  ;     RTTE = 1
%  ;  ENDIF
%  ;ENDIF
%  ;Sim_end
%
%$THETA (0,0.025)      ; BASE
%$OMEGA 0 FIX
%
%;Sim_start : add for simulation
%;$SIMULATION (5988566) (39978 UNIFORM) ONLYSIM NOPREDICTION SUB=100
%;Sim_end
%
%$ESTIM MAXEVAL=9990 METHOD=0 LIKE PRINT=1 MSFO=msfb51
%$COV PRINT=E
%
%$TABLE ID TIME SUR DOSE EVID NOPRINT ONEHEADER FILE=sdtab51
%$TABLE ID ICL IV IKA NOPRINT ONEHEADER FILE=patab51
%$TABLE ID CAV CAVH CON NOPRINT ONEHEADER FILE=cotab51
%$TABLE ID TYPE SMAX SMXH THR FE NOPRINT ONEHEADER FILE=catab51
%
%;Sim_start : add for simulation
%;$TABLE ID DV TIME RTTE      ; these are needed
%;       DOSE                   ; optional covariates
%;       NOPRINT NOAPPEND ONEHEADER FILE=fullsimtab51
%;Sim_end
%\end{lstlisting}
%\paragraph{MLXTRAN code ...}
%%\begin{lstlisting}
%%\end{lstlisting}


%\subsubsection{Warfarin Use Case -- warf$\_$p1$\_$simTRT$\_$drop$\_$estTRT$\_$bs.ctl}
%\paragraph{NM-TRAN code provided by Nick Holford and Mike Smith}
%\begin{lstlisting}
%$PROB ESTIMATE TIME TO EVENT
%$DATA ..\warf_sim_p1_drop_TRT.csv
%$INPUT ID TIME TRT
%AMT=DROP RATE=DROP ADDL=DROP II=DROP
%WT DVID DV MDV CP PCA INR
%$WARN NONE ; SUPPRESS WARNING FOR NON-ZERO DV RECORDS WITH MDV=1
%$ESTIM MAX=9990 SIG=3 NOABORT ;PRINT=1
%METHOD=CONDITIONAL LAPLACIAN LIKE
%$COV
% 
%;EVENT HAZARD
%$THETA
%(0,0.1,) ; POP_HBASE 1/Y BASELINE HAZARD (MAJOR HAEMORRHAGE)
%(0,0.4,) ; POP_BTATRT HAZARD RATIO INCREASE PER 1 MG/D WARFARIN
% 
%$OMEGA 0 FIX ; PPV_HBASE NEEDED TO KEEP NM-TRAN HAPPY
% 
%$SUBR ADVAN6 TOL=3
%$MODEL
%COMP=CUMHAZ
%$PK
%IF (NEWIND.LE.1) SRVZ=1 ; SURVIVOR FUNCTION AT TIME=0
% 
%HBASE=THETA(1)*EXP(ETA(1))/365 ; 1/Y -> 1/D
%BTATRT=THETA(2)
% 
%HAZ=HBASE *(1+BTATRT*TRT)
% 
%$DES
%DADT(1)=HAZ
% 
%$ERROR
% 
%CUMHAZ=A(1)
% 
%; PROB OF NOT HAVING DROPPED OUT UP TO THIS TIME
%SRVT=EXP(-CUMHAZ) ; SURVIVOR FUNCTION (T)
%IF (DVID.EQ.3.AND.DV.EQ.0) THEN ; CENSORED EVENT
% 
%      Y=SRVT ; SURVIVOR FUNCTION
%ENDIF
%IF (DVID.EQ.3.AND.DV.EQ.1) THEN
% 
%;      Y=SRVZ - SRVT ; LIKELIHOOD OF EVENT IN INTERVAL
% 
%      Y=SRVT*HAZ ; LIKELIHOOD AT EXACT EVENT TIME
%ENDIF
% 
%IF (DVID.EQ.3) THEN ; START OF NEW EVENT INTERVAL
% 
%      SRVZ=SRVT ; SURVIVOR FUNCTION AT START OF NEW INTERVAL
%ELSE ; SAVE RECURSIVE RANDOM VARIABLE (ONLY WORKS WITH NMVI)
% 
%      SRVZ=SRVZ ; SURVIVOR FUNCTION AT END OF PREVIOUS INTERVAL
%ENDIF
% \end{lstlisting}
%\paragraph{MLXTRAN code ...}
%\begin{lstlisting}
%\end{lstlisting}




%\subsubsection{}
%\paragraph{NM-TRAN code ...}
%\begin{lstlisting}
%\end{lstlisting}
%\paragraph{MLXTRAN code ...}
%\begin{lstlisting}
%\end{lstlisting}



%\begin{sidewaystable}[h]
%\caption{Performance After Post Filtering}  % title name of the table
%\centering  % centering table
%\begin{tabular}{l c c rrrrrrr}  % creating 10 columns
%\hline\hline                       % inserting double-line
% Audio &Audibility & Decision &\multicolumn{7}{c}{Sum of Extracted Bits} 
%\\ [0.5ex]   
%\hline              % inserts single-line
%% Entering 1st row
% & &soft &1 & $-1$ & 1 & 1 & $-1$ & $-1$ & 1  \\[-1ex]
%\raisebox{1.5ex}{Police} & \raisebox{1.5ex}{5}&hard
%&  2 & $-4$ & 4 & 4 & $-2$ & $-4$ & 4 \\[1ex]
%% Entering 2nd row
%& &soft & 1 & $-1$ & 1 & 1 & $-1$ & $-1$ & 1  \\[-1ex]
%\raisebox{1.5ex}{Beethoven} & \raisebox{1.5ex}{5}& hard
%&8 & $-8$ & 2 & 8 & $-8$ & $-8$ & 6 \\[1ex]
%% Entering 3rd row
%& &soft & 1 & $-1$ & 1 & 1 & $-1$ & $-1$ & 1  \\[-1ex]
%\raisebox{1.5ex}{Metallica} & \raisebox{1.5ex}{5}& hard
%&4 & $-8$ & 8 & 4 & $-8$ & $-8$ & 8  \\[1ex]
%% [1ex] adds vertical space
%\hline                          % inserts single-line
%\end{tabular}
%\label{tab:LPer}
%\end{sidewaystable}
%
