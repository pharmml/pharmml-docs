\subsection{PK with Time-to-event model}
\label{subsec:PKPDTTE}
Based on the CTS2 document \cite{Lavielle:2011}.


%%%%%%%%%%%%%%%%%%%%%%%%%%%%%%%%%%%%%%%%%%%%%%%%%%%%%%%%%%%%%%%%
\subsubsection{Introduction}
\label{subsec:exp3_intro}

This particular case is about the so called 'Time-to-event' data models. The cumulative hazard, $Ch(t)$, eq.\eqref{eq:hazardODE1} depends on the time varying hazard function, which is the function of the concentration $Cc(t)$. It follows for the cumulative hazard function
\begin{eqnarray}
\frac{dCh}{dt} = hazard = \beta \times Cc(t) \quad \Longrightarrow \quad Ch(t) = \beta \int_0^{t} Cc(\tilde{t}) \;d\tilde{t} = \beta \times AUC(Cc(0,t))  \label{eq:hazardODE1}
\end{eqnarray}
The survival function reads, eq.\eqref{eq:survivalFct},
\begin{eqnarray}
S(t) = \exp(-Ch(t)) = \exp(- \beta \times AUC(Cc(0,t))) \label{eq:survivalFct}
\end{eqnarray}


%%%%%%%%%%%%%%%%%%%%%%%%%%%%%%%%%%%%%%%%%%%%%%%%%%%%%%%%%%%%%%%%
%\subsubsection{Model description}
%\label{subsec:exp3_TaskDescription}

\paragraph{Individual parameters model}

Details omitted here...

\paragraph{Structural model}
DESCRIPTION oral 1-cpt model ka k V CL and survival model with beta -- hazard parameter
\begin{eqnarray}
k &=& CL/V \nonumber \\
\frac{dAd}{dt} &=&-ka \times Ad \nonumber \\
\frac{dAc}{dt}&=&ka \times Ad - k \times Ac \nonumber \\ 
\frac{dCh}{dt}&=& hazard \nonumber \\ 
Cc &=& Ac/V \nonumber
\end{eqnarray}


%Implementation:
%The whole time span $(0,220)$ can be divided into equidistant intervals and the function $S(t)$ will evaluated and compared with a uniform random number, $p$. If $S(t) < p$, then one event occurs, here a hemorrhaging. Because of \eqref{eq:MaxNumberEvent} only one event is taken into account, i.e. at this event we decrease $n$ by 1.

\paragraph{Observation model}
\begin{itemize}
\item
Type of observed variable -- discrete/time-to-event
\item
Event variable: $y$
\item
Event type -- right censored
\begin{itemize}
\item
Right censoring time = 200
\end{itemize}
\item
Hazard function: $h(t) = \beta \times Cc(t)$
\item
Survival function: $S(t) =exp(-Ch)$
\item
Maximum number of possible events = 1
\end{itemize}

\subsubsection{NM-TRAN code:}

\myStartLine
\lstset{language=NONMEMdataSet}
\begin{lstlisting}
missing
\end{lstlisting}
\myEndLine

\subsubsection{MLXTRAN code:}

\myStartLine
\lstset{language=MLXTRANcode}
\begin{lstlisting}
DESCRIPTION:
Joint PK and time to event data model

INPUT:
   parameter = {ka, V, Cl}

EQUATION:
   {Cc} = pkmodel(ka, V, Cl)
   dCh_dt = hazard	

OBSERVATION:
   hemorrhaging = { type = event, hazard = beta*Cc, survival=exp(-Ch)}

OUTPUT:
   output = {Cc, hemorrhaging}

\end{lstlisting}
\myEndLine

\subsubsection{PharmML code:}
The parameter $beta$ is assumed to be defined in the \xelem{ParameterModel} \emph{pm1}. 
$Cc$ is assumed to be defined in the \xelem{StructuralModel} \emph{sm1}.

\lstset{language=XML}
\begin{lstlisting}
        <ObservationModel blkId="om1">
            <Discrete>
                <TimeToEventData>
                    <EventVariable symbId="Hemorrhaging"/>
                    
                    <!-- h(t) = 1/lambda -->
                    <HazardFunction symbId="h">
                        <ct:Assign>
                            <math:Equation>
                                <math:Binop op="times">
                                    <ct:SymbRef blkIdRef="pm1" symbIdRef="beta"/>
                                    <ct:SymbRef blkIdRef="sm1" symbIdRef="Cc"/>
                                </math:Binop>
                            </math:Equation>
                        </ct:Assign>   
                    </HazardFunction>
                    
                    <!-- S = exp(-Ch) -->
                    <SurvivalFunction symbId="S">
                        <ct:Assign>
                            <math:Equation>
                                <math:Uniop op="exp">
                                    <math:Uniop op="minus">
                                        <ct:SymbRef blkIdRef="sm1" symbIdRef="Ch"/>
                                    </math:Uniop>
                                </math:Uniop>
                            </math:Equation>
                        </ct:Assign>
                    </SurvivalFunction>
                    
                    <Censoring censoringType="rightCensored">
                        <RightCensoringTime>
                            <ct:Assign>
                                <ct:Real>220</ct:Real> 
                            </ct:Assign>
                        </RightCensoringTime>
                    </Censoring>                    
                    
                    <MaximumNumberEvents>
                        <ct:Assign>
                            <ct:Real>1</ct:Real> 
                        </ct:Assign>
                    </MaximumNumberEvents>
                </TimeToEventData>
            </Discrete>
        </ObservationModel>
\end{lstlisting}

