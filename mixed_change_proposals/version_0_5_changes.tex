\documentclass[a4paper,10pt]{report}

\usepackage{graphicx}
\usepackage{hyperref}
\usepackage{amsmath}
\usepackage{amssymb}
\usepackage{xspace}
\pagestyle{headings}
\usepackage[total={6.5in,10in}, top=1.2in, left=.8in, includefoot]{geometry}
\usepackage{float}
\restylefloat{table}
\usepackage{listings}
\usepackage{color}
\definecolor{gray}{rgb}{0.4,0.4,0.4}
\definecolor{darkblue}{rgb}{0.0,0.0,0.58}
\definecolor{attributeColor}{rgb}{0.96,0.517,0.29}
\definecolor{darkgreen}{rgb}{0,.392,0}
\definecolor{stringColor}{rgb}{0.6,0.2,0}
\usepackage{array,multirow}
\usepackage{longtable}
\LTcapwidth=\textwidth
\usepackage{cleveref}
\usepackage{bbding}
\crefname{section}{�}{��}
\Crefname{section}{�}{��}
\usepackage[utf8]{inputenc}
\usepackage{tablefootnote}
\usepackage{algorithmic}
\usepackage[modulo, pagewise]{lineno}
\nolinenumbers
\usepackage{fancyhdr}
\pagestyle{fancy}
\fancyhead[RO]{PharmML 0.5}
\fancyhead[RE]{\itshape\nouppercase{\leftmark}}
%\fancyfoot[CE,CO]{\leftmark}
%\fancyfoot[LE,RO]{\thepage}
%\pagestyle{fancy}
%\fancyhf{}
\usepackage{array,arydshln}
\usepackage{wrapfig}

\setlength\dashlinedash{0.4pt}
\setlength\dashlinegap{1pt}
\setlength\arrayrulewidth{0.5pt}

\newcommand{\myStartLine}{\par
  \kern8pt % space above the rules
  \hrule height 0.5pt
  \kern3pt % space below the rules
}
\newcommand{\myEndLine}{\par
  \kern3pt % space above the rules
  \hrule height 1.5pt
  \kern12pt % space below the rules
}

\lstset{
  basicstyle=\ttfamily,
  columns=fullflexible,
  showstringspaces=false,
  commentstyle=\color{gray}\upshape
%  numbers=left,
%  stepnumber=5
}

\newcommand{\HRule}{\rule{\linewidth}{0.5mm}}

\lstdefinelanguage{XML}
{
  basicstyle=\ttfamily\footnotesize,
  morestring=[b]",
  morestring=[s]{>}{<},
  moredelim=[is][\bfseries\color{red}]{[*}{*]},
  moredelim=[s][\bfseries\color{darkblue}]{<}{\ },
  moredelim=[s][\bfseries\color{darkblue}]{</}{>},
  moredelim=[l][\bfseries\color{darkblue}]{/>},
  moredelim=[l][\bfseries\color{darkblue}]{>},
  morecomment=[s]{<?}{?>},
  morecomment=[s]{<![CDATA[}{]]>},
  moredelim=[s][\bfseries\color{darkgreen}]{<!--}{-->},
  commentstyle=\color{darkgreen},
  stringstyle=\color{stringColor},
  identifierstyle=\color{red},
  keywordstyle=\color{attributeColor},
  morekeywords={oid,columnId,columnIdRef,symbId,symbolType,op,columnNum,columnType,
  valueType,inputTarget,blkId,blkIdRef,symbIdRef,xmlns,version,type,VariableMapping,
  IndividualMapping,schemaLocation,xs,xsi,NONMEMdataSet,matrixType,opType,order,
  math,ct,ds,mdef,mstep,mml,un,name,definition,writtenVersion,id,inputType,oidRef,catId,
  length,default,vectorIndex,diagDefault,offDiagDefault,row,column,numbRows,numbCols,
  dataSymbol,modelSymbol,ordered,compartmentNo,compNo,ordered,linkFunction,varId,
  censoringType,dataSymbol,modelSymbol,MarkovOrder,deviationMatrixType,implementedBy,
  argument,admNumber,transformId,transformIdRef,metadataFile} % list your attributes here
}

\lstdefinelanguage{MLXTRANcode} 
{
  basicstyle=\ttfamily\footnotesize,
  morestring=[b]",
  morestring=[s]{>}{<},
  moredelim=[s][\bfseries\color{darkblue}]{<}{\ },
  moredelim=[s][\bfseries\color{darkblue}]{</}{>},
  moredelim=[l][\bfseries\color{darkblue}]{/>},
  moredelim=[l][\bfseries\color{darkblue}]{>},
  morecomment=[s]{<?}{?>},
  morecomment=[s]{<!--}{-->},
  morecomment=[s]{<![CDATA[}{]]>},
  commentstyle=\color{darkgreen},
  stringstyle=\color{stringColor},
  identifierstyle=\color{black},
  keywordstyle=\color{attributeColor},
  morekeywords={dads} % list your attributes here
}

\lstdefinelanguage{NONMEMdataSet}
{
  basicstyle=\sffamily\small,
  morestring=[b]",
  morestring=[s]{>}{<},
  moredelim=[is][\bfseries\color{red}]{[*}{*]},
  moredelim=[s][\bfseries\color{darkblue}]{<}{\ },
  moredelim=[s][\bfseries\color{darkblue}]{</}{>},
  moredelim=[l][\bfseries\color{darkblue}]{/>},
  moredelim=[l][\bfseries\color{darkblue}]{>},
  morecomment=[s]{<?}{?>},
  morecomment=[s]{<!--}{-->},
  morecomment=[s]{<![CDATA[}{]]>},
  commentstyle=\color{darkgreen},
  stringstyle=\color{stringColor},
  identifierstyle=\color{black},
  keywordstyle=\color{attributeColor},
  morekeywords={kjkj} % list your attributes here
}


\lstdefinelanguage{Elements}
{
  basicstyle=\sffamily\footnotesize,
  morestring=[b]",
  morestring=[s]{>}{<},
  moredelim=[s][\bfseries\color{darkblue}]{<}{\ },
  moredelim=[s][\bfseries\color{darkblue}]{</}{>},
  moredelim=[l][\bfseries\color{darkblue}]{/>},
  moredelim=[l][\bfseries\color{darkblue}]{>},
  morecomment=[s]{<?}{?>},
  morecomment=[s]{<!--}{-->},
  morecomment=[s]{<![CDATA[}{]]>},
  commentstyle=\color{darkgreen},
  stringstyle=\color{stringColor},
  identifierstyle=\color{black},
  keywordstyle=\color{attributeColor},
  morekeywords={kjkj} % list your attributes here
}

\newcommand{\cellml}{CellML\xspace}
\newcommand{\sbml}{SBML\xspace}
\newcommand{\sedml}{SED-ML\xspace}
\newcommand{\mathml}{MathML\xspace}
\newcommand{\uncertml}{UncertML\xspace}
\newcommand{\pml}{PharmML\xspace}
\newcommand{\pharmml}{PharmML\xspace}
\newcommand{\xelem}[1]{\texttt{<#1>}\index{XML Element!\texttt{<#1>}}}
\newcommand{\xatt}[1]{\texttt{#1}\index{XML Attribute!\texttt{#1}}}

\begin{document}

\begin{titlepage}
\begin{center}

% Upper part of the page. The '~' is needed because \\
% only works if a paragraph has started.
\includegraphics[width=0.35\textwidth]{./logo/ddmore_logo}~\\[1cm]

%\textsc{\LARGE }\\[1.5cm]
%
\textsc{\Large Internal Release}\\[0.5cm]

% Title
\HRule \\[0.4cm]
{ \huge \bfseries Extensions in PharmML 0.5 \\[0.4cm] }

\HRule \\[1.5cm]

% Author and supervisor
\begin{minipage}{0.5\textwidth}
\begin{flushleft} \large
\emph{Authors:}\\
Maciek J \textsc{Swat}\\
Sarala \textsc{Wimalaratne}\\
Niels Rode \textsc{Kristensen}
\end{flushleft}
\end{minipage}
\begin{minipage}{0.4\textwidth}
\begin{flushright} \large
\emph{with contributions from:} \\
Roberto \textsc{Bizzotto} \\
Marc \textsc{Lavielle}
\end{flushright}
\end{minipage}



\vfill

% Bottom of the page
{\large \today \\}
% {Version 0.4 was released on July 16, 2014}

\end{center}
\end{titlepage} 

\linenumbers
\tableofcontents


%%%%%%%%%%%%%%%%%%%%%%%%%%%%%%%%%%%%%%%%%%%%%%%%%%%%%%%%%%%%%%%%%%%%%%
%%%%%%%%%%%%%%%%%%%%%%%%%%%%%%%%%%%%%%%%%%%%%%%%%%%%%%%%%%%%%%%%%%%%%%%
%%%%%%%%%%%%%%%%%%%%%%%%%%%%%%%%%%%%%%%%%%%%%%%%%%%%%%%%%%%%%%%%%%%%%%%
%%%%%%%%%%%%%%%%%%%%%%%%%%%%%%%%%%%%%%%%%%%%%%%%%%%%%%%%%%%%%%%%%%%%%%%
\chapter{Overview}

This document describes extensions and changes in \pml compared to the
previously released version 0.4.1. It is not intended to serve as a proper specification. 
The upcoming public release, will come with a fully updated specification document.
The current document introduces the PK macros and few minor extensions.

\section{Extensions in version 0.5}

\begin{center}
\begin{longtable}{lll}
\hline
\hline
\pml element 			&  version $\le$ 0.4.1 			& version 0.5 \\
or modelling aspect 		&							& \\
\hline
PK macros			& \emph{not supported}			& \xelem{PKmacros} element with child elements, such as \\
(see \textsection\ref{chapter:PKmacros})	&				& \xelem{Compartment}, \xelem{Peripheral}, \dots, \xelem{Elimination} \\
					&							& and their attributes {\color{red} \scshape{new}} \\
\hline
MONOLIX datasets		& \emph{not supported}			& \xelem{MONOLIXdataSet} and \xelem{MONOLIXdataSetReference} \\
(see \textsection\ref{subsec:MLXdatasets})	&			& elements {\color{red} \scshape{new}} \\
\hline
Annotation reference  	& \emph{not supported}			& new \xatt{metadataFile} attribute in the \xelem{PharmML} \\
(see \textsection\ref{subsec:metadataFile})	&			& root element {\color{red} \scshape{new}} \\
\caption{Overview of major differences between versions $\leq$ 0.4.1 and 0.5}
\label{figTable:overviewTable}
\end{longtable}
\end{center}



%%%%%%%%%%%%%%%%%%%%%%%%%%%%%%%%%%%%%%%%%%%%%%%%%%%%%%%%%%%%%%%%%%%%%%%
%%%%%%%%%%%%%%%%%%%%%%%%%%%%%%%%%%%%%%%%%%%%%%%%%%%%%%%%%%%%%%%%%%%%%%%
%%%%%%%%%%%%%%%%%%%%%%%%%%%%%%%%%%%%%%%%%%%%%%%%%%%%%%%%%%%%%%%%%%%%%%%
%%%%%%%%%%%%%%%%%%%%%%%%%%%%%%%%%%%%%%%%%%%%%%%%%%%%%%%%%%%%%%%%%%%%%%%
\chapter{PK Macros}
\label{chapter:PKmacros}



%%%%%%%%%%%%%%%%%%%%%%%%%%%%%%%%%%%%%%%%%%%%%%%%%%%%%%%%%%%%%%%%%%%%%%
\section{Introduction}

A long standing problem is how to define effectively PK models, without using ODEs, in a tool-independent 
way. There are many tool-specific solutions, each of them using their own ways to define PK models. 
After an thorough analysis of a number of available approaches, the MLXTRAN PK macro system has
been selected as the one which should be supported by PharmML. It has the following features 
\begin{itemize}
\item
Tool-independent, but easily translatable to tool-specific libraries
\item
User friendly and easy to learn for users new to the field
\item
Flexible and providing consistent specification of compartmental PK models
\item
More models can be specified without ODEs, compared to nmadvan/PREDPP
\item
No limitations will be imposed, compared to nmadvan/PREDPP:
\begin{itemize}
\item
All PREDPP library models with underlying analytical solutions (ADVAN1-4, 11-12) can be specified. 
Templates are provided for these to help NONMEM users, see Section \ref{subsec:PREDPPinMACROS}.
\item
The PREDPP library models applying matrix exponential-based solutions (ADVAN 5, 7) are already 
being defined in a way very similar to the macro-based solution proposed.
\item
The PREDPP library models applying numerical ODE solution (ADVAN 6, 8-9, 10, 13) can still be 
specified by using ODEs explicitly (or by using/mixing with macro-definitions).
\end{itemize}
\end{itemize}

Adopting the PK macros should allow seamless translation of models between MDL and 
PharmML and MLXTRAN. Therefore, in the PharmML implantation of the macros we tried 
to stay as close as possible to the original MLXTRAN specification. We had to make sure 
that the macros, their attributes and possible assignments are defined in a unambiguous 
way. In the sections \ref{subsec:encodingRules} and \ref{subsec:fixedArguments} we 
explain the rules to be followed in their implementation and the translation process to and 
from PharmML.


%%%%%%%%%%%%%%%%%%%%%%%%%%%%%%%%%%%%%%%%%%%%%%%%%%%%%%%%%%%%%%%%%%%%%%
\subsection{The macro idea}
Figure \ref{fig:flowDiagram} visualises the principle of the PK macros solution and it's use. The module 
within the inner MDL/PharmML box shows few example PK macros and their attributes using which a 
PK model can be formulated. 
\begin{figure}[htb!]
\centering
  \includegraphics[width=160mm]{pics/PKmacrosPrinciple}
 \caption{A diagram explaining how the PK macro system can be used in practise. The inner 
 MDL/PharmML box contains the information to be stored in PharmML, i.e. the macros and arguments 
 from the PK model vocabulary. It can then be uniquely translated into a set of ODEs and for some 
 models there exists an analytic solution. Models encoded using the controlled vocabulary can be 
 mapped to tool specific implementations (dotted lines).}
 \label{fig:flowDiagram}
\end{figure}
In this case four macros, with one or more arguments, are used to encode a basic \emph{oral 1-compartment 
model with linear elimination} consisting of rolling items
\begin{enumerate}
\item
depot compartment, with amount \emph{Ad},
\item
central compartment, with amount \emph{Ac} and volume \emph{V},
\item
oral administration, 1$^{st}$ order absorption, \emph{ka}, and 
\item
linear elimination, with constant rate \emph{k}.
\end{enumerate}
%For other examples see Tab.\ref{tab:MappingTable}. 
This is the \emph{only} information needed to be encoded in PharmML for this particular PK model. It 
corresponds to the following macro
\lstset{language=NONMEMdataSet}
\begin{lstlisting}
		compartment(cmt=1,concentration=Cc,volume=V) 
		oral(cmt=1, ka) 
		elimination(cmt=1, k)
\end{lstlisting}
Any information about input, i.e. dosing times and amounts will be stored in the dataset 
or explicitly within the \xelem{TrialDesign}.

The key \marginpar{\HandCuffLeft} realisation is that every set of such macro statements, 
if correctly defined, can be translated automatically into a unique set of ODE's and/or 
algebraic equations, in this case:
\begin{align}
\frac{dAd}{dt} &= -ka \times Ad \nonumber \\
\frac{dAc}{dt} &= ka \times Ad - k \times Ac  \nonumber \\
C &= Ac/V \nonumber
\end{align}

Moreover, for some models such as the above one, there exists an analytic solution. This however cannot 
be derived automatically for an arbitrary ODE system. It requires a powerful symbolic calculation 
software or expert knowledge. Once the PK model macros and their attribute set have been defined, 
a mapping to tool-specific libraries can be provided, see Fig.\ref{fig:flowDiagram}. 
This again requires expert knowledge of each particular tool. 



%%%%%%%%%%%%%%%%%%%%%%%%%%%%%%%%%%%%%%%%%%%%%%%%%%%%%%%%%%%%%%%%%%%%%%%
\section{PK macros in PharmML}
\label{subsec:PKmacros}


%An essential aspect of the macros is their connectivity to the 
%remaining model elements and the data. The necessary extensions in the current structure are
%discussed, for the input in section \ref{subsec:LinkingMacrosDatasets} and for the output in 
%section \ref{subsec:macroOutputLink}. 

%{\color{red} \scshape{NEW}}
%%%%%%%%%%%%%%%%%%%%%%%%%%%%%%%%%%%%%%%%%%%%%%%%%%%%%%%%%
\subsection{Encoding rules}
\label{subsec:encodingRules}
To provide the full support for MLXTRAN PK macros, all macros and their (unordered) named 
arguments have to be encodable, see Table \ref{tab:MLXPLORElibrary}. Every macro has its own 
PharmML element, e.g. the basic \emph{concentration} macro with all its arguments, encoded as \xelem{Value} 
elements, will be encoded within the \xelem{Concentration} element. The optional attribute \xatt{argument}
is assigned a value corresponding to the argument of the macro. If no assignment is necessary a reference is defined.

\subparagraph{The basic principle for the implementation and translation to and from PharmML} 
Macros arguments are encoded as attributes of macro elements only if an assignment is required, 
otherwise we just refer to them with the standard \xelem{SymbRef} element. 

For example, the \emph{elimination} macro defining the linear elimination with rate constant \xatt{k}
from the first compartment reads
\lstset{language=NONMEMdataSet}
\begin{lstlisting}
		elimination(cmt=1, k)
\end{lstlisting}
and is encoded in PharmML as
\lstset{language=XML}
\begin{lstlisting}
            <!-- omitted other model elements-->

                <Elimination>
                    <Value argument="cmt">
                        <ct:Int>1</ct:Int>
                    </Value>
                    <Value>
                        <ct:SymbRef blkIdRef="pm1" symbIdRef="k"/>
                    </Value>
                </Elimination>
            </PKmacros>
        </StructuralModel>
\end{lstlisting}
This code makes the above principle and these two implementation options clear. In the first case, 
an assignment is necessary. The attribute \xatt{argument="cmt"} is used to specify the compartment argument 
of the macro \xelem{Elimination} and in the next line the number '1' is assigned to it.

In the second case, the elimination rate parameter \xatt{k} is simply referred to without an assignment. 
The mandatory here \xatt{blkIdRef} attribute points to the parameter model, \xatt{pm1}, where \xatt{k} 
is expected to be defined.


%%%%%%%%%%%%%%%%%%%%%%%%%%%%%%%%%%%%%%%%%%%%%%%%%%%%%%%%%
\subsection{Fixed argument values and exceptions}
\label{subsec:fixedArguments}
All but one macro have a fixed set of arguments, e.g. the \xatt{compartment} can take only these predefined
four arguments: \xatt{\{cmt, amount, volume, concentration\}}. The exception is the \xatt{peripheral} argument with three fixed
arguments \xatt{\{amount, volume, concentration\}} and a variable one, \xatt{k\_ij or k\_i\_j}, where \xatt{i} and
\xatt{j} stand for input and output compartment, respectively. These compartments numbers are set 
according to the model configuration. 

For example, in the case of models ADVAN11 or 12, see sections \ref{subsubsec:ADVAN11} and 
\ref{subsubsec:ADVAN12} for complete description, we need to set the \xatt{kij} arguments names accordingly to the model
specification, meaning that macro 
\lstset{language=NONMEMdataSet}
\begin{lstlisting}
		peripheral(k12, k21, amount=Ap1)
\end{lstlisting}
will be implemented as shown in the following snippet
\lstset{language=XML}
\begin{lstlisting}
                <Peripheral>
                    <Value>
                        <ct:SymbRef blkIdRef="pm1" symbIdRef="k12"/>
                    </Value>
                    <Value>
                        <ct:SymbRef blkIdRef="pm1" symbIdRef="k21"/>
                    </Value>
                    <!-- omitted amount argument -->
                </Peripheral>
\end{lstlisting}

If, on the other hand, the macro contains assignments, e.g.
\lstset{language=NONMEMdataSet}
\begin{lstlisting}
                peripheral(k12=0.6, k21=0.8, amount=Ap)
\end{lstlisting}
the implementation will be as shown in the next snippet 
\lstset{language=XML}
\begin{lstlisting}
                <Peripheral>
                    <Value argument="k12">
                        <ct:Real>0.6</ct:Real>
                    </Value>
                    <Value argument="k21">
                        <ct:Real>0.8</ct:Real>
                    </Value>
                    <!-- omitted amount argument -->
                </Peripheral>
\end{lstlisting}
Also the alternative notation, \xatt{k\_i\_j}, is possible, allowing \xatt{i} and \xatt{j} to take values between
0 and 99. Compartmental models of this complexity are quite unlikely, but it is possible to connect e.g.
the eighth and the fifteenth compartment as the following code shows
\lstset{language=XML}
\begin{lstlisting}
                <Peripheral>
                    <Value argument="k_08_15">
                        <ct:Real>0.0815</ct:Real>
                    </Value>
                    <!-- omitted amount argument -->
                </Peripheral>
\end{lstlisting}


%\newpage
%%%%%%%%%%%%%%%%%%%%%%%%%%%%%%%%%%%%%%%%%%%%%%%%%%%%%%%%%
\subsection{Macros and target tools}
\label{subsec:LinkingMacrosDatasets}

One of the main reason to introduce the system of macros is that they offer a flexible 
and efficient way for encoding PK models. The proposed macros are fully compatible with 
Monolix and its data format, see Appendix B in \cite{Monolix4.3.2UserGuide:2014}. 

Of course, from the perspective of the interoperability platform \marginpar{\HandCuffLeft}
it is crucial that the models stored in PharmML using macros can be unambiguously translated 
to NONMEM and other target tools. Especially the compatibility with few predefined PREDPP 
routines (i.e. ADVAN 1-4 and 10-12) is essential because of their popularity among the 
NONMEM modellers community. 

To understand the required translation rules, the knowledge of relevant NONMEM and 
MONOLIX features and the dataset formats these target tools use is helpful. 
For convenience, few most essential differences between these tools/formats from the 
perspective of the macros and their usage are listed
\begin{itemize}
\item
In PREDPP library (ADVAN 1-4, 10-12 routines) the \emph{DEPOT} compartment is 
always considered as the $1^{st}$ compartment, the central compartment as $2^{nd}$ 
and the peripheral compartments as $3^{rd}$, $4^{th}$ etc. 
\item
In contrast, MONOLIX doesn't assign a number to the \emph{DEPOT} 
compartment at all and the numbering starts usually with the central compartment. 
\item
The CMT column, standard in NONMEM, is not used in Monolix. It uses an ADM column instead.
\end{itemize}





%To provide such flexibility we have to 
%ensure that they can handle, for start, the three basic data formats 
%\begin{itemize}
%\item
%NONMEM dataset, as described in \cite{NONMEM:2009} (with restrictions such as 
%no support for PREDPP keywords and special encoding tricks)
%\item
%MONOLIX' version, a modified NONMEM dataset, see a detailed specification in \cite{Monolix4.3.2UserGuide:2014}.
%\item
%\xelem{TrialDesign} model provides the possibility to store data records in a tool-independent manner. 
%\xelem{IndividualDosing} element is the place where relevant dosing data can be encoded inline
%or be referred to from an external datafile. The actual mapping is defined within the \xelem{Activity}
%tag.  
%\end{itemize}
%There are many incompatible differences between these dataset formats, \marginpar{\HandCuffLeft} 
%here just a couple relevant points from the point of view of the PK macros


%Due to the lack of an common data format and in order to make sure that the macros can work 
%both with Monolix, NONMEM and \xelem{TrialDesing} coded datasets we needed a modification 
%in the original MLXTRAN macros.
%
%\begin{figure}[ht!]
%\centering
% \includegraphics[width=0.6\textwidth]{pics/MacrosDatasets}
%\caption{Macros are suppose to work with Monolix, NONMEM and \xelem{TrialDesing} coded 
%datasets. }
%\label{fig:MacrosDatasets}
%\end{figure}

%\subsubsection{Modifications}
%Figure \ref{fig:ComplexModel3A} and Table \ref{tab:datasetComparison} will be used 
%to illustrate the problem. Two administrations are defined, one oral, one intravenous.
%While MONOLIX (left) doesn't assign a number to the depot compartment, 
%NONMEM (right) does. The differences are inter-connected with the format of the dataset 
%compatible with each model version. Monolix uses administration type number stored 
%in the ADM column to identify the administration process. NONMEM requires a CMT 
%column in the dataset to link a drug administration with its target. 
%
%To describe an oral administration in the original MONOLIX macro system two lines 
%of code are required and is sufficient to inform the tool about the dosing process
%\lstset{language=NONMEMdataSet}
%\begin{lstlisting}
%		compartment(cmt=1, volume=V, concentration=Cc)
%		oral(adm=1, cmt=1, ka)
%\end{lstlisting}
%
%
%However, to make the macro work with the NONMEM dataset, 
%Figure \ref{tab:datasetComparison} (RIGHT), the following modifications (in red) 
%are required
%\lstset{language=NONMEMdataSet}
%\begin{lstlisting}
%		compartment(cmt=1, volume=V, concentration=Cc)
%		[*compartment(cmt=2, amount=Depot)*]
%		oral(adm=1, [*fromCmt=2,*] cmt=1,k)
%\end{lstlisting} 
%A new \xatt{compartment} macro for the depot needs to be defined and to make 
%the connection with the target compartment work a new attribute \xatt{fromCmt}
%has to be introduced.

\subsubsection{An illustrative example}
Figure \ref{fig:ComplexModel_Rules} and datasets in Table \ref{tab:C3ModelData} 
give us some clues about two alternative interpretations for a model with at 
least one oral administration, dependent on the target tool. Note, that apart from 
few predefined models, such as ADVAN1-4 and 10-12, one can assign an arbitrary 
number to the  (first) depot compartment in the NONMEM dataset. This is exactly what 
happens in this example. While in MONOLIX we would have only two compartments 
with assigned number to them, all four compartments have numbers assigned in NONMEM,
Figure \ref{fig:ComplexModel_Rules} (right). This mean that while translating a model 
from PharmML a transformation of the associated dataset is required. 
Note, the contrary to models corresponding to one of the ADVAN 1-4 and 10-12 routines, 
the depot compartment can be assigned an arbitrary number in NONMEM.

It is possible to translate a macro and dataset, utilising the ADM column to identify 
administrations, used by Monolix into an PREDPP/ADVAN based NONMEM encoding 
system with data set using the CMT column.


\begin{figure}[h!]
\centering
 \includegraphics[width=140mm]{pics/ComplexModel_Rules}
\caption{Two interpretations for a model with two oral administrations.
While MONOLIX (left) doesn't assign a number to the depot compartments, NONMEM (right) does. 
The translation of macro coded models to target tools requires in this case changes 
in the dataset as well, see Table \ref{tab:C3ModelData}.}
\label{fig:ComplexModel_Rules}
\end{figure}



\begin{table}[h!]
\footnotesize
\parbox{.5\linewidth}{
\centering
\begin{tabular}{ccccc}
  \hline
   \multicolumn{5}{c}{\textbf{MONOLIX}} \\
  \hline
ID	& TIME  & AMT	 & \textbf{ADM}  &	Y \\
  \hline
1	& 6	    & 10	& \textbf{1}	 & . \\
1	& 9	    & 20	& \textbf{2}	 & . \\
1	& 18	    & 10	& \textbf{1}	 & . \\
1	& 33	    & 20	& \textbf{2}	 & . \\
...     &  ...     &  ...     &  ...  & ... \\
1	& 0	    & .	& .	& 0 \\
1	& 12	    & .	& .	& 1.18 \\
...     &  ...     &  ...     & ...  & ...\\
\end{tabular}
}
\hfill
\parbox{.5\linewidth}{
\centering
\begin{tabular}{ccccc}
  \hline
   \multicolumn{5}{c}{\textbf{NONMEM}} \\
  \hline
ID	& TIME  & AMT	 & \textbf{\textcolor{red}{CMT}} &	DV \\
  \hline
1	& 6	    & 10	& \textbf{\textcolor{red}{3}}	 & . \\
1	& 9	    & 20	& \textbf{\textcolor{red}{4}}	 & . \\
1	& 18	    & 10	& \textbf{\textcolor{red}{3}}	 & . \\
1	& 33	    & 20	& \textbf{\textcolor{red}{4}}	 & . \\
...     &  ...     &  ...     &  ...  & ... \\
1	& 0	    & .	& 1	& 0 \\
1	& 12	    & .	& 1	& 1.18 \\
...     &  ...     &  ...     & ...  & ...\\
\end{tabular}
}
\caption{MONOLIX and NONMEM datasets for the model above, Figure 
\ref{fig:ComplexModel_Rules}. The translator has to make sure that the numbers 
in the ADM column of the former are properly converted to the CMT column in the latter dataset.}
\label{tab:C3ModelData}
\end{table}


\subsubsection{Basic translation rules}
The following few basic translation rules should be treated merely as a guidance.
Also, they are not covering the whole aspect of the model/data handling within the 
interoperability platform but are limited to the models encoded with macros. 
A more careful analysis is required to cover the structural model exchange between 
the key target tools and associated datasets.

The flow of the data within the framework \marginpar{\HandCuffLeft}  has been so far 
successfully avoided/neglected and needs to return to the agenda again. 

\subparagraph{Models corresponding to ADVAN 1-4 and 10-12}
Fortunately, the translation of models encoded with macros and having their equivalents 
in routines ADVAN1-4 and 10-12 in PREDPP and their datasets is straightforward. 
Simply, one has to translate a corresponding macro into its equivalent ADVAN 
routine and then to rename the ADM column in the MONOLIX data set into CMT, 
see for more details and examples in Section \ref{subsec:PREDPPinMACROS}.

\subparagraph{All other models}
For every oral administration one new depot compartment needs tho be introduced
and assigned a number. The Figure \ref{fig:ComplexModel_Rules} and datasets 
in Table \ref{tab:C3ModelData} can again help to understand what is required. 

In this example, \xatt{adm=1} and \xatt{adm=2} denote two oral administrations. 
In the NONMEM system, Figure \ref{fig:ComplexModel_Rules} (right), '1' and '2' are  
assigned to the central and peripheral compartment respectively, as in the MONOLIX, 
so that the next free numbers, '3' and '4', will be assigned to the depot compartments. 

Next, the NONMEM dataset needs to be produced based on the information stored 
in the MONOLIX dataset in such a way that it will match the new compartment 
numbering scheme with the CMT column replacing the ADM column. 

The oral administration, \xatt{adm=1}, aims now at the compartment 3, so that '1' 
in the ADM column will correspond to '3' in the CMT column. Similarly, \xatt{adm=2}, 
aims now at the compartment 4, so that '2' in the ADM column will be translated to '4' 
in the CMT column.

As additional rules, applicable for all models, one should keep in mind that 
(1) Y column in a MONOLIX dataset has to be renamed into DV of the NONMEM dataset 
and (2) the TINF column has to be renamed into RATE 
and its content recalculated accordingly. 

%\subparagraph{Column names} TINF and Y columns in the original datasets have 
%to be renamed to RATE and DV, respectively. 

%\begin{figure}[h!]
%\centering
% \includegraphics[width=120mm]{pics/ComplexModel3}
%\caption{Two model interpretations for a model with more then one administration 
%from which at least one is of oral type, dependent on the target tool. 
%While MONOLIX (left) doesn't assign a number to the depot compartment, 
%NONMEM (right) does. The difference is inter-connected with the format of the 
%dataset compatible with each model version, see Table \ref{tab:datasetComparison}.}
%\label{fig:ComplexModel3A}
%\end{figure}
%
%
%\begin{table}[ht!]
%\footnotesize
%\parbox{.5\linewidth}{
%\centering
%\begin{tabular}{ccccc}
%  \hline
%     \multicolumn{5}{c}{\textbf{MONOLIX}} \\
%  \hline
%ID	& TIME  & AMT	 & \textbf{ADM} &	Y \\
%  \hline
%1	& 0	    & 2.24	& \textbf{2}	& . \\
%1	& 1	    & .	& .		 	& 142 \\
%1	& 2	    & .	& .	 		& 54.9 \\
%1	& 3	    & .	& .			& 25.9 \\
%1	& 4	    & .	& .			& 17.5 \\
%1	& 6	    & 7	& \textbf{1}	& . \\
%1	& 7	    & .	& .			& 192 \\
%1	& 8	    & .	& .			& 141 \\
%1	& 9	    & .	& .			& 189 \\
%2	& 0	    & 2.73	& \textbf{2}	& . \\
%2	& 1	    & . 	& .			& 176 \\
%...     &  ...     &  ...     & ...  			& ...\\
%\end{tabular}
%}
%\hfill
%\parbox{.5\linewidth}{
%\centering
%\begin{tabular}{ccccc}
%  \hline
%   \multicolumn{5}{c}{\textbf{NONMEM}} \\
%  \hline
%ID	& TIME  & AMT	 & \textbf{\textcolor{red}{CMT}} &	Y \\
%  \hline
%1	& 0	    & 2.24	& \textbf{\textcolor{red}{1}}	& . \\
%1	& 1	    & .	& .		 	& 142 \\
%1	& 2	    & .	& .	 		& 54.9 \\
%1	& 3	    & .	& .			& 25.9 \\
%1	& 4	    & .	& .			& 17.5 \\
%1	& 6	    & 7	& \textbf{\textcolor{red}{2}}	& . \\
%1	& 7	    & .	& .			& 192 \\
%1	& 8	    & .	& .			& 141 \\
%1	& 9	    & .	& .			& 189 \\
%2	& 0	    & 2.73	& \textbf{\textcolor{red}{1}}	& . \\
%2	& 1	    & . 	& .	 		& 176 \\
%...     &  ...     &  ...     & ...  			& ...\\
%\end{tabular}
%}
%\caption{Comparison of NONMEM and MONOLIX datasets for the model shown 
%in Figure \ref{fig:ComplexModel3A}. NONMEM requires additionally a CMT column in 
%the dataset, whereas MONOLIX doesn't and uses instead administration type number
%stored in the ADM column.}
%\label{tab:datasetComparison}
%\end{table}


%\newpage
\subsubsection{Datasets mapping}
The elements defined in previous version 0.4.1, \cite{Swat:2014bb}, are reused here to encode the 
mapping between macros and the MONOLIX dataset, i.e. the mapping between 
\xatt{adm/type} attributes in the PK macros and the administration type as stored in the 
ADM column,
\begin{itemize}
\item 
\xelem{TargetMapping} element with the \xatt{blkIdRef} attribute, the latter one because 
in PharmML multiple structural models are allowed, so we have to specify which model is 
holding the target compartment definition
\item
\xelem{Map} element with
\begin{itemize}
\item 
\xatt{dataSymbol} -- attribute denoting the target symbol as encoded in a dataset and
\item 
new attribute \xatt{admNumber} -- to identify the target symbol in the model.
\end{itemize}
\end{itemize}

%\xatt{modelSymbol}/\xatt{cmtNumber}/

The use of these new elements is explained in the Table \ref{tab:mappingNONMEMAndMacros}. 
%(Left) the mapping if MONOLIX dataset is used, {right} for the NONMEM dat set.
\begin{table}[ht!]
\setlength{\tabcolsep}{5pt}
\begin{center}
\begin{tabular}{l}
  \hline
  \hline
MONOLIX datasets mapping \\ %  	& NONMEM datasets mapping \\
  \hline
\lstset{language=XML}
\begin{lstlisting}
<NONMEMdataSet oid="MLXoid">
    <!-- omitted details -->
    
    <ColumnMapping>
        <ds:ColumnRef columnIdRef="ADM"/>
        <ds:TargetMapping blkIdRef="sm3">
            <ds:Map dataSymbol="1" admNumber="1"/>
            <ds:Map dataSymbol="2" admNumber="2"/>
            <ds:Map dataSymbol="3" admNumber="3"/>
        </ds:TargetMapping>
    </ColumnMapping>
    
    <!-- omitted Definition of dataset  
    	columns such as ID,TIME,AMT,ADM  -->
\end{lstlisting}
%&
%\lstset{language=XML}
%\begin{lstlisting}
%<NONMEMdataSet oid="NMoid">
%    <!-- omitted details -->
%    
%    <ColumnMapping>
%        <ds:ColumnRef columnIdRef="CMT"/>
%        <ds:TargetMapping blkIdRef="sm3">
%            <ds:Map dataSymbol="2" cmtNumber="2"/>
%            <ds:Map dataSymbol="3" cmtNumber="3"/>
%            <ds:Map dataSymbol="4" cmtNumber="4"/>
%        </ds:TargetMapping>
%    </ColumnMapping>
%    
%    <!-- omitted Definition of dataset  
%    	columns such as ID,TIME,AMT,CMT  -->
%\end{lstlisting}
%\\
%& OR \\
%&
%\lstset{language=XML}
%\begin{lstlisting}
%<NONMEMdataSet oid="NMoid">
%    <!-- omitted details -->
%    
%    <ColumnMapping>
%        <ds:ColumnRef columnIdRef="CMT"/>
%        <ds:TargetMapping blkIdRef="sm3">
%            <ds:Map dataSymbol="2" modelSymbol="Ac"/>
%            <ds:Map dataSymbol="3" modelSymbol="Depot1"/>
%            <ds:Map dataSymbol="4" modelSymbol="Depot2"/>
%        </ds:TargetMapping>
%    </ColumnMapping>
%    
%    <!-- omitted Definition of dataset  
%    	columns such as ID,TIME,AMT,CMT  -->
%\end{lstlisting}
\\
    \hline
\end{tabular}
\caption{Mappings of MONOLIX type datasets and PK macros. For now the 
\xelem{NONMEMdataSet} element is used to encode MONOLIX dataset but this will be replaced
by an element with a more suitable name.}
\label{tab:mappingNONMEMAndMacros}
\end{center}
\end{table}



%\newpage
%%%%%%%%%%%%%%%%%%%%%%%%%%%%%%%%%%%%%%%%%%%%%%%%%%%%%%%%%
\subsection{Linking macros and \xelem{TrialDesign}}
\label{subsec:LinkingMacrosTrialDesign}
\xelem{TrialDesign} model provides an alternative to store data records (observations, 
dosing and covariates) in a tool-independent manner. The \xelem{IndividualDosing} 
element is the place where relevant dosing data can be encoded inline or be referred 
to from an external datafile. The actual mapping is defined within the \xelem{Activity} 
tag.

A minor extension in the \xelem{TrialDesign} is necessary to link the administrations coded
in \xelem{Activity} elements  to the targets in the PK macros and identified using the 
\xatt{adm} attribute, i.e.
\begin{itemize}
\item
new value \textit{administrationType} is added to the \xatt{inputTarget} attribute.
\end{itemize}
Otherwise, elements defined in the last section will be reused. The Table 
\ref{tab:mappingTrialDesignAndMacros} shows the implementations within the 
\xelem{TrialDesign} element in cases when (left) a PK model is expressed using 
algebraic equation with dose parameter, D, and (right) when using PK macros.
\begin{table}[ht!]
\setlength{\tabcolsep}{5pt}
\begin{center}
\begin{tabular}{ll}
  \hline
  \hline
Using dose \xatt{parameter}, D, and algebraic & Using value \emph{administrationType} for \xatt{inputTarget}  \\
equation for PK model, e.g. $C(t)=f(D,V,k,...)$ 	& and \xelem{TargetMapping}/\xelem{Map} elements   \\
(available in PharmML since version 0.2.1)	& 	\\
  \hline
\lstset{language=XML}
\begin{lstlisting}
<Activity oid="actBolusD">
    <Bolus>
        <DoseAmount inputTarget="parameter">
            <ct:SymbRef blkIdRef="sm3" symbIdRef="D"/>
            <ct:Assign>
                <ct:Real>10</ct:Real>
            </ct:Assign>
        </DoseAmount>
        <DosingTimes>
            <!-- e.g. 10 -->
        </DosingTimes>
    </Bolus>
</Activity>
\end{lstlisting}
&
\lstset{language=XML}
\begin{lstlisting}
<Activity oid="actBolusMacro">
    <Bolus>
        <DoseAmount inputTarget="admType">
            <ds:TargetMapping blkIdRef="sm3">
                <ds:Map admNumber="2"/>
            </ds:TargetMapping>
            <ct:Assign>
                <ct:Real>10</ct:Real>
            </ct:Assign>
        </DoseAmount>
        <!-- omitted DosingTimes -->
    </Bolus>
</Activity>
\end{lstlisting}
\\
    \hline
\end{tabular}
\caption{Comparison of the link between administration definitions and structural model, 
dependent on the formulation of the latter. (left) PK model expressed using algebraic equations
 with dose parameter, $D$, and when using the PK macros (right).}
\label{tab:mappingTrialDesignAndMacros}
\end{center}
\end{table}


%{\color{red} \scshape{NEW}}
%\newpage
%%%%%%%%%%%%%%%%%%%%%%%%%%%%%%%%%%%%%%%%%%%%%%%%%%%%%%%%%
\subsection{Connection between macros and the model}
\label{subsec:macroOutputLink}
The possible outputs of PK macros are amounts, e.g. $Ac$, and concentrations, e.g. $C$. 
A PK model defined by a system of macros will often be connected to a subsequent PD 
model, which expects the concentration, $C$, as one of the inputs. Another option is that the output
of the PK macros will be mapped directly to the data in the \xelem{ObservationModel}. 

For the \emph{compartment} macro in question, there are two possibilities, either
\begin{itemize}
\item
the concentration is defined in the macro
\lstset{language=NONMEMdataSet}
\begin{lstlisting}
		compartment(cmt=1,concentration=C, volume=V)
\end{lstlisting}
in which case no output has to be defined explicitly, or
\item
the alternative form of this macro is used
\lstset{language=NONMEMdataSet}
\begin{lstlisting}
		compartment(cmt=1,amount=Ac, volume=V)
\end{lstlisting}
and because eventually the concentration is required, a subsequent assignment for $C$ must be provided in MLXTRAN
in the \emph{EQUATION} block, i.e.
\lstset{language=NONMEMdataSet}
\begin{lstlisting}
		C = Ac/V
\end{lstlisting}
\end{itemize}
In PharmML, these two cases have to be treated accordingly. 
\begin{itemize}
\item
In the first case only the concentration variable, $C$, needs to be defined
\lstset{language=XML}
\begin{lstlisting}
        <StructuralModel blkId="sm1">
            <ct:Variable symbolType="real" symbId="C"/>
            
            <PKmacros>
                    <!-- omitted details of the macro 
                    compartment(cmt=1,concentration=C, volume=V) -->
\end{lstlisting}
\item
otherwise the amount variable, $Ac$, and concentration variable, $C$, with the $C=Ac/V$ assignment 
needs to be defined
\lstset{language=XML}
\begin{lstlisting}
        <StructuralModel blkId="sm1">
            <ct:Variable symbolType="real" symbId="Ac"/>
            <ct:Variable symbolType="real" symbId="Cc">
                <ct:Assign>
                    <math:Equation>
                        <math:Binop op="divide">
                            <ct:SymbRef symbIdRef="Ac"/>
                            <ct:SymbRef blkIdRef="pm1" symbIdRef="V"/>
                        </math:Binop>
                    </math:Equation>
                </ct:Assign>
            </ct:Variable>
            
            <PKmacros>
                <Compartment>
                    <!-- omitted details of the macro 
                    compartment(cmt=1,amount=Ac, volume=V) -->
\end{lstlisting}
\end{itemize}
If a model using the PK macros is defined properly, following the principles described above, 
both options make sure that the connection to a subsequent model or the \xelem{ObservationModel} 
will be established.


\section{Examples}
We start with the most popular PREDPP, ADVAN1-4 and 10-12, examples and go over later to 
more complex administrations arrangements.

%%%%%%%%%%%%%%%%%%%%%%%%%%%%%%%%%%%%%%%%%%%%%%%%%%%%%%%%%%%%%%%%%%%%%%
\subsection{Predefined PREDPP models using PK macros and PharmML}
\label{subsec:PREDPPinMACROS}

This section describes PREDPP library models as encoded in routines ADVAN1-4 and 10-12 
and their PK macros implementation. 


As explained in Section \ref{subsec:LinkingMacrosDatasets} there are differences between 
NONMEM and MONOLIX and their datasets one has to consider when dealing with PK macros.
They manifest themself in that way how the transfer rate constants are numbered/named. 
Models with more then two compartments and with 1st order input (using \emph{DEPOT} 
compartment) will have different transfer rate symbols, e.g.:
\begin{itemize}
\item
ADVAN 4
\begin{table}[ht]
\begin{center}
\begin{tabular}{lcc}
  \hline
  \hline
				 	& PREDPP routines 	& PK macros \\
  \hline
transfer rate constants 	& k23, k32 		& k12, k21  \\
   \hline
\end{tabular}
\end{center}
\end{table}

\item
ADVAN 12 
\begin{table}[ht]
\begin{center}
\begin{tabular}{lcc}
  \hline
  \hline
				 	& PREDPP routines 	& PK macros \\
  \hline
transfer rate constants 	& k23, k32 		& k12, k21  \\
					& k24, k42 		& k13, k31  \\
   \hline
\end{tabular}
\end{center}
\end{table}
\end{itemize}
Models described in this section use the MLXTRAN numbering convention in order to comply
with the PK macros notation.


To allow for a lossless translation between PREDPP coded models and PK macros one needs
to established well defined translation rules. To achieve this goal and provide help to other 
translation/converter teams we need to analyse the predefined PREDPP routines ADVAN1-4 \& 10-12
and more specifically to
\begin{enumerate}
\item
agree and/or understand what they mean
\item
dissect the routines into elementary components and finally 
\item
find equivalent formulation for each such element in the PK macro speak.
\end{enumerate}
The above steps are illustrated in Table \ref{tab:ADVAN_translation} in an abbreviated form. 
In columns \emph{Interpretation} and \emph{Elements} first two steps are implemented, 
the third column provides the according PK macro formulation.

\begin{center}
\begin{longtable}{lll}
  \hline
  \hline
ADVAN routine & Interpretation & MLXTRAN macro \\
\& default TRANS1 &		& \\
%ADVAN\_X & Interpretation & Elements & MLXTRAN macro \\
%(w. TRANS1)		&			 &		    &	\\
\hline
\multicolumn{3}{c}{one compartment}  \\[.1ex]
\hline
%1-comp, IV input
\lstset{language=NONMEMdataSet}
\begin{lstlisting}
ADVAN1
compartment
\end{lstlisting}
&
\lstset{language=Elements}
\begin{lstlisting}
- 1 compartment

- iv bolus administration

- linear elimination
\end{lstlisting}
& 
\lstset{language=MLXTRANcode}
\begin{lstlisting}
compartment(cmt=1, amount=Ac, volume=V)

iv(adm=1, cmt=1)

elimination(cmt=1, k)
\end{lstlisting} 

\\
& 
\\
\hdashline


%1-comp, 1st order input
\lstset{language=NONMEMdataSet}
\begin{lstlisting}
ADVAN2
\end{lstlisting}
&
\lstset{language=Elements}
\begin{lstlisting}
- 1 compartment

- oral administration, 1st order absorption

- linear elimination
\end{lstlisting}
%&
%
&
\lstset{language=MLXTRANcode}
\begin{lstlisting}
compartment(cmt=1, amount=Ac, volume=V)

oral(adm=1, cmt=1, ka)

elimination(cmt=1, k)
\end{lstlisting}

\\
& 
\\
\hdashline

%1-comp, IV input 
%with saturable elimination
\lstset{language=NONMEMdataSet}
\begin{lstlisting}
ADVAN10
\end{lstlisting}
&
\lstset{language=Elements}
\begin{lstlisting}
- 1 compartment

- iv bolus administration

- saturable elimination
\end{lstlisting}
%&
%
&
\lstset{language=MLXTRANcode}
\begin{lstlisting}
compartment(cmt=1, amount=Ac, volume=V)

iv(adm=1, cmt=1)

elimination(cmt=1, Km, Vm)
\end{lstlisting}

\\
\hline
\multicolumn{3}{c}{two compartments}  \\[.1ex]

  \hline

%2-comp, IV input
\lstset{language=NONMEMdataSet}
\begin{lstlisting}
ADVAN3 
\end{lstlisting}
&
\lstset{language=Elements}
\begin{lstlisting}
- 2 compartments 
one central (1) & one peripheral with
linear transfer rates

- iv bolus administration (into 1)

- linear elimination (from 1)
\end{lstlisting}
%&
%
&
\lstset{language=MLXTRANcode}
\begin{lstlisting}
compartment(cmt=1, amount=Ac, volume=V)
peripheral(k12, k21, amount=Ap)


iv(adm=1, cmt=1)

elimination(cmt=1, k)
\end{lstlisting}

\\
& 
\\
\hdashline


%2-comp, 1st order input 
\lstset{language=NONMEMdataSet}
\begin{lstlisting}
ADVAN4 
\end{lstlisting}
&
\lstset{language=Elements}
\begin{lstlisting}
- 2 compartments 
one central (1) & one peripheral with
linear transfer rates

- oral administration, 1st order absorption 
(into 1)

- linear elimination (from 1)
\end{lstlisting}
%&
%
&
\lstset{language=MLXTRANcode}
\begin{lstlisting}
compartment(cmt=1, amount=Ac, volume=V)
peripheral(k12, k21, amount=Ap)


oral(adm=1, cmt=1, ka)

elimination(cmt=1, k)
\end{lstlisting}

\\
\hline
\multicolumn{3}{c}{three compartments}  \\[.1ex]
  \hline
  
%3-comp, IV input 
\lstset{language=NONMEMdataSet}
\begin{lstlisting}
ADVAN11
\end{lstlisting}
&
\lstset{language=Elements}
\begin{lstlisting}
- 3 compartments 
one central (1) & two peripheral with
linear transfer rates

- iv bolus administration (into 1)

- linear elimination (from 1)
\end{lstlisting}
%&
%
&
\lstset{language=MLXTRANcode}
\begin{lstlisting}
compartment(cmt=1, amount=Ac, volume=V)
peripheral(k12, k21, amount=Ap1)
peripheral(k13, k31, amount=Ap2)

iv(adm=1, cmt=1)

elimination(cmt=1, k)
\end{lstlisting}

\\
& 
\\
\hdashline

%3-comp, 1st order input 
\lstset{language=NONMEMdataSet}
\begin{lstlisting}
ADVAN12
\end{lstlisting}
&
\lstset{language=Elements}
\begin{lstlisting}
- 3 compartments 
one central (1) & two peripheral with
linear transfer rates

- oral administration, 1st order absorption 
(into 1)

- linear elimination (from 1)
\end{lstlisting}
%&
%
&
\lstset{language=MLXTRANcode}
\begin{lstlisting}
compartment(cmt=1, amount=Ac, volume=V)
peripheral(k12, k21, amount=Ap1)
peripheral(k13, k31, amount=Ap2)

oral(adm=1, cmt=1, ka)

elimination(cmt=1, k)
\end{lstlisting}
    \\ [+1ex]
      \hline
      \\
\caption{Interpretation and translation of PREDPP models using ADVAN1-4 \& 10-12 routines, 
with default TRANS1 parameterization, into PK macros.}
\label{tab:ADVAN_translation}
\end{longtable}
\end{center}

\begin{figure}[htbp!]
\centering
 \includegraphics[width=170mm]{pics/AdvanMacrosMaster}
\caption{The overview of the ADVAN models, with their default parameterization, 
and their connection to PK macros based on the analysis in Table \ref{tab:ADVAN_translation}. 
Each ADVAN model can be uniquely mapped to macros and their attributes. For more 
details see Sections \ref{subsubsec:ADVAN1} to \ref{subsubsec:ADVAN12}.}
\label{fig:Advan1}
\end{figure}


\cleardoublepage
%%%%%%%%%%%%%%%%%%%%%%%%%%%%%%%%%%%%%%%%%%%%%%%%%%%%%%%%%%%%%%%%%%%%%%
\subsubsection{ADVAN1, TRANS1 -- 1-comp IV input}
\label{subsubsec:ADVAN1}
ODE formulation:
\begin{align}
\frac{dAc}{dt} &= - k \times Ac  \nonumber
\end{align}


\begin{figure}[htbp!]
\centering
 \includegraphics[width=100mm]{pics/Advan1}
\caption{ADVAN1 model, in this case compartment numbering is identical.}
\label{fig:Advan1}
\end{figure}

\begin{table}[ht!]
\footnotesize
\parbox{.5\linewidth}{
\centering
\begin{tabular}{ccccc}
  \hline
   \multicolumn{5}{c}{\textbf{MONOLIX}} \\
  \hline
ID & TIME & AMT & \textbf{ADM} & Y  \\
  \hline
1  & 0        & 10   & \textbf{1} & .       \\
1  & 2        & .      & .   & 5      \\
... &  ...      &  ...   &  ...   & ...  \\
\end{tabular}
}
\hfill
\parbox{.5\linewidth}{
\centering
\begin{tabular}{ccccc}
  \hline
   \multicolumn{5}{c}{\textbf{NONMEM}} \\
  \hline
ID & TIME & AMT & \textbf{\textcolor{red}{CMT}} & DV \\
  \hline
1  & 0        & 10   & \textbf{\textcolor{red}{1}}   & .    \\
1  & 2        & .      & 1    & 5   \\
... &  ...      &  ...   &  ... & ...  \\
\end{tabular}
}
\caption{MONOLIX and NONMEM datasets for the ADVAN1 model.}
\end{table}


\begin{table}[h!]
\setlength{\tabcolsep}{15pt}
\begin{center}
\begin{tabular}{l}
  \hline \hline
PK macro \\[-.25ex]
  \hline
\lstset{language=NONMEMdataSet}
\begin{lstlisting}
input = {V, k}
PK:
compartment(cmt=1, amount=Ac, volume=V)
iv(adm=1, cmt=1)
elimination(cmt=1, k)
\end{lstlisting}
%&
%\lstset{language=NONMEMdataSet}
%\begin{lstlisting}
%input = {V, k}
%PK:
%compartment(cmt=1, amount=Ac, volume=V)
%iv(adm=1, cmt=1)
%elimination(cmt=1, k)
%\end{lstlisting} 
\\
  \hline
\end{tabular}
\caption{PK macros  for the ADVAN1 model, as shown in Figure \ref{fig:ComplexModel1a} (left).}
\label{tab:advan1Table}
\end{center}
\end{table}


PharmML code\footnote{Only the ADVAN1 code example contains the parameter model, \xatt{pm1}, in which
all model parameters has to be defined. It is omitted in the remaining cases for simplicity.}
% The same holds for the \xelem{StructuralModel} tags}:
\lstset{language=XML}
\begin{lstlisting}
        <ParameterModel blkId="pm1">
            <!-- no IIV assumed for simplicity -->
            <SimpleParameter symbId="k"/>
            <SimpleParameter symbId="V"/>
        </ParameterModel>

        <StructuralModel blkId="sm1">
            <ct:Variable symbolType="real" symbId="Ac"/>
            
            <PKmacros>
                <Compartment>
                    <Value argument="cmt">
                        <ct:Int>1</ct:Int>
                    </Value>
                    <Value argument="amount">
                        <ct:SymbRef symbIdRef="Ac"/>
                    </Value>
                    <Value argument="volume">
                        <ct:SymbRef blkIdRef="pm1" symbIdRef="V"/>
                    </Value>
                </Compartment>
                
                <IV>
                    <Value argument="adm">
                        <ct:Int>1</ct:Int>
                    </Value>
                    <Value argument="cmt">
                        <ct:Int>1</ct:Int>
                    </Value>
                </IV>
                
                <Elimination>
                    <Value argument="cmt">
                        <ct:Int>1</ct:Int>
                    </Value>
                    <Value>
                        <ct:SymbRef blkIdRef="pm1" symbIdRef="k"/>
                    </Value>
                </Elimination>
            </PKmacros>
        </StructuralModel>
 \end{lstlisting}


%\cleardoublepage
%%%%%%%%%%%%%%%%%%%%%%%%%%%%%%%%%%%%%%%%%%%%%%%%%%%%%%%%%%%%%%%%%%%%%%
\subsubsection{ADVAN2, TRANS1 -- 1-comp 1st order input}
\label{subsubsec:advan2}
ODE formulation:
\begin{align}
\frac{dAd}{dt} &= -ka \times Ad \nonumber \\
\frac{dAc}{dt} &= ka \times Ad - k \times Ac  \nonumber
\end{align}


\begin{figure}[htbp!]
\centering
 \includegraphics[width=100mm]{pics/Advan2}
\caption{ADVAN2 model, with compartment numbering dependent on the target tool.}
\label{fig:Advan2}
\end{figure}


\begin{table}[ht!]
\footnotesize
\parbox{.5\linewidth}{
\centering
\begin{tabular}{ccccc}
  \hline
   \multicolumn{5}{c}{\textbf{MONOLIX}} \\
  \hline
ID & TIME & AMT & \textbf{ADM} & Y \\
  \hline
1  & 0        & 10   & \textbf{1} & .       \\
1  & 2        & .      & .     & 5      \\
... &  ...      &  ...   &  ...  & ...   \\
\end{tabular}
}
\hfill
\parbox{.5\linewidth}{
\centering
\begin{tabular}{ccccc}
  \hline
   \multicolumn{5}{c}{\textbf{NONMEM}} \\
  \hline
ID & TIME & AMT & \textbf{\textcolor{red}{CMT}} & DV \\
  \hline
1  & 0        & 10   & \textbf{\textcolor{red}{1}}   & .    \\
1  & 2        & .      & 2    & 5   \\
... &  ...      &  ...   &  ... & ...  \\
\end{tabular}
}
\caption{MONOLIX and NONMEM datasets for the ADVAN2 model.}
\end{table}

\lstset{language=MLXTRANcode}
\begin{lstlisting}
\end{lstlisting}

\begin{table}[h!]
\setlength{\tabcolsep}{15pt}
\begin{center}
%\begin{tabular*}{.95\textwidth}{@{\extracolsep{\fill} } ll}
\begin{tabular}{l}
  \hline \hline
PK macro  \\[-.25ex]
  \hline
\lstset{language=NONMEMdataSet}
\begin{lstlisting}
input = {ka, V, k}
PK:
compartment(cmt=1, amount=Ac, volume=V)
oral(adm=1, cmt=1, ka)
elimination(cmt=1, k)
\end{lstlisting}
%&
%\lstset{language=NONMEMdataSet}
%\begin{lstlisting}
%input = {ka, V, k}
%PK:
%compartment(cmt=1, amount=Ac, volume=V)
%[*compartment(cmt=2, amount=Depot)*]
%oral(adm=1, [*fromCmt=2,*] cmt=1, ka)
%elimination(cmt=1, k)
%\end{lstlisting} 
\\
  \hline
\end{tabular}
\caption{PK macros  for the ADVAN2 model, as shown in Figure \ref{fig:Advan2} (left).}
\label{tab:advan2Table}
\end{center}
\end{table}

PharmML code:
\lstset{language=XML}
\begin{lstlisting}
        <StructuralModel blkId="sm2">
            <ct:Variable symbolType="real" symbId="Ac"/>
            
            <PKmacros>
                <Compartment>
                    <Value argument="cmt">
                        <ct:Int>1</ct:Int>
                    </Value>
                    <Value argument="amount">
                        <ct:SymbRef symbIdRef="Ac"/>
                    </Value>
                    <Value argument="volume">
                        <ct:SymbRef blkIdRef="pm1" symbIdRef="V"/>
                    </Value>
                </Compartment>
                
                <Oral>
                    <Value argument="adm">
                        <ct:Int>1</ct:Int>
                    </Value>
                    <Value argument="cmt">
                        <ct:Int>1</ct:Int>
                    </Value>
                    <Value>
                        <ct:SymbRef blkIdRef="pm1" symbIdRef="ka"/>
                    </Value>
                </Oral>
                
                <Elimination>
                    <Value argument="cmt">
                        <ct:Int>1</ct:Int>
                    </Value>
                    <Value>
                        <ct:SymbRef blkIdRef="pm1" symbIdRef="k"/>
                    </Value>
                </Elimination>
            </PKmacros>
        </StructuralModel>
\end{lstlisting}


%\cleardoublepage
%%%%%%%%%%%%%%%%%%%%%%%%%%%%%%%%%%%%%%%%%%%%%%%%%%%%%%%%%%%%%%%%%%%%%%
\subsubsection{ADVAN3, TRANS1 -- 2-comp IV input}
ODE formulation:
\begin{align}
\frac{dAc}{dt} &= -k_{12}\times Ac + k_{21} \times Ap - k \times Ac  \nonumber \\
\frac{dAp}{dt} &= k_{12}\times Ac - k_{21} \times Ap  \nonumber
\end{align}

\begin{figure}[htbp!]
\centering
 \includegraphics[width=110mm]{pics/Advan3}
\caption{ADVAN3 model, in this case compartment numbering is identical.}
\label{fig:Advan3}
\end{figure}


\begin{table}[ht!]
\footnotesize
\parbox{.5\linewidth}{
\centering
\begin{tabular}{ccccc}
  \hline
   \multicolumn{5}{c}{\textbf{MONOLIX}} \\
  \hline
ID & TIME & AMT & \textbf{ADM} & Y \\
  \hline
1  & 0        & 10   & \textbf{1} & .       \\
1  & 2        & .      & .    & 5    \\
... &  ...      &  ...   &  ... & ...    \\
\end{tabular}
}
\hfill
\parbox{.5\linewidth}{
\centering
\begin{tabular}{ccccc}
  \hline
   \multicolumn{5}{c}{\textbf{NONMEM}} \\
  \hline
ID & TIME & AMT & \textbf{CMT} & DV \\
  \hline
1  & 0        & 10   & \textbf{1}   & .    \\
1  & 2        & .      & 2    & 5   \\
... &  ...      &  ...   &  ... & ...  \\
\end{tabular}
}
\caption{MONOLIX and NONMEM datasets for the ADVAN3 model.}
\end{table}


\begin{table}[h!]
\setlength{\tabcolsep}{15pt}
\begin{center}
%\begin{tabular*}{.95\textwidth}{@{\extracolsep{\fill} } ll}
\begin{tabular}{l}
  \hline \hline
PK macro \\[-.25ex]
  \hline
\lstset{language=NONMEMdataSet}
\begin{lstlisting}
input = {V, k, k12, k21}
PK:
compartment(cmt=1, amount=Ac, volume=V)
peripheral(k12, k21, amount=Ap)
iv(adm=1, cmt=1)
elimination(cmt=1, k)
\end{lstlisting}
%&
%\lstset{language=NONMEMdataSet}
%\begin{lstlisting}
%input = {V, k, k12, k21}
%PK:
%compartment(cmt=1, amount=Ac, volume=V)
%peripheral(k12, k21, amount=Ap)
%iv(adm=1, cmt=1)
%elimination(cmt=1, k)
%\end{lstlisting} 
\\
  \hline
\end{tabular}
\caption{PK macros  for the ADVAN3 model, as shown in Figure \ref{fig:Advan3} (left).}
\label{tab:advan3Table}
\end{center}
\end{table}


PharmML code:
\lstset{language=XML}
\begin{lstlisting}
        <StructuralModel blkId="sm3">
            
            <ct:Variable symbolType="real" symbId="Ac"/>
            <ct:Variable symbolType="real" symbId="Ap"/>
            
            <PKmacros>
                <Compartment>
                    <Value argument="cmt">
                        <ct:Int>1</ct:Int>
                    </Value>
                    <Value argument="amount">
                        <ct:SymbRef symbIdRef="Ac"/>
                    </Value>
                    <Value argument="volume">
                        <ct:SymbRef symbIdRef="V"/>
                    </Value>
                </Compartment>
                
                <Peripheral>
                    <Value>
                        <ct:SymbRef symbIdRef="k12"/>
                    </Value>
                    <Value>
                        <ct:SymbRef symbIdRef="k21"/>
                    </Value>
                    <Value argument="amount">
                        <ct:SymbRef symbIdRef="Ap"/>
                    </Value>
                </Peripheral>
                
                <IV>
                    <Value argument="adm">
                        <ct:Int>1</ct:Int>
                    </Value>
                    <Value argument="cmt">
                        <ct:Int>1</ct:Int>
                    </Value>
                </IV>
                
                <Elimination>
                    <Value argument="cmt">
                        <ct:Int>1</ct:Int>
                    </Value>
                    <Value>
                        <ct:SymbRef symbIdRef="k"/>
                    </Value>
                </Elimination>
            </PKmacros>
        </StructuralModel>
\end{lstlisting}


%\cleardoublepage
%%%%%%%%%%%%%%%%%%%%%%%%%%%%%%%%%%%%%%%%%%%%%%%%%%%%%%%%%%%%%%%%%%%%%%
\subsubsection{ADVAN4, TRANS1 -- 2-comp 1st order input}
ODE formulation:
\begin{align}
\frac{dAd}{dt} &= -ka \times Ad \nonumber \\
\frac{dAc}{dt} &= ka \times Ad - k_{12} \times Ac + k_{21} \times Ap - k \times Ac  \nonumber \\
\frac{dAp}{dt} &= k_{12} \times Ac - k_{21} \times Ap  \nonumber
\end{align}

\begin{figure}[htbp!]
\centering
 \includegraphics[width=110mm]{pics/Advan4}
\caption{ADVAN4 model, with compartment numbering dependent on the target tool. 
Note, that not only the compartment numbers are different in NONMEM coded model, 
the rate constants names are different as well.}
\label{fig:Advan4}
\end{figure}


\begin{table}[ht!]
\footnotesize
\parbox{.5\linewidth}{
\centering
\begin{tabular}{ccccc}
  \hline
   \multicolumn{5}{c}{\textbf{MONOLIX}} \\
  \hline
ID & TIME & AMT & \textbf{ADM} & Y \\
  \hline
1  & 0        & 10   & \textbf{1} & .       \\
1  & 2        & .      & . & 5        \\
... &  ...      &  ...   &  ...&  ...     \\
\end{tabular}
}
\hfill
\parbox{.5\linewidth}{
\centering
\begin{tabular}{ccccc}
  \hline
   \multicolumn{5}{c}{\textbf{NONMEM}} \\
  \hline
ID & TIME & AMT & \textbf{\textcolor{red}{CMT}} & DV \\
  \hline
1  & 0        & 10   & \textbf{\textcolor{red}{1}}   & .    \\
1  & 2        & .      & 2    & 5   \\
... &  ...      &  ...   &  ... & ...  \\
\end{tabular}
}
\caption{MONOLIX and NONMEM datasets for the ADVAN4 model.}
\end{table}

\begin{table}[h!]
\setlength{\tabcolsep}{15pt}
\begin{center}
%\begin{tabular*}{.95\textwidth}{@{\extracolsep{\fill} } ll}
\begin{tabular}{l}
  \hline \hline
PK macro  \\[-.25ex]
  \hline
\lstset{language=NONMEMdataSet}
\begin{lstlisting}
input = {ka, V, k, k12, k21}
PK:
compartment(cmt=1, amount=Ac, volume=V)
peripheral(k12, k21, amount=Ap)
oral(adm=1, cmt=1, ka)
elimination(cmt=1, k)
\end{lstlisting}
%&
%\lstset{language=NONMEMdataSet}
%\begin{lstlisting}
%input = {ka, V, k, k12, k21}
%PK:
%compartment(cmt=1, amount=Ac, volume=V)
%[*compartment(cmt=3, amount=Depot)*]
%peripheral(k12, k21, amount=Ap)
%oral(adm=1, [*fromCmt=3,*] cmt=1, ka)
%elimination(cmt=1, k)
%\end{lstlisting} 
\\
  \hline
\end{tabular}
\caption{PK macros  for the ADVAN4 model, as shown in Figure \ref{fig:Advan4} (left).}
\label{tab:advan4Table}
\end{center}
\end{table}


PharmML code:
\lstset{language=XML}
\begin{lstlisting}
        <StructuralModel blkId="sm4">
            <ct:Variable symbolType="real" symbId="Ac"/>
            <ct:Variable symbolType="real" symbId="Ap"/>
            <ct:Variable symbolType="real" symbId="Cc">
                <ct:Assign>
                    <math:Equation>
                        <math:Binop op="divide">
                            <ct:SymbRef symbIdRef="Ac"/>
                            <ct:SymbRef blkIdRef="pm1" symbIdRef="V"/>
                        </math:Binop>
                    </math:Equation>
                </ct:Assign>
            </ct:Variable>
            
            <PKmacros>
                <Compartment>
                    <Value argument="cmt">
                        <ct:Int>1</ct:Int>
                    </Value>
                    <Value argument="amount">
                        <ct:SymbRef symbIdRef="Ac"/>
                    </Value>
                    <Value argument="volume">
                        <ct:SymbRef blkIdRef="pm1" symbIdRef="V"/>
                    </Value>
                </Compartment>
                
                <Peripheral>
                    <Value>
                        <ct:SymbRef blkIdRef="pm1" symbIdRef="k12"/>
                    </Value>
                    <Value>
                        <ct:SymbRef blkIdRef="pm1" symbIdRef="k21"/>
                    </Value>
                    <Value argument="amount">
                        <ct:SymbRef symbIdRef="Ap"/>
                    </Value>
                </Peripheral>
                
                <Oral>
                    <Value argument="adm">
                        <ct:Int>1</ct:Int>
                    </Value>
                    <Value argument="cmt">
                        <ct:Int>1</ct:Int>
                    </Value>
                    <Value>
                        <ct:SymbRef blkIdRef="pm1" symbIdRef="ka"/>
                    </Value>
                </Oral>
                
                <Elimination>
                    <Value argument="cmt">
                        <ct:Int>1</ct:Int>
                    </Value>
                    <Value>
                        <ct:SymbRef blkIdRef="pm1" symbIdRef="k"/>
                    </Value>
                </Elimination>
            </PKmacros>
        </StructuralModel>
\end{lstlisting}


%\cleardoublepage
%%%%%%%%%%%%%%%%%%%%%%%%%%%%%%%%%%%%%%%%%%%%%%%%%%%%%%%%%%%%%%%%%%%%%%
\subsubsection{ADVAN10, TRANS1 -- 1-comp IV input with saturable elimination}
ODE formulation:
\begin{align}
\frac{dAc}{dt} &= - \frac{Vm \times Ac}{Km + Ac}  \nonumber
\end{align}

\begin{figure}[htbp!]
\centering
 \includegraphics[width=100mm]{pics/Advan10}
\caption{ADVAN10 model, in this case compartment numbering is identical.}
\label{fig:Advan10}
\end{figure}


\begin{table}[ht!]
\footnotesize
\parbox{.5\linewidth}{
\centering
\begin{tabular}{ccccc}
  \hline
   \multicolumn{5}{c}{\textbf{MONOLIX}} \\
  \hline
ID & TIME & AMT & \textbf{ADM} & Y \\
  \hline
1  & 0        & 10   & \textbf{1} & .       \\
1  & 2        & .      &	.	& 5        \\
... &  ...      &  ...   &  ... &  ...     \\
\end{tabular}
}
\hfill
\parbox{.5\linewidth}{
\centering
\begin{tabular}{ccccc}
  \hline
   \multicolumn{5}{c}{\textbf{NONMEM}} \\
  \hline
ID & TIME & AMT & \textbf{CMT} & DV \\
  \hline
1  & 0        & 10   & \textbf{1}   & .    \\
1  & 2        & .      & 1    & 5   \\
... &  ...      &  ...   &  ... & ...  \\
\end{tabular}
}
\caption{MONOLIX and NONMEM datasets for the ADVAN10 model.}
\end{table}


\begin{table}[h!]
\setlength{\tabcolsep}{15pt}
\begin{center}
%\begin{tabular*}{.95\textwidth}{@{\extracolsep{\fill} } ll}
\begin{tabular}{l}
  \hline \hline
PK macro \\[-.25ex]
  \hline
\lstset{language=NONMEMdataSet}
\begin{lstlisting}
input = {V, Km, Vm}
PK:
compartment(cmt=1, amount=Ac, volume=V)
iv(adm=1, cmt=1)
elimination(cmt=1, Km, Vm)
\end{lstlisting}
%&
%\lstset{language=NONMEMdataSet}
%\begin{lstlisting}
%input = {V, Km, Vm}
%PK:
%compartment(cmt=1, amount=Ac, volume=V)
%iv(adm=1, cmt=1)
%elimination(cmt=1, Km, Vm)
%\end{lstlisting} 
\\
  \hline
\end{tabular}
\caption{PK macros for the ADVAN10 model, as shown in Figure \ref{fig:Advan10} (left).}
\label{tab:advan10Table}
\end{center}
\end{table}


PharmML code:
\lstset{language=XML}
\begin{lstlisting}
        <StructuralModel blkId="sm10">
            <ct:Variable symbolType="real" symbId="Ac"/>
            <ct:Variable symbolType="real" symbId="Cc">
                <ct:Assign>
                    <math:Equation>
                        <math:Binop op="divide">
                            <ct:SymbRef symbIdRef="Ac"/>
                            <ct:SymbRef blkIdRef="pm1" symbIdRef="V"/>
                        </math:Binop>
                    </math:Equation>
                </ct:Assign>
            </ct:Variable>
            
            <PKmacros>
                <Compartment>
                    <Value argument="cmt">
                        <ct:Int>1</ct:Int>
                    </Value>
                    <Value argument="amount">
                        <ct:SymbRef symbIdRef="Ac"/>
                    </Value>
                    <Value argument="volume">
                        <ct:SymbRef blkIdRef="pm1" symbIdRef="V"/>
                    </Value>
                </Compartment>
                
                <IV>
                    <Value argument="adm">
                        <ct:Int>1</ct:Int>
                    </Value>
                    <Value argument="cmt">
                        <ct:Int>1</ct:Int>
                    </Value>
                </IV>
                
                <Elimination>
                    <Value argument="cmt">
                        <ct:Int>1</ct:Int>
                    </Value>
                    <Value>
                        <ct:SymbRef blkIdRef="pm1" symbIdRef="Km"/>
                    </Value>
                    <Value>
                        <ct:SymbRef blkIdRef="pm1" symbIdRef="Vm"/>
                    </Value>
                </Elimination>
            </PKmacros>
        </StructuralModel>
\end{lstlisting}


%\cleardoublepage
%%%%%%%%%%%%%%%%%%%%%%%%%%%%%%%%%%%%%%%%%%%%%%%%%%%%%%%%%%%%%%%%%%%%%%
\subsubsection{ADVAN11, TRANS1 -- 3-comp IV input}
\label{subsubsec:ADVAN11}
ODE formulation:
\begin{align} % placement: default is "center", options are "top" and "bottom"
\frac{dAc}{dt} & =  - k_{12} \times Ac + k_{21} \times Ap1- k_{13} \times Ac \nonumber \\
			& + k_{31} \times Ap2 - k \times Ac  \nonumber \\
\frac{dAp1}{dt} & =  k_{12} \times Ac - k_{21} \times Ap1  \nonumber \\
\frac{dAp2}{dt} & =  k_{13} \times Ac - k_{31} \times Ap2  \nonumber
\end{align} 


\begin{figure}[htbp!]
\centering
 \includegraphics[width=160mm]{pics/Advan11}
\caption{ADVAN11 model, in this case compartment numbering is identical.}
\label{fig:Advan11}
\end{figure}


\begin{table}[ht!]
\footnotesize
\parbox{.5\linewidth}{
\centering
\begin{tabular}{ccccc}
  \hline
   \multicolumn{5}{c}{\textbf{MONOLIX}} \\
  \hline
ID & TIME & AMT & \textbf{ADM} & Y \\
  \hline
1  & 0        & 10   & \textbf{1} & .       \\
1  & 2        & .      & . 	& 5        \\
... &  ...      &  ...   &  ...   &  ...    \\
\end{tabular}
}
\hfill
\parbox{.5\linewidth}{
\centering
\begin{tabular}{ccccc}
  \hline
   \multicolumn{5}{c}{\textbf{NONMEM}} \\
  \hline
ID & TIME & AMT & \textbf{CMT} & DV \\
  \hline
1  & 0        & 10   & \textbf{1}   & .    \\
1  & 2        & .      & 1    & 5   \\
... &  ...      &  ...   &  ... & ...  \\
\end{tabular}
}
\caption{MONOLIX and NONMEM datasets for the ADVAN11 model.}
\end{table}


\begin{table}[h!]
\setlength{\tabcolsep}{15pt}
\begin{center}
%\begin{tabular*}{.95\textwidth}{@{\extracolsep{\fill} } ll}
\begin{tabular}{c}
  \hline \hline
PK macro \\[-.25ex]
  \hline
\lstset{language=NONMEMdataSet}
\begin{lstlisting}
input = {V, k, k12, k21, k13, k31}
PK:
compartment(cmt=1, amount=Ac, volume=V)
peripheral(k12, k21, amount=Ap1)
peripheral(k13, k31, amount=Ap2)
iv(adm=1, cmt=1)
elimination(cmt=1, k)
\end{lstlisting}
%&
%\lstset{language=NONMEMdataSet}
%\begin{lstlisting}
%input = {V, k, k12, k21, k13, k31}
%PK:
%compartment(cmt=1, amount=Ac, volume=V)
%peripheral(k12, k21, amount=Ap1)
%peripheral(k13, k31, amount=Ap2)
%iv(adm=1, cmt=1)
%elimination(cmt=1, k)
%\end{lstlisting} 
\\
  \hline
\end{tabular}
\caption{PK macros  for the ADVAN11 model, as shown in Figure \ref{fig:Advan11} (left).}
\label{tab:advan11Table}
\end{center}
\end{table}


PharmML code:
\lstset{language=XML}
\begin{lstlisting}
        <StructuralModel blkId="sm11">
            <ct:Variable symbolType="real" symbId="Ac"/>
            <ct:Variable symbolType="real" symbId="Ap1"/>
            <ct:Variable symbolType="real" symbId="Ap2"/>
            <ct:Variable symbolType="real" symbId="Cc">
                <ct:Assign>
                    <math:Equation>
                        <math:Binop op="divide">
                            <ct:SymbRef symbIdRef="Ac"/>
                            <ct:SymbRef blkIdRef="pm1" symbIdRef="V"/>
                        </math:Binop>
                    </math:Equation>
                </ct:Assign>
            </ct:Variable>
            
            <PKmacros>
                <Compartment>
                    <Value argument="cmt">
                        <ct:Int>1</ct:Int>
                    </Value>
                    <Value argument="amount">
                        <ct:SymbRef symbIdRef="Ac"/>
                    </Value>
                    <Value argument="volume">
                        <ct:SymbRef blkIdRef="pm1" symbIdRef="V"/>
                    </Value>
                </Compartment>
                
                <Peripheral>
                    <Value>
                        <ct:SymbRef blkIdRef="pm1" symbIdRef="k12"/>
                    </Value>
                    <Value>
                        <ct:SymbRef blkIdRef="pm1" symbIdRef="k21"/>
                    </Value>
                    <Value argument="amount">
                        <ct:SymbRef symbIdRef="Ap1"/>
                    </Value>
                </Peripheral>
                
                <Peripheral>
                    <Value>
                        <ct:SymbRef blkIdRef="pm1" symbIdRef="k13"/>
                    </Value>
                    <Value>
                        <ct:SymbRef blkIdRef="pm1" symbIdRef="k31"/>
                    </Value>
                    <Value argument="amount">
                        <ct:SymbRef symbIdRef="Ap2"/>
                    </Value>
                </Peripheral>
                
                <IV>
                    <Value argument="adm">
                        <ct:Int>1</ct:Int>
                    </Value>
                    <Value argument="cmt">
                        <ct:Int>1</ct:Int>
                    </Value>
                </IV>
                
                <Elimination>
                    <Value argument="cmt">
                        <ct:Int>1</ct:Int>
                    </Value>
                    <Value>
                        <ct:SymbRef blkIdRef="pm1" symbIdRef="k"/>
                    </Value>
                </Elimination>
            </PKmacros>
        </StructuralModel>
\end{lstlisting}


%\cleardoublepage
%%%%%%%%%%%%%%%%%%%%%%%%%%%%%%%%%%%%%%%%%%%%%%%%%%%%%%%%%%%%%%%%%%%%%%
\subsubsection{ADVAN12, TRANS1 -- 3-comp 1st order input}
\label{subsubsec:ADVAN12}
ODE formulation:
\begin{align}
\frac{dAd}{dt} & =  -ka \times Ad \nonumber \\
\frac{dAc}{dt} & = ka \times Ad - k_{12} \times Ac + k_{21} \times Ap1 - k_{13} \times Ac \nonumber \\
			& + k_{31} \times Ap2 - k \times Ac  \nonumber \\
\frac{dAp1}{dt} & =  k_{12} \times Ac - k_{21} \times Ap1  \nonumber \\
\frac{dAp2}{dt} & =  k_{13} \times Ac - k_{31} \times Ap2  \nonumber
\end{align}

\begin{figure}[htbp!]
\centering
 \includegraphics[width=160mm]{pics/Advan12}
\caption{ADVAN12 model, with compartment numbering dependent on the target tool. 
Note, that not only the compartment numbers are different in NONMEM coded model, 
the rate constants names are different as well.}
\label{fig:Advan12}
\end{figure}


\begin{table}[ht!]
\footnotesize
\parbox{.5\linewidth}{
\centering
\begin{tabular}{ccccc}
  \hline
   \multicolumn{5}{c}{\textbf{MONOLIX}} \\
  \hline
ID & TIME & AMT & \textbf{ADM} & Y \\
  \hline
1  & 0        & 10   & \textbf{1} & .       \\
1  & 2        & .      & . 	& 5        \\
... &  ...      &  ...   &  ...  &  ...     \\
\end{tabular}
}
\hfill
\parbox{.5\linewidth}{
\centering
\begin{tabular}{ccccc}
  \hline
   \multicolumn{5}{c}{\textbf{NONMEM}} \\
  \hline
ID & TIME & AMT & \textbf{\textcolor{red}{CMT}} & DV \\
  \hline
1  & 0        & 10   & \textbf{\textcolor{red}{1}}   & .    \\
1  & 2        & .      & 2    & 5   \\
... &  ...      &  ...   &  ... & ...  \\
\end{tabular}
}
\caption{MONOLIX and NONMEM datasets for the ADVAN12 model.}
\end{table}


\begin{table}[h!]
\setlength{\tabcolsep}{15pt}
\begin{center}
%\begin{tabular*}{.95\textwidth}{@{\extracolsep{\fill} } ll}
\begin{tabular}{l}
  \hline \hline
PK macro \\[-.25ex]
  \hline
\lstset{language=NONMEMdataSet}
\begin{lstlisting}
input = {ka, V, k, k12, k21, k13, k31}
PK:
compartment(cmt=1, amount=Ac, volume=V)
peripheral(k12, k21, amount=Ap1)
peripheral(k13, k31, amount=Ap2)
oral(adm=1, cmt=1, ka)
elimination(cmt=1, k)
\end{lstlisting}
\\
  \hline
\end{tabular}
\caption{PK macros  for the ADVAN12 model, as shown in Figure \ref{fig:Advan12} (left).}
\label{tab:advan12Table}
\end{center}
\end{table}


PharmML code:
\lstset{language=XML}
\begin{lstlisting}
        <StructuralModel blkId="sm12">
            <ct:Variable symbolType="real" symbId="Ac"/>
            <ct:Variable symbolType="real" symbId="Ap1"/>
            <ct:Variable symbolType="real" symbId="Ap2"/>
            <ct:Variable symbolType="real" symbId="Cc">
                <ct:Assign>
                    <math:Equation>
                        <math:Binop op="divide">
                            <ct:SymbRef symbIdRef="Ac"/>
                            <ct:SymbRef blkIdRef="pm1" symbIdRef="V"/>
                        </math:Binop>
                    </math:Equation>
                </ct:Assign>
            </ct:Variable>
            
            <PKmacros>
                <Compartment>
                    <Value argument="cmt">
                        <ct:Int>1</ct:Int>
                    </Value>
                    <Value argument="amount">
                        <ct:SymbRef symbIdRef="Ac"/>
                    </Value>
                    <Value argument="volume">
                        <ct:SymbRef blkIdRef="pm1" symbIdRef="V"/>
                    </Value>
                </Compartment>
                
                <Peripheral>
                    <Value>
                        <ct:SymbRef blkIdRef="pm1" symbIdRef="k12"/>
                    </Value>
                    <Value>
                        <ct:SymbRef blkIdRef="pm1" symbIdRef="k21"/>
                    </Value>
                    <Value argument="amount">
                        <ct:SymbRef symbIdRef="Ap1"/>
                    </Value>
                </Peripheral>
                
                <Peripheral>
                    <Value>
                        <ct:SymbRef blkIdRef="pm1" symbIdRef="k13"/>
                    </Value>
                    <Value>
                        <ct:SymbRef blkIdRef="pm1" symbIdRef="k31"/>
                    </Value>
                    <Value argument="amount">
                        <ct:SymbRef symbIdRef="Ap2"/>
                    </Value>
                </Peripheral>
                
                <Oral>
                    <Value argument="adm">
                        <ct:Int>1</ct:Int>
                    </Value>
                    <Value argument="cmt">
                        <ct:Int>1</ct:Int>
                    </Value>
                    <Value>
                        <ct:SymbRef blkIdRef="pm1" symbIdRef="ka"/>
                    </Value>
                </Oral>
                
                <Elimination>
                    <Value argument="cmt">
                        <ct:Int>1</ct:Int>
                    </Value>
                    <Value>
                        <ct:SymbRef blkIdRef="pm1" symbIdRef="k"/>
                    </Value>
                </Elimination>
            </PKmacros>
        </StructuralModel>
\end{lstlisting}

%
%\begin{table}[ht!]
%\setlength{\tabcolsep}{5pt}
%\begin{center}
%\begin{tabular}{ll}
%  \hline
%PK macro 	& PharmML  \\
%  \hline
%%			\multicolumn{2}{c}{Mapping in a \xelem{CategoricalData} model}  \\  [.5ex]
%  \hline
%\lstset{language=XML}
%\begin{lstlisting}
%
%\end{lstlisting}
%&
%\lstset{language=XML}
%\begin{lstlisting}
%
%\end{lstlisting}
%\\
%    \hline
%\end{tabular}
%\caption{Code...}
%\label{tab:exampleXYZ}
%\end{center}
%\end{table}


%%\cleardoublepage
%\subsection{Pavia Whiteboard}
%
%\begin{figure}[htb!]
%\centering
%  \includegraphics[width=130mm]{pics/Pavia1}
% \caption{Blackboard with forcing function discussion.}
% \label{Pavia1}
%\end{figure}


% {\color{red} \scshape{NEW}}
%\cleardoublepage
%%%%%%%%%%%%%%%%%%%%%%%%%%%%%%%%%%%%%%%%%%%%%%%%%%%%%%%
\subsection{Alternative model parameterizations}

In case of routine others then the default TRANS1 the procedure is quite straightforward. 
The reparametrization, called \emph{Reparameterization Lines} in the NONMEM manual
\cite{NONMEM:2009}, can be defined in the \xelem{ParameterModel} leaving the PK macros in the 
\xelem{StructuralModel} unchanged. This is presented using the basic ADVAN1 and more 
complex ADVAN11 routines.

\subsubsection{ADVAN1 with TRANS2}
Following table shows the two allowed parameterizations for ADVAN1 and the according formula. 
\begin{table}[ht]
\begin{center}
\begin{tabular*}{.8\textwidth}{@{\extracolsep{\fill} } lll}
  \hline
  \hline
  TRANS1								& TRANS2						& Formula \\
  \hline
K Rate constant of elimination				& CL Clearance 					& K=CL/V \\
									& V Volume of distribution				& \\
\end{tabular*}
\caption{ADVAN1 parameters with TRANS1 and TRANS2.}
\end{center}
\end{table}

Table \ref{tab:ADVAN1TRANS1reparameterisation} shows the PK macros for the two options
\begin{table}[ht]
\begin{center}
\begin{tabular*}{.8\textwidth}{@{\extracolsep{\fill} } ll}
  \hline
  \hline
ADVAN1, TRANS1 & ADVAN1, TRANS4 \\
  \hline
\lstset{language=NONMEMdataSet}
\begin{lstlisting}
compartment(cmt=1, amount=Ac, volume=V)
iv(cmt=1)
elimination(cmt=1, k)
\end{lstlisting}
&
\lstset{language=NONMEMdataSet}
\begin{lstlisting}
compartment(cmt=1, amount=Ac, volume=V)
iv(cmt=1)
elimination(cmt=1, k=CL/V)
\end{lstlisting}

\end{tabular*}
\caption{PK macros for ADVAN1 using two different parameterizations.}
\label{tab:ADVAN1TRANS1reparameterisation}
\end{center}
\end{table}

\paragraph{Version 1}
PharmML code -- modified from section \ref{subsubsec:ADVAN1}. The reparameterization is done in 
the parameter model \xatt{pm1}:
\lstset{language=XML}
\begin{lstlisting}

        <ParameterModel blkId="pm1">
            
            <!-- Reparameterization Lines for k, CL, V -->
            <SimpleParameter symbId="k">
                <ct:Assign>
                    <math:Equation>
                        <math:Binop op="divide">
                            <ct:SymbRef symbIdRef="CL"/>
                            <ct:SymbRef symbIdRef="V"/>
                        </math:Binop>
                    </math:Equation>
                </ct:Assign>
            </SimpleParameter>
            
        </ParameterModel>

        <StructuralModel blkId="sm1">
            <ct:Variable symbolType="real" symbId="Ac"/>
            <PKmacros>
                    <!-- omitted, identical to ADVAN1, TRANS1 -->
            </PKmacros>
        </StructuralModel>
\end{lstlisting}

\paragraph{Version 2}
PharmML code -- with reparameterization within macros:
\lstset{language=XML}
\begin{lstlisting}
        <StructuralModel blkId="sm1b">
            <ct:Variable symbolType="real" symbId="Ac"/>
            <ct:Variable symbolType="real" symbId="Cc">
                <!-- omitted details -->
            </ct:Variable>
            
            <PKmacros>
                <!-- omitted details -->
                
                <Elimination>
                    <Value argument="cmt">
                        <ct:Int>1</ct:Int>
                    </Value>
                    <Value argument="k">
                        <ct:Assign>
                            <math:Equation>
                                <math:Binop op="divide">
                                    <ct:SymbRef blkIdRef="pm1" symbIdRef="CL"/>
                                    <ct:SymbRef blkIdRef="pm1" symbIdRef="V"/>
                                </math:Binop>
                            </math:Equation>
                        </ct:Assign>
                    </Value>
                </Elimination>
            </PKmacros>
        </StructuralModel>
\end{lstlisting}


\subsubsection{ADVAN11 with TRANS4}
Following table shows the two allowed parameterizations for ADVAN11 and the according 
re-parameterisation formulas. 
\begin{table}[ht]
\begin{center}
\begin{tabular*}{.95\textwidth}{@{\extracolsep{\fill} } lll}
  \hline
  \hline
  TRANS1								& TRANS4						& Formula \\
  \hline
K Rate constant of elimination				& CL Clearance 					& K=CL/V1 \\
K12 Rate constant from central to periph. 1	& V1 Central volume  				& K12=Q2/V1 \\
K21 Rate constant from periph. 1 to central	& Q2 Intercompartmental clearance 1  	& K21=Q2/V2 \\
K13 Rate constant from central to periph. 2	& V2 Peripheral volume 1  			& K13=Q3/V1 \\
K31 Rate constant from periph. 2 to central	& Q3 Intercompartmental clearance 2  	& K31=Q3/V3 \\
									& V3 Peripheral volume 2  			& V3=V3 \\
\end{tabular*}
\caption{ADVAN11 parameters with TRANS1 and TRANS4.}
\end{center}
\end{table}
and the following table shows the PK macros for the two routines

\begin{table}[ht]
\begin{center}
\begin{tabular*}{.95\textwidth}{@{\extracolsep{\fill} } ll}
  \hline
  \hline
ADVAN11, TRANS1 & ADVAN11, TRANS4 \\
  \hline

\lstset{language=NONMEMdataSet}
\begin{lstlisting}
compartment(cmt=1, amount=Ac, volume=V)
peripheral(k12, k21, amount=Ap1)
peripheral(k13, k31, amount=Ap2)
iv(cmt=1)
elimination(cmt=1, k)
\end{lstlisting}
&
\lstset{language=NONMEMdataSet}
\begin{lstlisting}
compartment(cmt=1, amount=Ac, volume=V3)
peripheral(k12 = Q2/V1, k21 = Q2/V2, amount=Ap1)
peripheral(k13 = Q3/V1, k31 = Q3/V3, amount=Ap2)
iv(cmt=1)
elimination(cmt=1, k = CL/V1)
\end{lstlisting}

\end{tabular*}
\caption{PK macros for ADVAN11 using two different parameterizations.}
\end{center}
\end{table}

\paragraph{Version 1}
PharmML code -- modified from section \ref{subsubsec:ADVAN11}. In this case the 
reparameterization is done in the parameter model \xatt{pm1}, here example for one 
parameter:
\lstset{language=XML}
\begin{lstlisting}
        <ParameterModel blkId="pm1">
            
            <!-- Reparameterization Lines for k12, Q, V1 -->
            <SimpleParameter symbId="k12">
                <ct:Assign>
                    <math:Equation>
                        <math:Binop op="divide">
                            <ct:SymbRef symbIdRef="Q2"/>
                            <ct:SymbRef symbIdRef="V1"/>
                        </math:Binop>
                    </math:Equation>
                </ct:Assign>
            </SimpleParameter>
            <!-- omitted parameters k21, k13, k31 & V3 -->
            
        </ParameterModel>

        <StructuralModel blkId="sm1">
            <ct:Variable symbolType="real" symbId="Ac"/>
            <ct:Variable symbolType="real" symbId="Ap1"/>
            <ct:Variable symbolType="real" symbId="Ap2"/>
            <PKmacros>
                    <!-- omitted, identical to ADVAN11, TRANS1 -->
            </PKmacros>
        </StructuralModel>
\end{lstlisting}

\paragraph{Version 2}
PharmML code -- with reparameterization within macros:
\lstset{language=XML}
\begin{lstlisting}
        <ParameterModel blkId="pm1">
            
            <!-- Q2 -->
            <IndividualParameter symbId="Q2">
                <!-- omitted details -->
            </IndividualParameter>
            
            <!-- V1 -->
            <IndividualParameter symbId="V1">
                <!-- omitted details -->
            </IndividualParameter>
            
            <!-- V2 -->
            <IndividualParameter symbId="V2">
                <!-- omitted details -->
            </IndividualParameter>
            
            <!-- omitted other parameters Q3, V3, CL -->
            
        </ParameterModel>

        <StructuralModel blkId="sm1">
            <ct:Variable symbolType="real" symbId="Ac"/>
            <ct:Variable symbolType="real" symbId="Ap1"/>
            <ct:Variable symbolType="real" symbId="Ap2"/>
            <PKmacros>
            
                <!-- omitted other macros -->

                <Peripheral>
                    <Value argument="k12">
                        <ct:Assign>
                            <math:Equation>
                                <math:Binop op="divide">
                                    <ct:SymbRef blkIdRef="pm1" symbIdRef="Q2"/>
                                    <ct:SymbRef blkIdRef="pm1" symbIdRef="V1"/>
                                </math:Binop>
                            </math:Equation>
                        </ct:Assign>
                    </Value>
                    <Value argument="k21">
                        <ct:Assign>
                            <math:Equation>
                                <math:Binop op="divide">
                                    <ct:SymbRef blkIdRef="pm1" symbIdRef="Q2"/>
                                    <ct:SymbRef blkIdRef="pm1" symbIdRef="V2"/>
                                </math:Binop>
                            </math:Equation>
                        </ct:Assign>
                    </Value>
                    <Value argument="amount">
                        <ct:SymbRef symbIdRef="Ap1"/>
                    </Value>
                </Peripheral>

                <!-- omitted other macros -->
            </PKmacros>
        </StructuralModel>
\end{lstlisting}


%\begin{figure}[htb!]
%\centering
%  \includegraphics[width=140mm]{pics/Pavia2}
% \caption{PK model macros discussion during the PharmML workshop in Pavia/Italy (November 2013).}
% \label{Pavia2}
%\end{figure}


\cleardoublepage
%%%%%%%%%%%%%%%%%%%%%%%%%%%%%%%%%%%%%%%%%%%%%%%%%%%%%%%
\subsection{Complex PK macro examples}
Additional examples have been encoded in PharmML in order to further validate the schema
and to test the implementation of more complex PK macros.

\textbf{Note} that the first two of the following models, C1 (versions C1a and C1b) and C2, 
can be treated as equivalent to that discussed in Sections \ref{subsubsec:advan2}, 
\ref{subsubsec:ADVAN12}, and modelled by the ADVAN2 ands ADVAN12 routines, 
respectively, with an additional IV administration.

Importantly, \marginpar{\HandCuffLeft} the according dataset transformation reduces 
to the renaming of the ADM and CMT columns \textbf{but} only if the oral administration 
is treated as the first one, i.e. with attribute \xatt{adm=1}. Otherwise a renumbering 
is required as well. Examples C1a and C1b illustrate this.

%\cleardoublepage
%%%%%%%%%%%%%%%%%%%%%%%%%%%%%%%%%%%%%%%%%%%%%%%%%%%%%%%
\subsubsection{Example C1a}
%The following example is from \cite{Monolix4.3Tutorial:2014}, page 70. Here the original description
%(with referrences to the MONOLIX dataset format suing \xatt{adm} instead of \xatt{cmt}
%column, see below):
%"An additional column \xatt{adm} is used to specify the different types of administration.
%Here, \xatt{adm=1} indicates an oral administration and \xatt{adm=2} an IV bolus administration.
%The datafile only contains information about the administration, not about the absorption 
%process which will be described in the model."
The following example is from \cite{Monolix4.3Tutorial:2014}, pages 70-74. This model
can be modelled by the ADVAN2 routine, with an additional IV administration. Because the 
oral administration is labeled as the first one, attribute \xatt{adm=1}, no renumbering is required 
when transforming the data and renaming the ADM to CMT column is sufficient to get the suitable 
NONMEM data set.

\begin{figure}[htbp!]
\centering
 \includegraphics[width=120mm]{pics/ComplexModel1}
\caption{C1a model, with compartment numbering dependent on the target tool.}
\label{fig:ComplexModel1a}
\end{figure}

\begin{table}[ht!]
\footnotesize
\parbox{.5\linewidth}{
\centering
\begin{tabular}{ccccc}
  \hline
     \multicolumn{5}{c}{\textbf{MONOLIX}} \\
  \hline
ID	& TIME  & AMT	 & \textbf{ADM} &	Y \\
  \hline
1	& 0	    & 2.24	& \textbf{2}	& . \\
1	& 1	    & .	& .		 	& 142 \\
1	& 2	    & .	& .	 		& 54.9 \\
...     &  ...     &  ...     & ...  			& ...\\
1	& 6	    & 7	& \textbf{1}	& . \\
1	& 7	    & .	& .			& 192 \\
1	& 8	    & .	& .			& 141 \\
...     &  ...     &  ...     & ...  			& ...\\
2	& 0	    & 2.73	& \textbf{2}	& . \\
2	& 1	    & . 	& .			& 176 \\
...     &  ...     &  ...     & ...  			& ...\\
\end{tabular}
}
\hfill
\parbox{.5\linewidth}{
\centering
\begin{tabular}{ccccc}
  \hline
   \multicolumn{5}{c}{\textbf{NONMEM}} \\
  \hline
ID	& TIME  & AMT	 & \textbf{\textcolor{red}{CMT}} &	DV \\
  \hline
1	& 0	    & 2.24	& \textbf{\textcolor{red}{2}}	& . \\
1	& 1	    & .	& 2		 	& 142 \\
1	& 2	    & .	& 2	 		& 54.9 \\
...     &  ...     &  ...     & ...  			& ...\\
1	& 6	    & 7	& \textbf{\textcolor{red}{1}}	& . \\
1	& 7	    & .	& 2			& 192 \\
1	& 8	    & .	& 2			& 141 \\
...     &  ...     &  ...     & ...  			& ...\\
2	& 0	    & 2.73	& \textbf{\textcolor{red}{2}}	& . \\
2	& 1	    & . 	& 2	 		& 176 \\
...     &  ...     &  ...     & ...  			& ...\\
\end{tabular}
}
\caption{NONMEM and MONOLIX datasets for the C1a model. Because of the equivalence to 
ADVAN2 no renumbering is required when transforming the data and renaming the ADM 
to CMT column.}
\label{tab:C1aTable}
\end{table}

\begin{table}[h!]
\setlength{\tabcolsep}{15pt}
\begin{center}
\begin{tabular}{c}
%\begin{tabular*}{.95\textwidth}{@{\extracolsep{\fill} } ll}
  \hline \hline
PK macro  \\[-.25ex]
  \hline
\lstset{language=NONMEMdataSet}
\begin{lstlisting}
input = {F, ka, V, k}
PK:
compartment(cmt=1, amount=Ac)
oral(adm=1, cmt=1, ka, p=F)
iv(adm=2, cmt=1)
elimination(cmt=1, k)
Cc = Ac/V
\end{lstlisting}
%&
%\lstset{language=NONMEMdataSet}
%\begin{lstlisting}
%input = {F, ka, V, k}
%PK:
%compartment(cmt=1, amount=Ac)
%[*compartment(cmt=2, amount=Depot)*]
%oral(adm=1, [*fromCmt=2,*] cmt=1, ka, p=F)
%iv(adm=2, cmt=1)
%elimination(cmt=1, k)
%Cc = Ac/V
%\end{lstlisting} 
\\
  \hline
\end{tabular}
\caption{PK macros for the C1a model, as shown in Figure \ref{fig:ComplexModel1a} (left).}
\label{tab:C1aMacrosTable}
\end{center}
\end{table}

PharmML code:
\lstset{language=XML}
\begin{lstlisting}
        <StructuralModel blkId="smC1a">
            <ct:Variable symbolType="real" symbId="Ap"/>
            <ct:Variable symbolType="real" symbId="Cc">
                <ct:Assign>
                    <math:Equation>
                        <math:Binop op="divide">
                            <ct:SymbRef symbIdRef="Ac"/>
                            <ct:SymbRef blkIdRef="pm1" symbIdRef="V"/>
                        </math:Binop>
                    </math:Equation>
                </ct:Assign>
            </ct:Variable>
            
            <PKmacros>
                <!-- compartment(cmt=1, amount=Ac) -->
                <Compartment>
                    <Value argument="cmt">
                        <ct:Int>1</ct:Int>
                    </Value>
                    <Value argument="amount">
                        <ct:SymbRef symbIdRef="Ac"/>
                    </Value>
                </Compartment>
                
                <!-- oral(adm=1, cmt=1, ka, p=F) -->
                <Oral>
                    <Value argument="adm">
                        <ct:Int>1</ct:Int>
                    </Value>
                    <Value argument="cmt">
                        <ct:Int>1</ct:Int>
                    </Value>
                    <Value>
                        <ct:SymbRef blkIdRef="pm1" symbIdRef="ka"/>
                    </Value>
                    <Value argument="p">
                        <ct:SymbRef blkIdRef="pm1" symbIdRef="F"/>
                    </Value>
                </Oral>
                
                <!-- iv(adm=2, cmt=1) -->
                <IV>
                    <Value argument="adm">
                        <ct:Int>2</ct:Int>
                    </Value>
                    <Value argument="cmt">
                        <ct:Int>1</ct:Int>
                    </Value>
                </IV>
                
                <!-- elimination(cmt=1, k) -->
                <Elimination>
                    <Value argument="cmt">
                        <ct:Int>1</ct:Int>
                    </Value>
                    <Value>
                        <ct:SymbRef blkIdRef="pm1" symbIdRef="k"/>
                    </Value>
                </Elimination>
            </PKmacros>
        </StructuralModel>
\end{lstlisting}


%\cleardoublepage
%%%%%%%%%%%%%%%%%%%%%%%%%%%%%%%%%%%%%%%%%%%%%%%%%%%%%%%
\subsubsection{Example C1b}
This example is very similar to the previous one with the difference in the administration order.
IV is labeled as the first one, with attribute \xatt{adm=1}, the oral as the second, \xatt{adm=2}, 
Figure \ref{fig:ComplexModel1b}. The model can still be modelled by the ADVAN2 routine, 
with an additional IV administration, but the dataset \marginpar{\HandCuffLeft} translation 
will not be limited to the renaming of column names only as in C1a case. Now the numbers in 
columns ADM and CMT will not be identical, see Table \ref{tab:C1bTable}.

\begin{figure}[htbp!] 
\centering
 \includegraphics[width=130mm]{pics/ComplexModel1b}
\caption{C1b model, with compartment numbering dependent on the target tool.}
\label{fig:ComplexModel1b}
\end{figure}

\begin{table}[ht!]
\footnotesize
\parbox{.5\linewidth}{
\centering
\begin{tabular}{ccccc}
  \hline
     \multicolumn{5}{c}{\textbf{MONOLIX}} \\
  \hline
ID	& TIME  & AMT	 & \textbf{ADM} &	Y \\
  \hline
1	& 0	    & 2.24	& \textbf{1}	& . \\
1	& 1	    & .	& .		 	& 142 \\
1	& 2	    & .	& .	 		& 54.9 \\
...     &  ...     &  ...     & ...  			& ...\\
1	& 6	    & 7	& \textbf{2}	& . \\
1	& 7	    & .	& .			& 192 \\
1	& 8	    & .	& .			& 141 \\
...     &  ...     &  ...     & ...  			& ...\\
2	& 0	    & 2.73	& \textbf{1}	& . \\
2	& 1	    & . 	& .			& 176 \\
...     &  ...     &  ...     & ...  			& ...\\
\end{tabular}
}
\hfill
\parbox{.5\linewidth}{
\centering
\begin{tabular}{ccccc}
  \hline
   \multicolumn{5}{c}{\textbf{NONMEM}} \\
  \hline
ID	& TIME  & AMT	 & \textbf{\textcolor{red}{CMT}} &	DV \\
  \hline
1	& 0	    & 2.24	& \textbf{\textcolor{red}{2}}	& . \\
1	& 1	    & .	& 2		 	& 142 \\
1	& 2	    & .	& 2	 		& 54.9 \\
...     &  ...     &  ...     & ...  			& ...\\
1	& 6	    & 7	& \textbf{\textcolor{red}{1}}	& . \\
1	& 7	    & .	& 2			& 192 \\
1	& 8	    & .	& 2			& 141 \\
...     &  ...     &  ...     & ...  			& ...\\
2	& 0	    & 2.73	& \textbf{\textcolor{red}{2}}	& . \\
2	& 1	    & . 	& 2	 		& 176 \\
...     &  ...     &  ...     & ...  			& ...\\
\end{tabular}
}
\caption{NONMEM and MONOLIX datasets for the C1b model. In contrast to the previous case, a 
renumbering is required when transforming the data and renaming the ADM to CMT column.}
\label{tab:C1bTable}
\end{table}

\begin{table}[h!]
\setlength{\tabcolsep}{15pt}
\begin{center}
\begin{tabular}{c}
%\begin{tabular*}{.95\textwidth}{@{\extracolsep{\fill} } ll}
  \hline \hline
PK macro  \\[-.25ex]
  \hline
\lstset{language=NONMEMdataSet}
\begin{lstlisting}
input = {F, ka, V, k}
PK:
compartment(cmt=1, amount=Ac)
oral(adm=[*2*], cmt=1, ka, p=F)
iv(adm=[*1*], cmt=1)
elimination(cmt=1, k)
Cc = Ac/V
\end{lstlisting}
%&
%\lstset{language=NONMEMdataSet}
%\begin{lstlisting}
%input = {F, ka, V, k}
%PK:
%compartment(cmt=1, amount=Ac)
%[*compartment(cmt=2, amount=Depot)*]
%oral(adm=1, [*fromCmt=2,*] cmt=1, ka, p=F)
%iv(adm=2, cmt=1)
%elimination(cmt=1, k)
%Cc = Ac/V
%\end{lstlisting} 
\\
  \hline
\end{tabular}
\caption{PK macros  for the C1b model, as shown in Figure \ref{fig:ComplexModel1b} (left).}
\label{tab:C1bMacrosTable}
\end{center}
\end{table}


PharmML code is identical to the previous example, and will not be shown, except the 
amended \xatt{adm} attribute assignments, \xatt{adm=\textbf{\textcolor{red}{2}}} for oral 
and \xatt{adm=\textbf{\textcolor{red}{1}}} for IV administration, see Table \ref{tab:C1bMacrosTable}.
%\lstset{language=XML}
%\begin{lstlisting}
%                <Oral>
%                    <Value argument="adm">
%                        <ct:Int>2</ct:Int>
%                    </Value>
%                    <!-- omitted details -->
%                </Oral>
%                
%                <IV>
%                    <Value argument="adm">
%                        <ct:Int>1</ct:Int>
%                    </Value>
%                    <!-- omitted details -->
%                </IV>
%\end{lstlisting}


%\cleardoublepage
%%%%%%%%%%%%%%%%%%%%%%%%%%%%%%%%%%%%%%%%%%%%%%%%%%%%%%%
\subsubsection{Example C2}
Modified from \cite{MLXTRANforMonolix:2014}. This model can be modelled by the 
ADVAN12 routine, with an additional IV administration.

\begin{figure}[htbp]
\centering
 \includegraphics[width=160mm]{pics/ComplexModel2}
\caption{C2 model, with compartment numbering dependent on the target tool.}
\label{fig:ComplexModel2}
\end{figure}


\begin{table}[h!]
\footnotesize
\parbox{.5\linewidth}{
\centering
\begin{tabular}{cccccc}
  \hline
   \multicolumn{6}{c}{\textbf{MONOLIX}} \\
  \hline
ID	& TIME  & AMT	 & \textbf{ADM} &  TINF &	Y \\
  \hline
1	& 6	    & 10	& \textbf{1}	 & 2	 & . \\
1	& 9	    & 20	& \textbf{2}	 & .	 & . \\
1	& 18	    & 10	& \textbf{1}	 & 2	 & . \\
1	& 30	    & 10	& \textbf{1}	 & 2	 & . \\
1	& 33	    & 20	& \textbf{2}	 & .	 & . \\
...     &  ...     &  ...     &  ...  & ...  & ... \\
1	& 0	    & .	& .	& .	& 0 \\
1	& 6	    & .	& .	& .	& 0 \\
1	& 12	    & .	& .	& .	& 1.18 \\
...     &  ...     &  ...     & ...  & ...  & ...\\
\end{tabular}
}
\hfill
\parbox{.5\linewidth}{
\centering
\begin{tabular}{cccccc}
  \hline
   \multicolumn{6}{c}{\textbf{NONMEM}} \\
  \hline
ID	& TIME  & AMT	 & \textbf{\textcolor{red}{CMT}} &  RATE &	DV \\
  \hline
1	& 6	    & 10	& \textbf{\textcolor{red}{1}}	 & 5	 & . \\
1	& 9	    & 20	& \textbf{\textcolor{red}{2}}	 & .	 & . \\
1	& 18	    & 10	& \textbf{\textcolor{red}{1}}	 & 5	 & . \\
1	& 30	    & 10	& \textbf{\textcolor{red}{1}}	 & 5	 & . \\
1	& 33	    & 20	& \textbf{\textcolor{red}{2}}	 & .	 & . \\
...     &  ...     &  ...     &  ...  & ...  & ... \\
1	& 0	    & .	& 1	& .	& 0 \\
1	& 6	    & .	& 1	& .	& 0 \\
1	& 12	    & .	& 1	& .	& 1.18 \\
...     &  ...     &  ...     & ...  & ...  & ...\\
\end{tabular}
}
\caption{NONMEM and MONOLIX datasets for the C2 model.}
\end{table}


\begin{table}[ht!]
\setlength{\tabcolsep}{1pt}
\begin{center}
\begin{tabular}{c}
%\begin{tabular*}{1.05\textwidth}{@{\extracolsep{\fill} } ll}
  \hline \hline
PK macro  \\[-.25ex]
  \hline
\lstset{language=NONMEMdataSet}
\begin{lstlisting}
input = {V, alpha, beta} 
PK:
compartment(cmt=1, volume=V, concentration=Cc)
iv(adm=1, cmt=1, p=0.1, Tlag=t/(t + 10))
oral(adm=2, cmt=1, Tk0=0.1)
elimination(cmt=1, k=0.2)
peripheral(k12=0.6, k21=0.8, amount=Ap2)
peripheral(k13=0.6+alpha, k31=0.8+beta, amount=Ap3)
\end{lstlisting}
%&
%\lstset{language=NONMEMdataSet}
%\begin{lstlisting}
%compartment(cmt=1, volume=V, concentration=Cc)
%[*compartment(cmt=4, amount=Depot1)*]
%iv(adm=1, cmt=1, p=0.1, Tlag=t/(t + 10))
%oral(adm=2, [*fromCmt=4,*] cmt=1, Tk0=0.1)
%elimination(cmt=1, k=0.2)
%peripheral(k12=0.6, k21=0.8, amount=Ap2)
%peripheral(k13=0.6+alpha, k31=0.8+beta, amount=Ap3)
%\end{lstlisting} 
\\
  \hline
\end{tabular}
\caption{PK macros  for the C2 model, as shown in Figure \ref{fig:ComplexModel2} (left).}
\label{tab:C3Table}
\end{center}
\end{table}


PharmML code:
\lstset{language=XML}
\begin{lstlisting}
        <StructuralModel blkId="smC2">
            <ct:Variable symbolType="real" symbId="Ap2"/>
            <ct:Variable symbolType="real" symbId="Ap3"/>
            <ct:Variable symbolType="real" symbId="Cc"/>
            
            <PKmacros>
                <Compartment>
                    <Value argument="cmt">
                        <ct:Int>1</ct:Int>
                    </Value>
                    <Value argument="volume">
                        <ct:SymbRef blkIdRef="pm1" symbIdRef="V"/>
                    </Value>
                    <Value argument="concentration">
                        <ct:SymbRef symbIdRef="Cc"/>
                    </Value>
                </Compartment>
                
                <IV>
                    <Value argument="adm">
                        <ct:Int>1</ct:Int>
                    </Value>
                    <Value argument="cmt">
                        <ct:Int>1</ct:Int>
                    </Value>
                    <Value argument="p">
                        <ct:Real>0.1</ct:Real>
                    </Value>
                    <Value argument="Tlag">
                        <ct:Assign>
                            <math:Equation>
                                <math:Binop op="divide">
                                    <ct:SymbRef symbIdRef="t"/>
                                    <math:Binop op="plus">
                                        <ct:SymbRef symbIdRef="t"/>
                                        <ct:Real>10</ct:Real>
                                    </math:Binop>
                                </math:Binop>
                            </math:Equation>
                        </ct:Assign>
                    </Value>
                </IV>
                
                <Oral>
                    <Value argument="adm">
                        <ct:Int>2</ct:Int>
                    </Value>
                    <Value argument="cmt">
                        <ct:Int>1</ct:Int>
                    </Value>
                    <Value argument="Tk0">
                        <ct:Real>0.1</ct:Real>
                    </Value>
                </Oral>
                
                <Elimination>
                    <Value argument="cmt">
                        <ct:Int>1</ct:Int>
                    </Value>
                    <Value argument="k">
                        <ct:Real>0.2</ct:Real>
                    </Value>
                </Elimination>
                
                <Peripheral>
                    <Value argument="k12">
                        <ct:Real>0.6</ct:Real>
                    </Value>
                    <Value argument="k21">
                        <ct:Real>0.8</ct:Real>
                    </Value>
                    <Value argument="amount">
                        <ct:SymbRef symbIdRef="Ap2"/>
                    </Value>
                </Peripheral>
                
                <Peripheral>
                    <Value argument="k13">
                        <ct:Assign>
                            <math:Equation>
                                <math:Binop op="plus">
                                    <ct:Real>0.6</ct:Real>
                                    <ct:SymbRef blkIdRef="pm1" symbIdRef="alpha"/>
                                </math:Binop>
                            </math:Equation>
                        </ct:Assign>
                    </Value>
                    <Value argument="k31">
                        <ct:Assign>
                            <math:Equation>
                                <math:Binop op="plus">
                                    <ct:Real>0.8</ct:Real>
                                    <ct:SymbRef blkIdRef="pm1" symbIdRef="beta"/>
                                </math:Binop>
                            </math:Equation>
                        </ct:Assign>
                    </Value>
                    <Value argument="amount">
                        <ct:SymbRef symbIdRef="Ap3"/>
                    </Value>
                </Peripheral>
                
            </PKmacros>
        </StructuralModel>
\end{lstlisting}



%\cleardoublepage
%%%%%%%%%%%%%%%%%%%%%%%%%%%%%%%%%%%%%%%%%%%%%%%%%%%%%%%
\subsubsection{Example C3}
From \cite{MLXTRANforMonolix:2014}. Figure \ref{fig:ComplexModel3} is illustrating this model in two versions. 

\begin{figure}[htbp]
\centering
 \includegraphics[width=160mm]{pics/ComplexModel3}
\caption{C3 model, with compartment numbering dependent on the target tool.}
\label{fig:ComplexModel3}
\end{figure}

\begin{table}[h!]
\footnotesize
\parbox{.5\linewidth}{
\centering
\begin{tabular}{cccccc}
  \hline
   \multicolumn{6}{c}{\textbf{MONOLIX}} \\
  \hline
ID	& TIME  & AMT	 & \textbf{ADM} &  TINF &	Y \\
  \hline
1	& 6	    & 10	& \textbf{1}	 & .	 & . \\
1	& 9	    & 20	& \textbf{2}	 & .	 & . \\
1	& 12	    & 30	& \textbf{3}	 & 2	 & . \\
1	& 18	    & 10	& \textbf{1}	 & .	 & . \\
1	& 30	    & 10	& \textbf{1}	 & .	 & . \\
1	& 33	    & 20	& \textbf{2}	 & .	 & . \\
1	& 36	    & 30	& \textbf{3}	 & 2	 & . \\
...     &  ...     &  ...     &  ...  & ...  & ... \\
1	& 0	    & .	& .	& .	& 0 \\
1	& 6	    & .	& .	& .	& 0 \\
1	& 12	    & .	& .	& .	& 1.18 \\
...     &  ...     &  ...     & ...  & ...  & ...\\
\end{tabular}
}
\hfill
\parbox{.5\linewidth}{
\centering
\begin{tabular}{cccccc}
  \hline
   \multicolumn{6}{c}{\textbf{NONMEM}} \\
  \hline
ID	& TIME  & AMT	 & \textbf{\textcolor{red}{CMT}} &  RATE &	DV \\
  \hline
1	& 6	    & 10	& \textbf{\textcolor{red}{3}}	 & .	 & . \\
1	& 9	    & 20	& \textbf{\textcolor{red}{4}}	 & .	 & . \\
1	& 12	    & 30	& \textbf{\textcolor{red}{2}}	 & \textbf{15}	 & . \\
1	& 18	    & 10	& \textbf{\textcolor{red}{3}}	 & .	 & . \\
1	& 30	    & 10	& \textbf{\textcolor{red}{3}}	 & .	 & . \\
1	& 33	    & 20	& \textbf{\textcolor{red}{4}}	 & .	 & . \\
1	& 36	    & 30	& \textbf{\textcolor{red}{2}}	 & \textbf{15}	 & . \\
...     &  ...     &  ...     &  ...  & ...  & ... \\
1	& 0	    & .	& 1	& .	& 0 \\
1	& 6	    & .	& 1	& .	& 0 \\
1	& 12	    & .	& 1	& .	& 1.18 \\
...     &  ...     &  ...     & ...  & ...  & ...\\
\end{tabular}
}
\caption{NONMEM and MONOLIX datasets for the C3 model.}
\end{table}

\begin{table}[ht!]
\setlength{\tabcolsep}{15pt}
\begin{center}
\begin{tabular}{c}
%\begin{tabular*}{.95\textwidth}{@{\extracolsep{\fill} } ll}
  \hline \hline
PK macro  \\[-.25ex]
  \hline
\lstset{language=NONMEMdataSet}
\begin{lstlisting}
input = {F1, F2, ka1, ka2, k1, k2, kl, V}
PK:
compartment(cmt=1, amount=Al)
compartment(cmt=2, amount=Ac, volume=V)
oral(adm=1, cmt=1, ka1, p=F1)
oral(adm=2, cmt=2, ka2, p=F2)
iv(adm=3, cmt=2)
transfer(from=1, to=2, kt=kl)
elimination(cmt=1, k1)
elimination(cmt=2, k2)
\end{lstlisting}
%&
%\lstset{language=NONMEMdataSet}
%\begin{lstlisting}
%input = {F1, F2, ka1, ka2, k1, k2, kl, V}
%PK:
%compartment(cmt=1, amount=Al)                                                   
%compartment(cmt=2, amount=Ac)
%[*compartment(cmt=3, amount=Depot1)*]
%[*compartment(cmt=4, amount=Depot2)*]
%oral(adm=1, [*fromCmt=3,*] cmt=1, ka1, p=F1)
%oral(adm=2, [*fromCmt=4,*] cmt=2, ka2, p=F2)
%iv(adm=3, cmt=2)
%transfer(from=1, to=2, kt=kl)
%elimination(cmt=1, k1)
%elimination(cmt=2, k2)
%Cc = Ac/V
%\end{lstlisting} 
\\
  \hline
\end{tabular}
\caption{PK macros  for the C3 model, as shown in Figure \ref{fig:ComplexModel3} (left).}
\label{tab:C2Table}
\end{center}
\end{table}


PharmML code:
\lstset{language=XML}
\begin{lstlisting}
        <StructuralModel blkId="smC3">
            <ct:Variable symbolType="real" symbId="Al"/>
            <ct:Variable symbolType="real" symbId="Ac"/>
            <ct:Variable symbolType="real" symbId="Cc"/>
            
            <PKmacros>
                <Compartment>
                    <Value argument="cmt">
                        <ct:Int>1</ct:Int>
                    </Value>
                    <Value argument="amount">
                        <ct:SymbRef symbIdRef="Al"/>
                    </Value>
                </Compartment>
                
                <Compartment>
                    <Value argument="cmt">
                        <ct:Int>2</ct:Int>
                    </Value>
                    <Value argument="amount">
                        <ct:SymbRef symbIdRef="Ac"/>
                    </Value>
                    <Value argument="volume">
                        <ct:SymbRef blkIdRef="pm1" symbIdRef="V"/>
                    </Value>
                </Compartment>
                
                <Oral>
                    <Value argument="adm">
                        <ct:Int>1</ct:Int>
                    </Value>
                    <Value argument="cmt">
                        <ct:Int>1</ct:Int>
                    </Value>
                    <Value>
                        <ct:SymbRef blkIdRef="pm1" symbIdRef="ka1"/>
                    </Value>
                    <Value argument="p">
                        <ct:SymbRef blkIdRef="pm1" symbIdRef="F1"/>
                    </Value>
                </Oral>
                
                <Oral>
                    <Value argument="adm">
                        <ct:Int>2</ct:Int>
                    </Value>
                    <Value argument="cmt">
                        <ct:Int>2</ct:Int>
                    </Value>
                    <Value>
                        <ct:SymbRef blkIdRef="pm1" symbIdRef="ka2"/>
                    </Value>
                    <Value argument="p">
                        <ct:SymbRef blkIdRef="pm1" symbIdRef="F2"/>
                    </Value>
                </Oral>
                
                <IV>
                    <Value argument="adm">
                        <ct:Int>3</ct:Int>
                    </Value>
                    <Value argument="cmt">
                        <ct:Int>2</ct:Int>
                    </Value>
                </IV>
                
                <Transfer>
                    <Value argument="from">
                        <ct:Int>1</ct:Int>
                    </Value>
                    <Value argument="to">
                        <ct:Int>2</ct:Int>
                    </Value>
                    <Value argument="kt">
                        <ct:SymbRef blkIdRef="pm1" symbIdRef="kl"/>
                    </Value>
                </Transfer>
                
                <Elimination>
                    <Value argument="cmt">
                        <ct:Int>1</ct:Int>
                    </Value>
                    <Value>
                        <ct:SymbRef blkIdRef="pm1" symbIdRef="k1"/>
                    </Value>
                </Elimination>
                
                <Elimination>
                    <Value argument="cmt">
                        <ct:Int>1</ct:Int>
                    </Value>
                    <Value>
                        <ct:SymbRef blkIdRef="pm1" symbIdRef="k2"/>
                    </Value>
                </Elimination>
                
            </PKmacros>
        </StructuralModel>
\end{lstlisting}


%\cleardoublepage
%%%%%%%%%%%%%%%%%%%%%%%%%%%%%%%%%%%%%%%%%%%%%%%%%%%%%%%
\subsubsection{Example C4}
The following example is from 'Four models' document \cite{LavielleFourModels:2014}.
%Here the original description:
%"In this example, one type of dose is administered orally (adm=1) and absorbed into a latent 
%compartment following a first-order absorption process, a second type is administered o
%rally (adm=2) and absorbed into the central compartment following a zero-order absorption process,
%a third type is directly administered intravenously to the central compartment (adm=3).
%The transfer from the latent to the central compartment is linear. A peripheral compartment
%is linked to the central compartment. The drug is eliminated by a linear process from the
%latent compartment and a nonlinear process from the central one. Here, Al and Ac are the
%drug amounts in the latent and central compartments." 
Figure \ref{fig:ComplexModel4} is illustrating this model in two versions. 

\begin{figure}[htbp!]
\centering
 \includegraphics[width=160mm]{pics/ComplexModel4}
\caption{C4 model, with compartment numbering dependent on the target tool.}
\label{fig:ComplexModel4}
\end{figure}

\begin{table}[h!]
\footnotesize
\parbox{.5\linewidth}{
\centering
\begin{tabular}{cccccc}
  \hline
   \multicolumn{6}{c}{\textbf{MONOLIX}} \\
  \hline
ID	& TIME  & AMT	 & \textbf{ADM} &  TINF &	Y \\
  \hline
1	& 6	    & 10	& \textbf{1}	 & .	 & . \\
1	& 9	    & 20	& \textbf{2}	 & .	 & . \\
1	& 12	    & 30	& \textbf{3}	 & 2	 & . \\
1	& 18	    & 10	& \textbf{1}	 & .	 & . \\
1	& 30	    & 10	& \textbf{1}	 & .	 & . \\
1	& 33	    & 20	& \textbf{2}	 & .	 & . \\
1	& 36	    & 30	& \textbf{3}	 & 2	 & . \\
...     &  ...     &  ...     &  ...  & ...  & ... \\
1	& 0	    & .	& .	& .	& 0 \\
1	& 6	    & .	& .	& .	& 0 \\
1	& 12	    & .	& .	& .	& 1.18 \\
...     &  ...     &  ...     & ...  & ...  & ...\\
\end{tabular}
}
\hfill
\parbox{.5\linewidth}{
\centering
\begin{tabular}{cccccc}
  \hline
   \multicolumn{6}{c}{\textbf{NONMEM}} \\
  \hline
ID	& TIME  & AMT	 & \textbf{\textcolor{red}{CMT}} &  RATE &	DV \\
  \hline
1	& 6	    & 10	& \textbf{\textcolor{red}{4}}	 & .	 & . \\
1	& 9	    & 20	& \textbf{\textcolor{red}{5}}	 & .	 & . \\
1	& 12	    & 30	& \textbf{\textcolor{red}{2}}	 & \textbf{15}	 & . \\
1	& 18	    & 10	& \textbf{\textcolor{red}{4}}	 & .	 & . \\
1	& 30	    & 10	& \textbf{\textcolor{red}{4}}	 & .	 & . \\
1	& 33	    & 20	& \textbf{\textcolor{red}{5}}	 & .	 & . \\
1	& 36	    & 30	& \textbf{\textcolor{red}{2}}	 & \textbf{15}	 & . \\
...     &  ...     &  ...     &  ...  & ...  & ... \\
1	& 0	    & .	& 1	& .	& 0 \\
1	& 6	    & .	& 1	& .	& 0 \\
1	& 12	    & .	& 1	& .	& 1.18 \\
...     &  ...     &  ...     & ...  & ...  & ...\\
\end{tabular}
}
\caption{NONMEM and MONOLIX datasets for the C4 model.}
\end{table}

\begin{table}[ht!]
\setlength{\tabcolsep}{15pt}
\begin{center}
\begin{tabular}{c}
%\begin{tabular*}{.95\textwidth}{@{\extracolsep{\fill} } ll}
  \hline \hline
PK macro  \\[-.25ex]
  \hline
\lstset{language=NONMEMdataSet}
\begin{lstlisting}
input = {F1, F2, ka, Tk0, kl, V, k, Vm, Km}
PK:
compartment(cmt=1, amount=Al)
compartment(cmt=2, amount=Ac)
oral(adm=1, cmt=1, ka, p=F1)
oral(adm=2, cmt=2, Tk0, p=F2)
iv(adm=3, cmt=2)
transfer(from=1, to=2, kt=kl)
peripheral(k23, k32, amount=Ap)
elimination(cmt=1, k)
elimination(cmt=2, Km, Vm)
\end{lstlisting}
\\
  \hline
\end{tabular}
\caption{PK macros  for the C4 model, as shown in Figure \ref{fig:ComplexModel4} (left).}
\label{tab:C4Table}
\end{center}
\end{table}


PharmML code:
\lstset{language=XML}
\begin{lstlisting}
        <StructuralModel blkId="smC4">
            <ct:Variable symbolType="real" symbId="Al"/>
            <ct:Variable symbolType="real" symbId="Ac"/>
            <ct:Variable symbolType="real" symbId="Cc">
                <ct:Assign>
                    <math:Equation>
                        <math:Binop op="divide">
                            <ct:SymbRef symbIdRef="Ac"/>
                            <ct:SymbRef blkIdRef="pm1" symbIdRef="V"/>
                        </math:Binop>
                    </math:Equation>
                </ct:Assign>
            </ct:Variable>
            
            <PKmacros>
                <Compartment>
                    <Value argument="cmt">
                        <ct:Int>1</ct:Int>
                    </Value>
                    <Value argument="amount">
                        <ct:SymbRef symbIdRef="Al"/>
                    </Value>
                </Compartment>
                
                <Compartment>
                    <Value argument="cmt">
                        <ct:Int>2</ct:Int>
                    </Value>
                    <Value argument="amount">
                        <ct:SymbRef symbIdRef="Ac"/>
                    </Value>
                </Compartment>
                
                <Oral>
                    <Value argument="adm">
                        <ct:Int>1</ct:Int>
                    </Value>
                    <Value argument="cmt">
                        <ct:Int>1</ct:Int>
                    </Value>
                    <Value>
                        <ct:SymbRef blkIdRef="pm1" symbIdRef="ka"/>
                    </Value>
                    <Value argument="p">
                        <ct:SymbRef blkIdRef="pm1" symbIdRef="F1"/>
                    </Value>
                </Oral>
                
                <Oral>
                    <Value argument="adm">
                        <ct:Int>2</ct:Int>
                    </Value>
                    <Value argument="cmt">
                        <ct:Int>2</ct:Int>
                    </Value>
                    <Value>
                        <ct:SymbRef blkIdRef="pm1" symbIdRef="Tk0"/>
                    </Value>
                    <Value argument="p">
                        <ct:SymbRef blkIdRef="pm1" symbIdRef="F2"/>
                    </Value>
                </Oral>
                
                <IV>
                    <Value argument="type">
                        <ct:Int>3</ct:Int>
                    </Value>
                    <Value argument="cmt">
                        <ct:Int>2</ct:Int>
                    </Value>
                </IV>
                
                <Transfer>
                    <Value argument="from">
                        <ct:Int>1</ct:Int>
                    </Value>
                    <Value argument="to">
                        <ct:Int>2</ct:Int>
                    </Value>
                    <Value argument="kt">
                        <ct:SymbRef blkIdRef="pm1" symbIdRef="kl"/>
                    </Value>
                </Transfer>
                
                <Peripheral>
                    <Value>
                        <ct:SymbRef blkIdRef="pm1" symbIdRef="k23"/>
                    </Value>
                    <Value>
                        <ct:SymbRef blkIdRef="pm1" symbIdRef="k32"/>
                    </Value>
                    <Value argument="amount">
                        <ct:SymbRef symbIdRef="Ap"/>
                    </Value>
                </Peripheral>
                
                <Elimination>
                    <Value argument="cmt">
                        <ct:Int>1</ct:Int>
                    </Value>
                    <Value>
                        <ct:SymbRef blkIdRef="pm1" symbIdRef="k"/>
                    </Value>
                </Elimination>
                
                <Elimination>
                    <Value argument="cmt">
                        <ct:Int>2</ct:Int>
                    </Value>
                    <Value>
                        <ct:SymbRef blkIdRef="pm1" symbIdRef="Vm"/>
                    </Value>
                    <Value>
                        <ct:SymbRef blkIdRef="pm1" symbIdRef="Km"/>
                    </Value>
                </Elimination>
            </PKmacros>
        </StructuralModel>
\end{lstlisting}




%%%%%%%%%%%%%%%%%%%%%%%%%%%%%%%%%%%%%%%%%%%%%%%%%%%%%%%%%%%%%%%%%%%%%%
\chapter{Other extensions}
\label{chapter:otherextensions}


%%%%%%%%%%%%%%%%%%%%%%%%%%%%%%%%%%%%%%%%%%%%%%%%%%
\subsection{Conditional column mapping}
Sometimes column mapping is conditional on other columns. Such case was discussed previously
when dealing with mapping of multiple dependent variables, see Section 3.1 in \pml version 0.3.1 
changes document \cite{Swat:2014aa}. Their values where stored in one DV column, but the mapping to 
their targets in the observation model was conditoinal on an additional column ORIG.

A similar situation appears when mapping dosing parameters, D and tD, in algebraic 
PK models such as
\begin{align*}
C_{SS}(t) &= \frac{D}{V}\frac{K_a}{K_a - k} \bigg(\frac{e^{-k (t-t_D)}}{1-e^{-k \tau}}-\frac{e^{-K_a (t-t_D)}}{1-e^{-K_a\tau}}\bigg) 
\end{align*}
and datasets. The dataset has structure shown in the Table \ref{tab:conditionalMapping}.
\begin{table}[ht!]
\small
\begin{center}
\begin{tabular}{cccc}
  \hline
  \hline
ID	& TIME  & AMT	 &	DV \\
  \hline
1	& 0	    & 10	& . \\
1	& 6	    & .	& 5 \\
1	& 9	    & .	& 12 \\
1	& 18	    & .	& 43 \\
...	& ...      & ...	& ... \\
2	& 0	    & 10	& . \\
2	& 6	    & .	& 7 \\
2	& 9	    & .	& 22 \\
2	& 18	    & .	& 54 \\
...	& ...      & ...	& ...
\end{tabular}
\caption{Example of a dataset when conditional mapping is applied.}
\label{tab:conditionalMapping}
\end{center}
\end{table}

Dose amount and time, D and tD, are going to be assigned values from columns AMT
and TIME, respectively, but only when AMT $\neq$ 0. This was not possible so far and 
this missing feature is supported by relaxing the \xelem{ColumnMapping} format by
the addition of the \xelem{Piecewise} element. Few examples
in the previous release, \emph{example2-4.xml}, suffered form this limitation which is now elevated.
Using the \xelem{Piecewise} tag, as in \xelem{MultipleDVMapping}, the mapping can be
encoded as the following code snippet shows
 
\lstset{language=XML}
\begin{lstlisting}
            <!-- map 'amt' column and 'D' -->
            <ColumnMapping>
                <ds:ColumnRef columnIdRef="amt"/>
                <ct:SymbRef blkIdRef="sm1" symbIdRef="D"/>
            </ColumnMapping>
            
            <!-- map 'tD' and 'time' if 'amt' != 0 -->
            <ColumnMapping>
                <ds:ColumnRef columnIdRef="time"/>
                <ds:Piecewise>
                    <math:Piece>
                        <ct:SymbRef blkIdRef="sm1" symbIdRef="tD"/>
                        <math:Condition>
                            <math:LogicBinop op="neq">
                                <ds:ColumnRef columnIdRef="amt"/>
                                <ct:Real>0</ct:Real>
                            </math:LogicBinop>
                        </math:Condition>
                    </math:Piece>
                </ds:Piecewise>
            </ColumnMapping>
\end{lstlisting}
Similarly to this steady-state administration model the same solution can be applied for single dosing 
models formulated algebraically, see \cite{Bertrand:2008} for examples of such models. 
The target mapping for multiple doses models formulated algebraically available also in 
\cite{Bertrand:2008} has not been tested so far.

%%%%%%%%%%%%%%%%%%%%%%%%%%%%%%%%%%%%%%%%%%%%%%%%%%
\subsection{Handling MONOLIX datasets}
\label{subsec:MLXdatasets}
The support for MONOLIX datasets is now required in connection with the PK macros, 
so the new elements \xelem{MONOLIXdataSet} and \xelem{MONOLIXdataSetReference} have 
been introduced which have the same location within the \xelem{ModellingSteps} and features 
as NONMEM related elements. 


%%%%%%%%%%%%%%%%%%%%%%%%%%%%%%%%%%%%%%%%%%%%%%%%%%
\subsection{New attribute \xatt{metadataFile} in the root element}
\label{subsec:metadataFile}
The new attribute \xatt{metadataFile} has been provided in the root element \xelem{PharmML} 
to allow for the link between the RDF annotation file and the according \pml model file. 
The code snippet below shows how the annotation file, \xatt{myModel.rdf}, is referenced from the \pml model. 

\lstset{language=XML}
\begin{lstlisting}
	<PharmML xmlns="http://www.pharmml.org/2013/03/PharmML"
    		xmlns:xsi="http://www.w3.org/2001/XMLSchema-instance"
    		...
    		writtenVersion="0.5" metadataFile="example3_NONMEM.rdf" id="i1">
    
    		<ct:Name>Testing new attribute</ct:Name>
    		...
\end{lstlisting}


%%%%%%%%%%%%%%%%%%%%%%%%%%%%%%%%%%%%%%%%%%%%%%%%%%%%%%%%%%%%%%%%%%%%%%
%%%%%%%%%%%%%%%%%%%%%%%%%%%%%%%%%%%%%%%%%%%%%%%%%%%%%%%%%%%%%%%%%%%%%%
%%%%%%%%%%%%%%%%%%%%%%%%%%%%%%%%%%%%%%%%%%%%%%%%%%%%%%%%%%%%%%%%%%%%%%

\chapter{Appendix A}
\label{sec:currentApproaches}

%%%%%%%%%%%%%%%%%%%%%%%%%%%%%%%%%%%%%%%%%%%%%%%%%%%%%%%%%%%%%%%%%%%%%%
\section{MLXTRAN PK macros}

PK macros offer an equation-free encoding solution for pharmacokinetic models. 
The novel system allows to implement in MLXTRAN virtually any compartmental 
model within well-defined constraints. See \cite{MLXTRANforMonolix:2014} for a detailed 
description. Tab.\ref{tab:MLXPLORElibrary} lists the available PK macros with their arguments. 

\begin{table}[ht!]
\begin{center}
\begin{tabular}{lll}
  \hline
  \hline
Component & Macro & Arguments \\
  \hline
Compartment 			& \textit{compartment} 	& cmt, amount, volume, concentration \\
Peripheral compartment	& \textit{peripheral}		& kij, k$\_$i$\_$j, amount, volume, concentration \\
Effect compartment 		& \textit{effect} 			& cmt, ke0, concentration \\
Absorption from a depot 	& \textit{depot} 			& type/adm, Tlag, p, target, Tk0, ka, Ktr/Mtt \\
Absorption for IV dose 	& \textit{iv} 			& cmt, type/adm, Tlag, p \\
0-order absorption 		& \textit{absorption}		& cmt, type/adm, Tlag, p, Tk0 \\[-.5ex]
					& or \textit{oral}			& \\
1st order absorption 	& \textit{absorption}		& cmt, type/adm, Tlag, p, ka, Ktr/Mtt \\[-.5ex]
					& or \textit{oral}			& \\
Linear elimination 		& \textit{elimination}		& cmt, volume, k, CL \\
MM elimination 		& \textit{elimination}		& cmt, volume, Km, Vm \\
Transfer 				& \textit{transfer}		& from, to, kt \\
  \hline
\end{tabular}
\caption{PK macros and their arguments as used in MLXPLORE, based on \cite{MLXTRANforMonolix:2014}.
Please, note that there are more component listed, 10, then macros, 8, which are actually defined in MLXTRAN
system. The presumed redundant definitions are meant to support the user in the use of the macros
by providing the relevant set of arguments for a given administration type.}
\label{tab:MLXPLORElibrary}
\end{center}
\end{table}

Consider, the following basic example in which three macros with their arguments correspond 
uniquely to a well-defined system of 2 ODE's, one for the depot compartment, one for the central compartment, 
respectively. 
\paragraph{Example 1}
1-comp, 1st order absorption, linear elimination
%oral(cmt=1, Tk0, p=Fr0) 
\lstset{language=NONMEMdataSet}
\begin{lstlisting}
	compartment(cmt=1,concentration=Cc,volume=V) 
	oral(cmt=1, ka) 
	elimination(cmt=1, k)
\end{lstlisting}

The advantages of this approach
is a very intuitive and efficient way to encode a PK model which is less error prone than when 
encoding the ODEs by hand. The system has been in use for several years now, since Monolix 
version 3.2, which provides a very valuable proof of concept.


\section{PREDPP library}
The best known PK library in the field is the PREDPP library used in NONMEM \cite{PREDPP:2011}. 
It contains  
customised 1, 2 or 3 compartmental PK models, such as ADVAN1-4 and 10-12, and 
general model routines, such as ADVAN 5-9 and 13, used to build PK models containing linear
and non-linear components and being of arbitrary complexity. 
Additionally available are so called translator routines, TRANS1-6, allowing for various 
parameterisations. Additional features such as Sn, Fn, Rn, Dn etc. provide means to define 
scaling, bioavailability, infusion rate and duration etc., respectively. Tab.\ref{tab:PREDPPlibrary} 
lists available components of the PREDPP library. 

%\subparagraph{Note}
As \marginpar{\HandCuffLeft}  powerful as this system seems to be, even for a very simple 1-compartmental oral 
model with saturable rather then linear elimination, the user must implement the ODE system 
using ADVAN6 or similar. It is therefore a hybrid solution offering only a small
number of ready-to-use models, and not a generic solution like the macro based systems.
%to be discussed below. 
%This limitations makes PREDPP unsuitable for a standard PK library, 
%see discussion in Section \ref{subsec:PKmacros}.


\begin{table}[ht!]
\begin{center}
\begin{tabular}{ll}
  \hline
  \hline
  Component & Description \\
  \hline
  ADVAN 1 	& 1-comp IV input \\
  ADVAN 2 	& 1-comp 1st order input \\
  ADVAN 10 	& 1-comp IV input with saturable elimination \\
  ADVAN 3 	& 2-comp IV input \\
  ADVAN 4 	& 2-comp 1st order input \\
  ADVAN 11 	& 3-comp IV input \\
  ADVAN 12 	& 3-comp 1st order input \\
  ADVAN 5 and 7 	& linear ODE models \\
  ADVAN 6, 8, 9 and 13 	& general ODE models \\
  TRANS1 		& ADVAN1: \{K\} \\[-.25ex]
			& ADVAN2: \{K, KA\} \\[-.25ex]
			& ADVAN3: \{K, K12, K21\} \\[-.25ex]
			& ADVAN4: \{K, K23, K32, KA\} \\[-.25ex]
			& ADVAN10: \{KM, VM\} \\[-.25ex]
			& ADVAN11: \{K, K12, K21, K13, K31\} \\[-.25ex]
			& ADVAN12: \{KA, K, K23, K32, K24, K42\} \\[-.25ex]
  TRANS2 		& ADVAN1: \{CL, V\} \\[-.25ex]
 			& ADVAN2: \{CL, V, KA\} \\
  TRANS3 		& ADVAN3: \{CL, V, Q, VSS\} \\[-.25ex]
 			& ADVAN4: \{CL, V, Q, VSS, KA\} \\
  TRANS4 		& ADVAN3: \{CL, V1, Q, V2\} \\[-.25ex]
 			& ADVAN4: \{CL, V2, Q, V3, KA\} \\
				& ADVAN6: \{ALPHA, BETA, K32, KA \} \\[-.25ex]
			& ADVAN11: \{CL, V1, Q2, V2, Q3, V3\} \\[-.25ex]
				& ADVAN12: \{CL, V2, Q3, V3, Q4, V4, KA \} \\[-.25ex]
  TRANS5 	& ADVAN3: \{AOB, ALPHA, BETA\} \\[-.25ex]
 				& ADVAN4: \{AOB, ALPHA, BETA, KA\} \\
  TRANS6		& ADVAN11: \{CL, V2, Q3, V3, Q4, V4, KA\} \\[-.25ex]
				& ADVAN12: \{ALPHA, BETA, GAMMA, K32, K42, KA \} \\[-.25ex]
  Sn			& Scale parameter for compartment 'n' \\
  Fn			& Bioavailability for compartment 'n' \\
  Rn			& Rate for compartment 'n' \\
  Dn			& Duration for compartment 'n' \\
  ALAGn		& Absorption lag for compartment 'n' \\
  A\_0(n)		& Compartment initialization for compartment 'n' \\
%  Xn			& X parameter for compartment 'n'\\
  On 			& Output fraction for compartment 'n' \\
  \hline
\end{tabular}
\caption{PREDPP library: Subroutines and parameters, based on \cite{Metrum:2012} and \cite{PREDPP:2011}.}
\label{tab:PREDPPlibrary}
\end{center}
\end{table}


%\newpage
%%%%%%%%%%%%%%%%%%%%%%%%%%%%%%%%%%%%%%%%%%%%%%%%%%%
%\section{PK macros}
%PK macros are a new option extending the standard encoding ways for PK models. 
%For example, a macro for a 1-comparmental IV model with linear elimination (corresponds to ADVAN1) reads
%\lstset{language=NONMEMdataSet}
%\begin{lstlisting}
%	compartment(cmt=1, amount=Ac, volume=V)
%	iv(adm=1, cmt=1)
%	elimination(cmt=1, k)
%\end{lstlisting}
%which encoded in the proposed PharmML reads
%\lstset{language=XML}
%\begin{lstlisting}
%        <StructuralModel blkId="sm3">
%            <ct:Variable symbolType="real" symbId="Ac"/>
%            
%            <PKmacros>
%                <!-- omitted Compartment macro -->
%                <IV>
%                    <Value argument="adm">
%                        <ct:Int>1</ct:Int>
%                    </Value>
%                    <Value argument="cmt">
%                        <ct:Int>1</ct:Int>
%                    </Value>
%                </IV>
%                <!-- omitted Elimination macro -->
%            </PKmacros>
%        </StructuralModel>
%\end{lstlisting}
%The following table shows how the mapping between the target, in this case the administration 
%number identified with the attribute \xatt{adm} works for \xelem{TrailDesing} and MONOLIX dataset.
%\begin{table}[h!]
%\setlength{\tabcolsep}{1pt}
%\begin{center}
%\begin{tabular}{ll}
%  \hline
%\xelem{Activity}/\xelem{TrialDesign}  	& MONOLIX dataset \\
%  \hline
%\lstset{language=XML}
%\begin{lstlisting}
%<Activity oid="bolusIV">
%    <Bolus>
%        <DoseAmount inputTarget="admType"> 
%            <TargetMapping blkIdRef="sm3">
%                <ds:Map admNumber="1"/>
%            </TargetMapping>
%            <ct:Assign>
%                <ct:Real>100</ct:Real>
%            </ct:Assign>
%        </DoseAmount>
%        <!-- omitted details -->
%    </Bolus>
%</Activity>        
%\end{lstlisting}
%&
%\lstset{language=XML}
%\begin{lstlisting}
%<MONOLIXdataSet oid="MLXoid">
%    
%    <ColumnMapping>
%        <ds:ColumnRef columnIdRef="ADM"/>
%        <ds:TargetMapping blkIdRef="sm1">
%            <ds:Map dataSymbol="1" admNumber="1"/>
%        </ds:TargetMapping>
%    </ColumnMapping>             
%    
%    <DataSet>
%    
%    <!-- omitted details -->
%\end{lstlisting}
%\\
%  \hline
%  \end{tabular}
%\vspace{-1.5em}
%\caption{Mapping rules for macro encoded PK model for \xelem{TrialDesing} and a MONOLIX dataset.}
%\label{tab:targetMapping3}
%\end{center}
%\end{table}




\bibliographystyle{plain}
\bibliography{pharmml-specification}
\end{document}





%TEMPLATES 
%% 1. Template for table with figures
%\begin{figure}[htbp]
%\centering
%\begin{tabular}{cc}
% \includegraphics[width=80mm]{pics/pic1} & 
% \includegraphics[width=80mm]{pics/pic2} \\
% \includegraphics[width=80mm]{pics/pic3} &
% \includegraphics[width=80mm]{pics/pic4}
%\end{tabular}
%\caption{about the figure}
%\label{figTable:labelText}
%\end{figure}

%\begin{table}[ht]
%\begin{center}
%\begin{tabular}{rrrrrrrrrrr}
%  \hline
% & 1 & 2 & 3 & 4 & 5 & 6 & 7 & 8 & 9 & 10 \\ 
%  \hline
%1 & 0.24 & -1.47 & -0.56 & 0.24 & 0.71 & 1.23 & 0.44 & 0.40 & 1.10 & 1.84 \\ 
%   \hline
%\end{tabular}
%\end{center}
%\end{table}
 

%\begin{figure}[htb!]
%\centering
%  \includegraphics[width=105mm]{}
% \caption{}
% \label{fig:myplot}
%\end{figure}

%PIECE-WISE
%f(z) =     \left\{ \begin{array}{rcl}
%         value1 & \mbox{for} & condition1 \\ 
%         value1 & \mbox{for} & condition1
%             \end{array}\right.


