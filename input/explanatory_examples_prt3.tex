%%%%%%%%%%%%%%%%%%%%%%%%%%%%%%%%%%%%%%%%%%%%%%%%%%%%%%%%%%%%%%%%
%%%%%%%%%%%%%%%%%%%%%%%%%%%%%%%%%%%%%%%%%%%%%%%%%%%%%%%%%%%%%%%%
%%%%%%%%%%%%%%%%%%%%%%%%%%%%%%%%%%%%%%%%%%%%%%%%%%%%%%%%%%%%%%%%

\eglabel{6}
\section{Example \theexamples: Joint PKPD model with count data}
\label{sec:eg6}
The following examples feature a new aspect of the \pml, the support of discrete 
data models\footnote{The discrete data models are coming in a large variety, which 
would be worth a detailed discussion. This and the next example cover only the very 
basic cases. More than 20 additional discrete 
date examples, encoded completely in \pml, can be found on our webpage \url{http://pharmml.org}. 
See also discrete data model templates in the appendix, \ref{chapter:codeTemplates}.}. 
The first one is using the Poisson distribution to describe count data\footnote{The example is encoded 
in \xatt{example6\_NONMEM.xml} with design sourced from a NONMEM datafile.}, 
from the MLXTRAN tutorial, \cite{Monolix4.3Tutorial:2014}.
%%%%%%%%%%%%%%%%%%%%%%%%%%%%%%%%%%%%%%%%%%%%%%%%%%%%%%%%%%%%%%%%
\subsection{Description}
\label{subsec:exp6_intro}
 
The essential bit of information for this task is the probability distribution, the probability 
mass function (PMF), of count data Y, which can be defined in either un-transformed, 
P(Y=k), or transformed, log(P(Y=k)), form. As in example \ref{sec:eg1}, the underlying 
PK model is 1-compartmental oral model and will be omitted here. We assume the basic 
Poisson model which reads
\begin{eqnarray}
&& P\big(Y_{\iijj}=k | \Cc_{\iijj}, \psi_i\big) =  \frac{e^{-\lambda_{\iijj}} \lambda^k_{\iijj}}{k!} \label{eq:poissonModel}
\end{eqnarray}
with concentration dependent mean $\lambda$, defined as
\begin{eqnarray}
&& \lambda_{\iijj} = \lambda_0 \Big(1 - \frac{\Cc_{\iijj}}{IC_{50} + \Cc_{\iijj}}\Big)  \label{eq:lambdasurface}
\end{eqnarray}
\begin{figure}[htbp]
\centering
%\begin{tabular}{cc}
\includegraphics[width=.43\textwidth]{pics/CTS4_lambda_threeSurfaces} 
%& \raisebox{0\height}{\includegraphics[scale=0.45]{pics/CTS4_poissonScan}}
%\end{tabular}
\caption{$\lambda$--surface as function of \emph{Cc}, $\lambda_0$ and $IC_{50}$ 
plotted for $\Cc = \{1,5,15\}$.}
\label{fig:lambdasurface}
\end{figure}
also called \emph{Poisson intensity}. Here, $\lambda$, depends on the parameters 
$\lambda_0$ and $IC_{50}$ which are sampled from log-normal distribution. 
$\lambda_0$, stands here for the baseline seizure count prior to any drug. 
The value of $\lambda$, parameterised with $IC_{50}$ and $\lambda_0$, is reduced 
by the concentration in the central compartment, $\Cc$, which is visualised for three different values 
of $\Cc = \{1,5,15\}$ in \ref{fig:lambdasurface} ($1\equiv$ green, $5 \equiv$ 
red, $15 \equiv$ blue).


\begin{figure}[ht!]
\centering
\begin{tabular}{cc}
\includegraphics[width=.49\textwidth]{pics/CTS4_PK_meanLambda_armB} & 
\raisebox{0.05\height}{\includegraphics[scale=0.4875]{pics/CTS4_lambda}}
\end{tabular}
\caption{(left) The time courses for the concentration and mean Poisson intensity, $\lambda$. 
(right) Poisson intensity as given by eq.(\ref{eq:lambdasurface}) for all subject 
as function of the concentration between 0 and the maximum achievable 
concentration across 20 subjects.}
\label{fig:lambdaTimeCourse}
\end{figure}

\subsubsection{Individual parameters model}
\begin{eqnarray}
\lambda_0 & \sim&  \mbox{logNormal}(\pop_{\lambda_0}, \omega_{\lambda_0}); \quad  \pop_{\lambda_0} = 5, \quad \omega_{\lambda_0} = 0.2 \nonumber \\
IC_{50} &\sim& \mbox{logNormal}(\pop_{IC_{50}}, \omega_{IC_{50}}); \quad  \pop_{IC_{50}} = 5, \quad \omega_{IC_{50}} = 0 \nonumber
\end{eqnarray}


\subsubsection{Observation model}
We apply a combined residual error model to the continuous PK output variable 
\var{Cc} and Poisson error distribution for the discrete PD component, defined by \var{Y}, 
as the following table shows 

%\begin{table*}[h!]
\begin{center}
\small
\renewcommand{\arraystretch}{1.1}% 
\begin{tabular*}{0.8\linewidth}{@{\extracolsep{\fill}} >{\bfseries}l l l}\toprule
Output Variable & \textbf{\itshape Cc} &\textbf{\itshape Y}\\\midrule
Observation Name & Concentration & State \\
Units & $\mg/l$ & -- \\
Type & Continuous & Discrete/Count \\
Model & Combined & Poisson\\
Parameters 	& $a, b$ 	& $\lambda_0, IC_{50}$\\
%Regressor	& --		& $Cc$ \\
\bottomrule
\end{tabular*}
\end{center}
Two additional important bits of information which are part of the discrete observation
model to be provided are 
\begin{itemize}
\item
Intensity Parameter, $\lambda$, given by eq. (\ref{eq:lambdasurface})
\item
Link function -- $\log$.
\end{itemize}

%%%%%%%%%%%%%%%%%%%%%%%%%%%%%%%%%%%%%%%%%%%%%%%%%%%%%%%%%%%%%%%%
\subsubsection{Modelling Steps}
Additionally to an estimation task (not described further), the PK and PD output variables are 
to be generated by the simulation with their associated time points are shown below:

\begin{center}
\small
\renewcommand{\arraystretch}{1.1}% 
\begin{tabular*}{0.9\linewidth}{@{\extracolsep{\fill}} >{\bfseries}l c c}\toprule
Output Variable & \textbf{\itshape Cc} &\textbf{\itshape $\lambda$}\\\midrule
Observation times & [0.5, 4 : 4 : 48, 52 : 24 : 192, 196 : 4 : 250] & 0 : 24 : 192\\
\bottomrule
\end{tabular*}
\end{center}
See Figure \ref{fig:lambdaTimeCourse} for a typical result plots for a group of subjects.

%%%%%%%%%%%%%%%%%%%%%%%%%%%%%%%%%%%%%%%%%%%%%%%%%%%%%%%%%%%%%%%%
\subsection{Observation model}
While the continuous observation model and its features have been described 
in very detail in the previous examples, the error distribution of the discrete effect 
given by eqs.(\ref{eq:poissonModel}) 
and (\ref{eq:lambdasurface}) will be described in the following for the first time.

First we have to indicate the type of the observation model using the elements 
\xelem{Discrete} and specifically for this example the \xelem{CountData}. The next 
mandatory elements are the count index, \emph{k}, (required only when the PMF is 
explicitly specified, see below) and the \xelem{CountVariable} which will be used later
to establish the link between the model and the related date set, see Table 
\ref{tab:example6_dataSet}.
Then the characteristic parameter for the Poisson distribution as function of the drug 
concentration, $Cc$, encoded as \emph{Lambda}, is implemented using \xelem{IntensityParameter}
as the following snippet shows
\lstset{language=XML}
\begin{lstlisting}
        <ObservationModel blkId="om1">
            <Discrete>
                <CountData>
                    <ct:Variable symbolType="int" symbId="k"/>
                    <CountVariable symbId="Y"/>
                    
                    <!-- Poisson intensity - function of drug concentration, Cc -->                    
                    <IntensityParameter symbId="Lambda">
                        <ct:Assign>
                            <math:Equation>
                                <math:Binop op="times">
                                    <ct:SymbRef blkIdRef="pm1" symbIdRef="lambda0"/>
                                    <math:Binop op="minus">
                                        <ct:Real>1</ct:Real>
                                        <math:Binop op="divide">
                                            <ct:SymbRef blkIdRef="sm1" symbIdRef="Cc"/>
                                            <math:Binop op="plus">
                                                <ct:SymbRef blkIdRef="pm1" symbIdRef="IC50"/>
                                                <ct:SymbRef blkIdRef="sm1" symbIdRef="Cc"/>
                                            </math:Binop>
                                        </math:Binop>
                                    </math:Binop>
                                </math:Binop>
                            </math:Equation>
                        </ct:Assign>
                    </IntensityParameter>
                    <!-- see next listing for the continuation of the observation model -->
\end{lstlisting}
Now the Poisson probability mass function (PMF) can be defined, but here 
in the transformed format 
\begin{align}
\log(P\big(Y_{\iijj} & =k | \Cc_{\iijj}, \psi_i\big)) =  -\lambda_{\iijj} +k\times \lambda_{\iijj} - \log(k!) \nonumber
\end{align}
which can be done in various ways, either
\begin{itemize}
\item
using the UncertML standard and referring to \emph{Lambda} as in the following snippet
\lstset{language=XML}
\begin{lstlisting}
                    <PMF linkFunction="log">
                        <PoissonDistribution xmlns="http://www.uncertml.org/3.0" 
                            definition="http://www.uncertml.org/3.0">
                            <rate>
                                <var varId="Lambda"/>
                            </rate>
                        </PoissonDistribution>
                    </PMF>
                </CountData>
            </Discrete>
        </ObservationModel>
\end{lstlisting}
with \xatt{linkFunction} attribute indicating the logarithmic transformation, or 
\item
by encoding explicitly the PMF in the transformed format and specifying as before
the applied link function, as the following snippet shows
\lstset{language=XML}
\begin{lstlisting}
                    <PMF linkFunction="log">
                        <math:LogicBinop op="eq">
                            <ct:SymbRef symbIdRef="Y"/>
                            <ct:SymbRef symbIdRef="k"/>
                        </math:LogicBinop>
                        <ct:Assign>
                            <Equation xmlns="http://www.pharmml.org/pharmml/0.6/Maths">
                                <Binop op="minus">
                                    <Binop op="plus">
                                        <Uniop op="minus">
                                            <ct:SymbRef symbIdRef="Lambda"/>
                                        </Uniop>
                                        <Binop op="times">
                                            <ct:SymbRef symbIdRef="k"/>
                                            <Uniop op="log">
                                                <ct:SymbRef symbIdRef="Lambda"/>
                                            </Uniop>
                                        </Binop>
                                    </Binop>
                                    <Uniop op="factln">
                                        <ct:SymbRef symbIdRef="k"/>
                                    </Uniop>
                                </Binop>
                            </Equation>
                        </ct:Assign>
                    </PMF>
                </CountData>
            </Discrete>
        </ObservationModel> 
\end{lstlisting}
\end{itemize}

Note, that although the UncertML driven solution is very simple and straightforward
to implement, it is also limited to only this case, see also Section \ref{subsec:DiscreteData}. 
In short, for count data models, only the basic Poisson model is available in the UncertML standard.


%%%%%%%%%%%%%%%%%%%%%%%%%%%%%%%%%%%%%%%%%%%%%%%%%%%%%%%%%%%%%%%%
\subsection{NONMEM dataset}
\label{sec:eg6-NONMEMdataset}
The remaining part is the the data and trial design as sourced from the 
NONMEM dataset (using dummy values). Table \ref{tab:example6_dataSet} show a typical dataset required for 
an estimation task.
\begin{table}[htdp]
\begin{center}
\small
\renewcommand{\arraystretch}{1.1}% 
\begin{tabular}{rrrrrrrr}\toprule
ID 	& TIME	& AMT	& Y		& DVID \\ \midrule
1 	& 0 		& 0.25 	& . 		& . \\ 
1 	& 4 		& . 		& 19.6 	& 1 \\ 
1 	& 8 		& . 		& 16 	& 2 \\ 
1 	& 12 	& . 		& 9.5 	& 1 \\ 
1 	& 18 	& . 		& 3.4 	& 1 \\ 
1 	& 24 	& . 		& 2 		& 2 \\ 
2 	& 0 		& 0.25	& 16 	& . \\ 
2 	& 4 		& . 		& 14.8 	& 1 \\ 
2 	& 8 		& . 		& 6 		& 2 \\ 
2 	& 12 	& . 		& 8.1 	& 1 \\ 
2 	& 18 	& . 		& 2.2 	& 1 \\ 
2 	& 24 	& . 		& 0 		& 2 \\ 
...	& ...		& ...		& ...		& ...	\\ \bottomrule
\end{tabular}
\end{center}
\caption{A dataset used in example for first two subjects.
The additional column DVID is used to specify the type of data. Here, 
DVID = 1 is used for a continuous response and DVID = 2 for count data.}
\label{tab:example6_dataSet}
\end{table}%

\lstset{language=XML}
\begin{lstlisting}
        <mstep:ExternalDataSet toolName="NONMEM" oid="NMoid">
            <mstep:ColumnMapping>
                <ds:ColumnRef columnIdRef="TIME"/>
                <ct:SymbRef symbIdRef="t"/>
            </mstep:ColumnMapping>
            <mstep:ColumnMapping>
                <ds:ColumnRef columnIdRef="AMT"/>
                <ct:SymbRef blkIdRef="sm1" symbIdRef="Ad"/>
            </mstep:ColumnMapping>
            <mstep:MultipleDVMapping>
                <ds:ColumnRef columnIdRef="DV"/>
                <mstep:Piecewise>
                    <math:Piece>
                        <ct:SymbRef blkIdRef="om1" symbIdRef="Y"/>
                        <math:Condition>
                            <math:LogicBinop op="eq">
                                <ds:ColumnRef columnIdRef="DVID"/>
                                <ct:Int>2</ct:Int>
                            </math:LogicBinop>
                        </math:Condition>
                    </math:Piece>
                    <math:Piece>
                        <ct:SymbRef blkIdRef="om2" symbIdRef="Cc_obs"/>
                        <math:Condition>
                            <math:LogicBinop op="eq">
                                <ds:ColumnRef columnIdRef="DVID"/>
                                <ct:Int>1</ct:Int>
                            </math:LogicBinop>
                        </math:Condition>
                    </math:Piece>
                </mstep:Piecewise>
            </mstep:MultipleDVMapping>
            <ds:DataSet>
                <ds:Definition>
                    <ds:Column columnId="ID" columnType="id" valueType="string" columnNum="1"/>
                    <ds:Column columnId="TIME" columnType="idv" valueType="real" columnNum="2"/>
                    <ds:Column columnId="AMT" columnType="dose" valueType="real" columnNum="3"/>
                    <ds:Column columnId="DV" columnType="dv" valueType="real" columnNum="4"/>
                    <ds:Column columnId="DVID" columnType="dvid" valueType="real" columnNum="5"/>
                </ds:Definition>
                <ds:ExternalFile oid="dataOid">
                    <ds:path>datasets/example_poisson.csv</ds:path>
                </ds:ExternalFile>
            </ds:DataSet>
        </mstep:ExternalDataSet>
\end{lstlisting}

Mapping of dosing related data has been described previously on
multiple occasions and will not be discussed here. The conditional mapping required 
in the case when multiple observations have to be mapped using the \xelem{mstep:MultipleDVMapping}
has been described in example 1, see section \ref{sec:eg1-NONMEMdataset}. 
What is new however is the mapping target when dealing with count data.
The count variable, \emph{Y}, as defined in the very beginning of the \xelem{CountData} element
in the observation model \xatt{om1} with
\lstset{language=XML}
\begin{lstlisting}
		<CountVariable symbId="Y"/>
\end{lstlisting}
is the target for this conditional mapping. The observation column, DV, is mapped to $y$ if DVID=2 and
to \emph{Cc\_obs} if DVID=1.




%%%%%%%%%%%%%%%%%%%%%%%%%%%%%%%%%%%%%%%%%%%%%%%%%%%%%%%%%%%%%%%%
%%%%%%%%%%%%%%%%%%%%%%%%%%%%%%%%%%%%%%%%%%%%%%%%%%%%%%%%%%%%%%%%
%%%%%%%%%%%%%%%%%%%%%%%%%%%%%%%%%%%%%%%%%%%%%%%%%%%%%%%%%%%%%%%%

%\newpage

\eglabel{7}
\section{Example \theexamples: Joint PKPD model with categorical data}
\label{sec:eg7}

This last example features another type of discrete models supported by \pml -- 
the categorical data models.

%%%%%%%%%%%%%%%%%%%%%%%%%%%%%%%%%%%%%%%%%%%%%%%%%%%%%%%%%%%%%%%%
\subsection{Description}
\label{subsec:exp7_intro} 

In this example, there are two categories, $k \in {0,1}$, i.e. the effect outcome is either 0 or 1. 
The underlying PK model is again an 1-compartmental oral model and will not be described
again. The probability for the category 1 is given by the following formula 
\begin{eqnarray}
	p1 &=& \frac{1}{1+\exp(-\theta_1 - \theta_2 \log(\Cc))} 	\label{eq:p1surface}
\end{eqnarray}
which is plotted in \ref{fig:p1surface}. This $p1$--surface is also function of $\theta_1$, 
$\theta_2$ and $log(\Cc)$ which is visualised for three different values of $\Cc = \{1,5,15\}$. 

\begin{figure}[htbp]
\begin{center}
\includegraphics[width=.45\textwidth]{pics/p1_threeSurfaces.png}
\caption{p1 probability surface as function of $\theta_1$ and $\theta_2$ plotted for 
Cc = \{1,5,15\} ($1\equiv$ green, $5 \equiv$ red, $15 \equiv$ blue). }
\label{fig:p1surface}
\end{center}
\end{figure}

\subsubsection{Individual parameters model}
\begin{eqnarray}
theta1& \sim&  \mbox{Normal}(\pop_{theta1}, \omega_{theta1}); \quad \pop_{theta1}=-1,\quad \omega_{theta1}=0.3 \nonumber \\
theta2& \sim&  \mbox{logNormal}(\pop_{theta2}, \omega_{theta2}); \quad \pop_{theta2}=1,\quad \omega_{theta2}=0.2 \nonumber 
\end{eqnarray}

\subsubsection{Observation model}
We apply a combined residual error model to the continuous PK output variable 
\var{Cc} and Poisson error distribution for the discrete PD component, defined by \var{Y}, 
as the following table shows 

%\begin{table*}[h!]
\begin{center}
\small
\renewcommand{\arraystretch}{1.1}% 
\begin{tabular*}{0.8\linewidth}{@{\extracolsep{\fill}} >{\bfseries}l l l}\toprule
Output Variable & \textbf{\itshape Cc} &\textbf{\itshape Y}\\\midrule
Observation Name & Concentration & State \\
Units & $\mg/l$ & -- \\
Type & Continuous & Discrete/Categorical \\
Model & Combined & Binomial\\
Parameters 	& $a, b$ 	& $\theta_1$, $\theta_2$\\
%Regressor	& --		& \emph{Cc} \\
\bottomrule
\end{tabular*}
\end{center}

%%%%%%%%%%%%%%%%%%%%%%%%%%%%%%%%%%%%%%%%%%%%%%%%%%%%%%%%%%%%%%%%
\subsubsection{Modelling Steps}
Apart from an estimation task (not described further), the PK and PD output variables are to be 
generated by the simulation with their associated time points are shown below:
\begin{center}
\small
\renewcommand{\arraystretch}{1.1}% 
\begin{tabular*}{0.9\linewidth}{@{\extracolsep{\fill}} >{\bfseries}l c c}\toprule
Output Variable & \textbf{\itshape Cc} &\textbf{\itshape $p1$}\\\midrule
Observation times & [0.5, 4 : 4 : 48, 52 : 24 : 192, 192 : 4 : 250] & 0 : 24 : 288\\
\bottomrule
\end{tabular*}
\end{center}
See Figure \ref{fig:p1timecourse} for a typical result plots for a group of subjects.

\begin{figure}[htbp]
\centering
\begin{tabular}{cc}
\includegraphics[width=.6\textwidth]{pics/p1_armA} 
\end{tabular}
\caption{Plots for first arm in the study. (top) Concentration, \emph{Cc}, time course of the PK model. 
(bottom) Probability, \emph{p1}, time course as defined in eq.\ref{eq:p1surface}.}
\label{fig:p1timecourse}
\end{figure}


%%%%%%%%%%%%%%%%%%%%%%%%%%%%%%%%%%%%%%%%%%%%%%%%%%%%%%%%%%%%%%%%
\subsection{Observation model}
The \xelem{ObservationModel} for count data is the new element in this example.
There are different ways to define such model and we present one of those.\footnote{For 
an alternative encoding see examples on our website \url{http://pharmml.org}.}
Because we are dealing with nominal categorical data, we store this information 
using the attribute \xatt{ordered}. Next we encode the probability for category $1$ 
as defined by equation \ref{eq:p1surface}, as the following example shows, which 
will be used later in the definition of the PMF.
\lstset{language=XML}
\begin{lstlisting}
        <ObservationModel blkId="om1">
            <Discrete>
                <CategoricalData ordered="no">
                    
                    <ct:Variable symbolType="real" symbId="p1">
                        <ct:Assign>
                            <math:Equation>
                                <math:Binop op="divide">
                                    <ct:Real>1</ct:Real>
                                    <math:Binop op="plus">
                                        <ct:Real>1</ct:Real>
                                        <math:Uniop op="exp">
                                            <math:Binop op="minus">
                                                <math:Uniop op="minus">
                                                    <ct:SymbRef blkIdRef="pm1" symbIdRef="theta1"/>
                                                </math:Uniop>
                                                <math:Binop op="times">
                                                    <ct:SymbRef blkIdRef="pm1" symbIdRef="theta2"/>
                                                    <math:Uniop op="log">
                                                        <ct:SymbRef blkIdRef="sm1" symbIdRef="Cc"/>
                                                    </math:Uniop>
                                                </math:Binop>
                                            </math:Binop>
                                        </math:Uniop>
                                    </math:Binop>
                                </math:Binop>
                            </math:Equation>                            
                        </ct:Assign>
                    </ct:Variable>
\end{lstlisting}
Next the categories are encoded withn \xelem{ListOfCategories}, the 
\xelem{CategoryVariable}, \emph{y}. At last, the aforementioned probability mass function 
using UncertML is specified, i.e. the binomial distribution with the probability of success equal 
$p1$ specified above, as the following code snippet shows  
\lstset{language=XML}
\begin{lstlisting}
                    <ListOfCategories> 
                        <Category symbId="cat0"/>
                        <Category symbId="cat1"/>
                    </ListOfCategories>
                    
                    <CategoryVariable symbId="y"/>
                    
                    <PMF linkFunction="identity">
                        <BinomialDistribution xmlns="http://www.uncertml.org/3.0" definition="">
                            <numberOfTrials>
                                <nVal>1</nVal>
                            </numberOfTrials>
                            <probabilityOfSuccess>
                                <var varId="p1"/>
                            </probabilityOfSuccess>
                        </BinomialDistribution>
                    </PMF>
                </CategoricalData>
            </Discrete>
        </ObservationModel>
\end{lstlisting}
This is because UncertML does not allow the encoding of assignments within their elements, such as 
\xelem{probabilityOfSuccess}, so that we first have to define $p1$ and used its reference in the 
PMF definition.

%%%%%%%%%%%%%%%%%%%%%%%%%%%%%%%%%%%%%%%%%%%%%%%%%%%%%%%%%%%%%%%%
\subsection{NONMEM dataset}
\label{sec:eg7-NONMEMdataset}
The remaining part is the data and trial design as sourced from the 
NONMEM dataset (with dummy values). Table \ref{tab:example7_dataSet} show a typical dataset required for 
an estimation task.
\begin{table}[htdp]
\begin{center}
\small
\renewcommand{\arraystretch}{1.1}% 
\begin{tabular}{rrrrrrrr}\toprule
ID 	& TIME	& AMT	& Y		& DVID \\ \midrule
1 	& 0 		& 0.25 	& . 		& . \\ 
1 	& 1 		& . 		& 1	 	& 2 \\ 
1 	& 4 		& . 		& 2.9 	& 1 \\ 
1 	& 8 		& . 		& 1 		& 2 \\ 
1 	& 10		& . 		& 0	 	& 2 \\ 
1 	& 12 	& . 		& 8.5 	& 1 \\ 
1 	& 18 	& . 		& 6.4 	& 1 \\ 
1 	& 24 	& . 		& 0 		& 2 \\ 
2 	& 0 		& 0.25	& 6.1 	& . \\ 
2 	& 4 		& . 		& 8.5 	& 1 \\ 
2 	& 8 		& . 		& 0 		& 2 \\ 
2 	& 12 	& . 		& 1.1 	& 1 \\ 
2 	& 18 	& . 		& 5.3 	& 1 \\ 
2 	& 24 	& . 		& 1 		& 2 \\ 
...	& ...		& ...		& ...		& ...	\\ \bottomrule
\end{tabular}
\end{center}
\caption{A dataset used in example for first two subjects.
The column DVID is used to specify the type of data. Here, 
DVID = 1 is used for a continuous response and DVID = 2 for categorical data, 
with only two values allowed, either 0 for \xatt{cat0} or 1 for \xatt{cat1}.}
\label{tab:example7_dataSet}
\end{table}%

%%%%%%%%%%%%%%%%%%%%%%%%%%%%%%%%%%%%%%%%%%%%%%%%%%%%%%%%%%%%%%%%
\paragraph{\emph{Nested} mapping}
Because of the fact that we are dealing with a joint PK/PK model, we have to 
make use of the \xelem{MultipleDVMapping} element described in examples before.
However, due to the fact that we are dealing with a categorical observed variable, 
whose categories have to matched to their symbol identifiers as defined in the observation
model, we need additional \xelem{CategoryMapping} as shown in the following 
code snippet. 
\lstset{language=XML}
\begin{lstlisting}
        <mstep:ExternalDataSet toolName="NONMEM" oid="NMoid">
            
            <mstep:ColumnMapping>
                <ds:ColumnRef columnIdRef="TIME"/>
                <ct:SymbRef symbIdRef="t"/>
            </mstep:ColumnMapping>
            <mstep:ColumnMapping>
                <ds:ColumnRef columnIdRef="AMT"/>
                <ct:SymbRef blkIdRef="sm1" symbIdRef="Ad"/>
            </mstep:ColumnMapping>
            <mstep:MultipleDVMapping>
                <ds:ColumnRef columnIdRef="DV"/>
                <mstep:Piecewise>
                    <math:Piece>
                        <ct:SymbRef blkIdRef="om1" symbIdRef="y"/>
                        <math:CategoryMapping>
                            <ds:Map dataSymbol="0" modelSymbol="cat0"/>
                            <ds:Map dataSymbol="1" modelSymbol="cat1"/>
                        </math:CategoryMapping>
                        <math:Condition>
                            <math:LogicBinop op="eq">
                                <ds:ColumnRef columnIdRef="DVID"/>
                                <ct:Int>2</ct:Int>
                            </math:LogicBinop>
                        </math:Condition>
                    </math:Piece>
                    <math:Piece>
                        <ct:SymbRef blkIdRef="om2" symbIdRef="C_obs"/>
                        <math:Condition>
                            <math:LogicBinop op="eq">
                                <ds:ColumnRef columnIdRef="DVID"/>
                                <ct:Int>1</ct:Int>
                            </math:LogicBinop>
                        </math:Condition>
                    </math:Piece>
                </mstep:Piecewise>
            </mstep:MultipleDVMapping>
            
            <!-- Dataset omitted, identical as in previous example -->
        </mstep:ExternalDataSet>
\end{lstlisting}
In this \emph{nested} mapping we map the symbols used in the data set,
e.g. \xatt{dataSymbol="0"} with category symbols, \xatt{modelSymbol="cat0"},
of the variable \emph{y}.


%%%%%%%%%%%%%%%%%%%%%%%%%%%%%%%%%%%%%%%%%%%%%%%%%%%%%%%%%%%%%%%%
\section{More examples}
A number of additional continuous and discrete data, PharmML coded, model examples is available on our website, 
such as
\begin{itemize}
\item
Basic and complex PK macros -- 12 examples, among others models corresponding to seven predefined 
PREDPP models, i.e. ADVAN1-4, 10-12 \cite{NONMEM:2009}.
\item
Discrete data models -- 22 examples for 
\begin{itemize}
\item
count data models, e.g. zero-inflated, negative binomial, generalized Poisson, Poisson Mixture model. 
\item
categorical data models, e.g. nominal and ordinal regression models such as proportional odds or models with 
discrete- and continuous-time Markov dependency and 
\item
time-to-event data models, e.g. constant or concentration dependent hazard, Weibull model etc.
\end{itemize}
\item
DDEs -- 3 examples (e.g. arthritis and SEIRS models \cite{MLXTRANforMonolix:2014}).
\end{itemize}


%%%%%%%%%%%%%%%%%%%%%%%%%%%%%%%%%%%%%%%%%%%%%%%%%%%%%%%%%%%%%%%%
%%%%%%%%%%%%%%%%%%%%%%%%%%%%%%%%%%%%%%%%%%%%%%%%%%%%%%%%%%%%%%%%
%%%%%%%%%%%%%%%%%%%%%%%%%%%%%%%%%%%%%%%%%%%%%%%%%%%%%%%%%%%%%%%%

%\newpage
%
%\eglabel{8}
%\section{Example \theexamples: Joint PKPD model with Time-to-event effect}
%\label{sec:eg8}
%
%%%%%%%%%%%%%%%%%%%%%%%%%%%%%%%%%%%%%%%%%%%%%%%%%%%%%%%%%%%%%%%%%
%\subsection{Description}
%\label{subsec:exp8_intro} 
%Time-to-effect is yet another type of discrete models supported by \pml, see Section 
%\ref{subsec:DiscreteData} for detailed description.
%In this example we consider a concentration dependent hazard function for which 
%the cumulative hazard reads
%
%\begin{eqnarray}
%\frac{dCh}{dt} = hazard = \beta \times Cc(t) \quad \Longrightarrow \quad Ch(t) = \beta \int_0^{t} Cc(\tilde{t}) \;d\tilde{t} = \beta \times AUC(Cc(0,t))  \label{eq:hazardODE1}
%\end{eqnarray}
%The survival function reads, eq.\eqref{eq:survivalFct},
%\begin{eqnarray}
%S(t) = \exp(-Ch(t)) = \exp(- \beta \times AUC(Cc(0,t))) \label{eq:survivalFct}
%\end{eqnarray}
%
%\begin{figure}[htb!]
%\centering
%\begin{tabular}{cc}
% \includegraphics[width=70mm]{pics/example8_singleCc} & 
% \includegraphics[width=70mm]{pics/example8_singleCh} \\
% \includegraphics[width=70mm]{pics/example8_singleS} &
%\end{tabular}
%\caption{Single subject time course for concentration, \emph{Cc}, cumulative hazard, \emph{Ch}, 
%and survival function, \emph{S}.}
%\end{figure}
%
%
%\subsubsection{Individual parameters model}
%\begin{eqnarray}
%beta& \sim&  \mbox{logNormal}(pop_{beta}, \omega_{beta}); \quad \pop_{beta}=0.13,\quad \omega_{beta}=0.2 \nonumber
%\end{eqnarray}
%
%\subsubsection{Observation model}
%We apply a combined residual error model to the continuous PK output variable 
%\var{Cc} and Poisson error distribution for the discrete PD component, defined by \var{Y}, 
%as the following table shows 
%
%%\begin{table*}[h!]
%\begin{center}
%\small
%\renewcommand{\arraystretch}{1.1}% 
%\begin{tabular*}{0.8\linewidth}{@{\extracolsep{\fill}} >{\bfseries}l l l}\toprule
%Output Variable & \textbf{\itshape Cc} &\textbf{\itshape Y}\\\midrule
%Observation Name & Concentration & State \\
%Units & $\mg/l$ & -- \\
%Type & Continuous & Discrete/Categorical \\
%Model & Combined & Binomial\\
%Parameters 	& $a, b$ 	& $\theta_1$, $\theta_2$\\
%%Regressor	& --		& $Cc$ \\
%\bottomrule
%\end{tabular*}
%\end{center}
%
%%%%%%%%%%%%%%%%%%%%%%%%%%%%%%%%%%%%%%%%%%%%%%%%%%%%%%%%%%%%%%%%%
%\subsubsection{Modelling Steps}
%
%The PK and PD output variables to be generated by the simulation and
%their associated time points are shown below:
%
%\begin{center}
%\small
%\renewcommand{\arraystretch}{1.1}% 
%\begin{tabular*}{0.9\linewidth}{@{\extracolsep{\fill}} >{\bfseries}l c c}\toprule
%Output Variable & \textbf{\itshape Cc} &\textbf{\itshape $p1$}\\\midrule
%Observation times & [0.5,4 : 4 : 48, 52 : 24 : 192, 192 : 4 : 250] & 0 : 24 : 288\\
%\bottomrule
%\end{tabular*}
%\end{center}
%
%
%%%%%%%%%%%%%%%%%%%%%%%%%%%%%%%%%%%%%%%%%%%%%%%%%%%%%%%%%%%%%%%%%
%\subsection{Observation model}
%
%\lstset{language=XML}
%\begin{lstlisting}
%
%\end{lstlisting}
%
%%\begin{figure}
%%\centering
%%\begin{tabular}{cc}
%% \includegraphics[width=80mm]{pics/example8_PK1_PK2} &
%% \includegraphics[width=80mm]{pics/example8_hazardPD_S}
%%\end{tabular}
%%\caption{PK as calculated from the PK model (top left) and with residual error model (bottom left). Hazard and survival function plot, here for arm A (right.)}
%%\end{figure}
%
%
%%%%%%%%%%%%%%%%%%%%%%%%%%%%%%%%%%%%%%%%%%%%%%%%%%%%%%%%%%%%%%%%%
%\subsection{NONMEM dataset and mappings}
%\label{sec:eg8-NONMEMdataset}
%The remaining part is the the data and trial design as sourced from the 
%NONMEM dataset. The mapping is identical to the previous examples, with the 
%


