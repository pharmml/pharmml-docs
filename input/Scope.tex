
\chapter{Scope of \pharmml}
\label{chap:scope}


\section{Introduction}

The scope of pharmacometric models is very wide, as such models can be empirical as well as
mechanistic; describe continuous as well as discrete data types; and be deterministic, stochastic
or a mixture of both. It is a challenging endeavour to accommodate this variety of possibilities
under one computational standard, therefore it is indispensable to split the task into multiple
steps of subsequent specifications and to define precisely the scope of every release.

In this chapter we define the scope of the functionality in
\pharmml, i.e. what information a \pharmml document can
represent. As is common practise in software engineering we have 
described the functionality as a set of ``features''.  It is important to remember 
that this specification is the second public release of \pharmml and over remaining 
time we expect additional features to be provided by future releases of the language.

The language is organised into three sections, which we believe naturally
describe the logical organisation of a pharmacometric model and its
associated tasks. Consequently we have grouped the features to match this
organisation. The sections are:
\begin{description}
\item[Model Definition] A description of the model, incl.\ the structural model, the model
  parameters, relevant covariates, the variability components, and the observations.
\item[Trial Design] A description of the design of a clinical trial associated with the model
  (for example a trial from which data is available to estimate the parameters of the model or
  a trial to be simulated with the model).
\item[Modelling Steps] A description of steps or tasks performed with
  the model. Typically this describes how the model has been used, for
  example to estimate its parameters or to perform a simulation.
\end{description}

\section{Model Definition}
\subsection{Structural Model}

\pharmml can encode:
\begin{itemize}
\item{A structural model defined by a set of algebraic equations.} Typically the explicit solution to a simple PK model, or a dose-response model.
\item{A structural model defined by a system of ODEs with initial conditions.} 
%\item{A structural model defined in \sbml format.}
\item{Delay Differential Equations (DDEs).}
\item{Compartmental PK models implemented using PK macros.}
\end{itemize}

\subsection{Covariate Model}

\pharmml can encode:
\begin{itemize}
\item{Continuous covariates.} These can be sampled from a probability distribution, 
used with an applied transformation and interpolated.
\item{Categorical covariates.} These can also be sampled from a probability distribution.
\end{itemize}

\subsection{Parameter Model}

\pharmml can encode:
\begin{itemize}
\item{Gaussian models with linear or non-linear covariate model.}
\item{General, equation based model such as those described in \cite{Keizer:2011aa}.}
\item{Random effects at arbitrary levels of variability.}
\item{Correlation of the random effects, described by a correlation or covariance matrix.}
\end{itemize}

\subsection{Variability Model}

\pharmml supports the following levels of variability:
\begin{itemize}
\item{Between-Subject Variability (BSV).} Aka inter-individual variability (IIV).
\item{Inter-Occasion Variability (IOV).} Such as within-subject variability.
\item{Higher levels of variability above BSV.} Such as variability between countries or centres.
\item{Lower levels of variability below IOV.} Such as variability between sub-occasions within occasions.
\end{itemize}

\subsection{Observations Model}

\pharmml supports the following observation model:
\begin{itemize}
\item{Continuous data models.} A residual error model applied to one or more observation variables..
%\item{Autocorrelation of the residual errors in a continuous observation model.}
\item{Discrete data models such as}
\begin{itemize}
\item{Count data models.} Such as Poisson, negative binomial, zero-inflated Poisson models etc.
\item{Nominal and ordered categorical models.} Logistic regression, proportional odds models etc.
\item{Time-to-event models.}
\end{itemize}
\end{itemize}

\section{Trial Design}

\subsection{Dataset based}

\pharmml supports external NONMEM/Monolix datasets which can be sourced 
to provide the full study design and the according experimental records. 
 
\subsection{\xelem{TrialDesign} based}

\pharmml can encode the following features of a trial design \emph{explicitly}:
\begin{itemize}
\item{Bolus and infusion dosing.}
\item{Multiple dosing regimens including mixed bolus and infusion.}
\item{Single, Repeated and Steady state dosing.}
\item{Dosing to more than one compartment.}
\item{Simple, parallel and cross-over designs.}
\item{Washout periods.}
\item{Occasions -- defined by time interval within a treatment epoch.}
\item{Trials with different centres or other levels of organisation above study arms}
\end{itemize}

\section{Modelling Steps}

\pharmml can encode the following features related to the task(s) associated with a model:
\begin{itemize}
\item{Estimation utilising the maximum likelihood principle.}
\item{Simulation of the model.}
\end{itemize}

\section{General}

\begin{description}
\item{Metadata annotation} Provides support to enable metadata
descriptions of the \pharmml document.
\item{Extension mechanism} Provides support to enable the extension of
the \pharmml document.
\end{description}

